\section{Руководство пользователя}
\subsection{Регистрация пользователя}
Перед работой с сервисом пользователю необходимо пройти процедуру регистрации. Для того, чтобы это сделать, необходимо на странице авторизации нажать кнопку регистрации (рисунок \ref{login-1}).
\addimghere{login}{1}{Страница авторизации с возможностью перейти к регистрации}{login-1}
В появившуюся форму (рисунок \ref{registration}) необходимо ввести все данные. Пароль необходимо выбирать надёжный, состоящий не менее чем из 6 символов и содержать как минимум одну прописную букву, одну строчную букву, одну цифру и один специальный символ.
Адрес электронной почты должен быть действующим.
\addimghere{registration}{1}{Регистрация пользователя}{registration}
После отправки данных для регистрации, на почту придёт письмо с ссылкой (рисунок \ref{confirmation}), перейдя по которой пользователь подтвердит регистрацию.
\addimghere{confirmation}{1}{Подтверждение регистрации пользователя}{confirmation}
\subsection{Авторизация пользователя}
После успешного прохождения процесса регистрации, пользователь может авторизироваться, воспользовавшись формой авторизации (рисунок \ref{login}).
\addimghere{login}{1}{Авторизация пользователя}{login}
В случае, если пользователь забыл пароль, необходимо с формы авторизации перейти на форму восстановления пароля с помощью предназначенной для этого кнопки.
После нажатия откроется окно восстановления пароля (рисунок \ref{reset-password}), в которое необходимо ввести адрес электронной почты, после чего на почту придёт письмо с ссылкой для подтверждения восстановления пароля.
Перейдя по ссылке, пользователь при необходимости может указать новый пароль.
\addimghere{reset-password}{1}{Восстановление пароля}{reset-password}
\subsection{Работа с фотографиями}
Для загрузки одной или нескольких фотографий на сервис, необходимо открыть страницу для загрузки фотографий и перетащить фотографии на кнопку загрузки множества фотографий, или одну фотографию на кнопку загрузки одной фотографии.
После выбора необходимых фотографий требуется нажать на кнопку загрузки фотографий для одной или нескольких фотографий соответственно.
\addimghere{upload}{1}{Загрузка фотографии}{upload}
После загрузки фотографии внизу появится ссылка для перехода на страницу фотографии (рисунок \ref{uploaded}), можно перейти на страницу с фотографией 
\addimghere{uploaded}{1}{Форма загрузки фотографий после загрузки}{uploaded}
В случае, если перейдя на страницу с фотографией не отображаются индексированные данные о фотографии, такие как теги и информация о цветах (рисунок \ref{unprocessed-photo}), необходимо обновить страницу.
Фотографии индексируются с некоторой задержкой.
\addimghere{unprocessed-photo}{0.8}{Страница с неиндексированной фотографией}{unprocessed-photo}
После индексации фотографии данные о индексации доступны на странице с фотографией (рисунок \ref{processed-photo}).
\addimghere{processed-photo}{0.8}{Страница с индексированной фотографией}{processed-photo}
Для просмотра всех фотографий пользователя, необходимо перейти на страницу просмотра фотографий пользователя.
Если фотографии находятся в общем доступе, то они будут видны всем пользователям.
\addimghere{feed}{0.8}{Лента фотографий пользователя, отображенная по времени создания фотографий}{feed}

\clearpage