\sectioncounter
\section{Обоснование выбора средств программной реализации}

В рамках работы было принято разработать продукт минимальной жизнедеятельности по подготовленному проекту. 
Учитывая возможности имеющегося оборудования и программного обеспечения, необходимо создать работоспособный прототип программного продукта, избегая таких недостатков, как высокая стоимость и длительные этапы внедрения.
Необходимо особое внимание уделить серверной части приложения, так как именно ее функционал отличает реализуемый программный продукт от аналогичной продукции, функционирующей на рынке.

\subsection{PostgreSQL}

Для разработки базы данных для продукта была выбрана свободная к распространению система управления базами данных с открытым исходным кодом - PostgreSQL \cite{postgresql-doc}.
PostgreSQL - реляционная СУБД, это значит, что данные в ней хранятся в виде логически связанных между собой таблиц, представляющих в общем случае модель предметной области.

Работать с PostgreSQL можно не только с помощью интерфейса командной строки, но и с помощью графического интерфейса инструмента, поставляемого в комплекте - pgAdmin. 
Это позволяет упростить и ускорить работу с базами данных в PostgreSQL.

PgAdmin работает через интерфейс брраузера, а значит обладает независимостью от аппаратной составляющей.
Многие из базовыых и наиболее часто используемых SQL-операций в PgAdmin сведены к интуитивно понятному кнопочному интерфейсу.

\subsection{Java}

В качестве языка программиования для программного продукта был выбран язык программирования Java \cite{java-doc}, так как он отличается отличается быстротой, высоким уровнем защиты и надежностью.
Java - сильно типизированный объектно-ориентированный язык программирования, что позволяет проще производить отладку программного кода, а также легче формализовывать объекты предметной области в виде программного кода.

Программы на Java транслируются в промежуточный между машинными командами и языком программирования код Java, который впоследствии обрабатываются и передаются на выполнение оборудованию виртуальной машиной Java.

Преимуществами выполнения кода в среде виртуальной машины является независимость от оборудования и операционной системы.
Это позволяет выполняться Java коду на любом оборудовании, для которого существует реализация виртуальной машины.
Также особенностью данного вида выполнения кода является повышенный уровень безопасности. Во время выполнения программы, она контролируется виртуальной машиной.
В случае, если запрашиваемая программным приложением операция требует уровень прав доступа выше установленных, программа немедленно завершается.

\subsection{Spring Framework}

В качестве основного фреймворка для разработки был выбран Spring Framework \cite{spring-book}, так как он универсален и предоставляет широкий функционал для разработки как масштабных корпоративных, так и небольших десктопных приложений.
Spring Framework - широкомасштабный фреймворк с открытым исходным кодом для Java.
Spring достаточно массово распространен в сообществе Java-программистов и является главным образом альтернативой модели построения корпоративных приложений Java, определенных её стандартами.
Spring предоставляет огромную свободу разработчикам при проектировании систем различного назначения. Он предлагает документацию, описывающую большую часть способов решения возникающих при проектировании приложений любого масштаба проблем.

Ядро фреймворка устроено таким образом, что может быть применено в любом приложении, написанным с использованием Java. 
Также существует большое количество платных и бесплатных усовершенствований и расширений от сообщества для построения веб-приложений на Java в корпоративном сегменте.

\subsection{Deeplearning4j}

В качестве вспомогательной библиотеки для интеграции нейронной сети в программное средство была выбрана библиотека Deeplearning4j \cite{deeplearning4j-book}.
Deeplearning4j - библиотека с открытым исходным кодом, написанная для языков Java и Scala, предоставляющая широкий функционал для обучения нейронных сетей и их последующей интеграции в Java приложение.
Библиотека имеет полную документацию, множество примеров и обширное сообщество, включает реализацию ограниченной машины Больцмана, глубокой сети доверия, глубокого автокодировщика, стекового автокодировщика с фильтрацией шума, рекурсивной тензорной нейронной сети.

\clearpage