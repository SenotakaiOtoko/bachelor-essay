\section{Обоснование выбора средств программной реализации}

В рамках работы было принято разработать продукт минимальной жизнедеятельности по подготовленному проекту. 
Учитывая возможности имеющегося оборудования и программного обеспечения, необходимо создать работоспособный прототип программного продукта, избегая таких недостатков, как высокая стоимость и длительные этапы внедрения.
Необходимо особое внимание уделить серверной части приложения, так как именно ее функционал отличает реализуемый программный продукт от аналогичной продукции, функционирующей на рынке.

\subsection{PostgreSQL}

Для разработки базы данных для продукта была выбрана свободная к распространению система управления базами данных с открытым исходным кодом - PostgreSQL.
PostgreSQL - реляционная СУБД, это значит, что данные в ней хранятся в виде логически связанных между собой таблиц, представляющих в общем случае модель предметной области.

Работать с PostgreSQL можно не только с помощью интерфейса командной строки, но и с помощью графического интерфейса инструмента, поставляемого в комплекте - pgAdmin. 
Это позволяет упростить и ускорить работу с базами данных в PostgreSQL.

PgAdmin работает через интерфейс брраузера, а значит обладает независимостью от аппаратной составляющей.
Многие из базовыых и наиболее часто используемых SQL-операций в PgAdmin сведены к интуитивно понятному кнопочному интерфейсу.

\subsection{Java}

В качестве языка программиования для программного продукта был выбран язык программирования Java, так как он отличается отличается быстротой, высоким уровнем защиты и надежностью.
Java - сильно типизированный объектно-ориентированный язык программирования, что позволяет проще производить отладку программного кода, а также легче формализовывать объекты предметной области в виде программного кода.

Программы на Java транслируются в байт-код Java, выполняемый виртуальной машиной Java (JVM) — программой, обрабатывающей байтовый код и передающей инструкции оборудованию как интерпретатор.

Достоинством подобного способа выполнения программ является полная независимость байт-кода от операционной системы и оборудования, что позволяет выполнять Java-приложения на любом устройстве, для которого существует соответствующая виртуальная машина.
Другой важной особенностью технологии Java является гибкая система безопасности, в рамках которой исполнение программы полностью контролируется виртуальной машиной.
Любые операции, которые превышают установленные полномочия программы (например, попытка несанкционированного доступа к данным или соединения с другим компьютером), вызывают немедленное прерывание.

\subsection{Spring Framework}

В качестве основного фреймворка для разработки был выбран Spring Framework, так как он универсален и предоставляет широкий функционал для разработки как масштабных корпоративных, так и небольших десктопных приложений.
Spring Framework - универсальный фреймворк с открытым исходным кодом для Java-платформы.
Несмотря на то, что Spring не обеспечивает какую-либо конкретную модель программирования, он широко распространён в Java-сообществе главным образом как альтернатива и замена стандартной модели построения корпоративных приложений, являющейся частью Java.
Spring предоставляет боольшую свободу Java-разработчикам в проектировании; кроме того, он предоставляет хорошо документированные и лёгкие в использовании средства решения проблем, возникающих при создании приложений корпоративного масштаба.

Особенности ядра Spring применимы в любом Java-приложении, и существует множество расширений и усовершенствований для построения веб-приложений на Java Enterprise платформе.

\subsection{Deeplearning4j}

В качестве вспомогательной библиотеки для интеграции нейронной сети в программное средство была выбрана библиотека Deeplearning4j.
Deeplearning4j - библиотека с открытым исходным кодом, написанная для языков Java и Scala, предоставляющая широкий функционал для обучения нейронных сетей и их последующей интеграции в Java приложение.
Библиотека имеет полную документацию, множество примеров и обширное сообщество, включает реализацию ограниченной машины Больцмана, глубокой сети доверия, глубокого автокодировщика, стекового автокодировщика с фильтрацией шума, рекурсивной тензорной нейронной сети.

\clearpage