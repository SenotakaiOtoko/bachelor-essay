\anonsection{Введение}

Фотография - неотьемлемая составляющая процессов производства и потребления медиаконтента.
Данный формат контента является доминирующим в информационном пространстве. Он обладает рядом преймуществ перед другими видами представления информации: Считывается раньше текста, может быть как самостоятельным, так и дополняющим другие виды представления информации, обладает более низким порогом вхождения для производства контента конкурентного качества и более широким распространением.
В данной сфере возрастает необходимость в качественных инструментах, позволяющих производителям контента сосредоточиться непосредственно на этапе производства, а не на организации производимого контента в хранилищах и на площадках, предназначенных для потребления контента.

Для эффективной работы 

Фотохостинг --- это интернет сервис, спроектированный и работающий по методике web 2.0, позволяющий публиковать изображения в интернете с целью хранения и/или показа фотографий другим пользователям сети интернет.

Для эффективной работы облачной инфраструктуры требуется эффективная структура и организация.
Небольшая команда из специалистов и бизнес-пользователей может создать обоснованный план и организовать свою работу в инфраструктуре.
Данная выделенная группа может намного эффективнее построить и управлять нестандартной облачной инфраструктурой, чем если компании будут просто продолжать добавлять дополнительные сервера и сервисы для поддержки центра обработки данных (ЦОД).

IaaS (Infrastructure as a Service) --- это предоставление пользователю компьютерной и сетевой инфраструктуры, их обслуживание как услуги в форме виртуализации, то есть виртуальной инфраструктуры.
Другими словами, на базе физической инфраструктуры дата-центров (ДЦ) провайдер создает виртуальную инфраструктуру, которую предоставляет пользователям как сервис.
Стоит отметить, что IaaS не предполагает передачи в аренду программного обеспечения, а всего лишь предоставляет доступ к вычислительным мощностям.

Технология виртуализации ресурсов позволяет физическое оборудование (сервера, хранилища данных, сети передачи данных) разделить между пользователями на несколько частей, которые используются ими для выполнения текущих задач.
К примеру, на одном физическом сервере можно запустить сотни виртуальных серверов, а пользователю для решения задач выделить время доступа к ним.

\clearpage