\anonsection{Введение}

Фотография - неотъемлемая составляющая процессов производства и потребления медиа контента.
Данный формат контента является доминирующим в информационном пространстве. Он обладает рядом преимуществ перед другими видами представления информации: считывается раньше текста, может в качестве самостоятельного источника информации, а также дополнять другие виды представления информации, обладает более низким порогом вхождения для производства контента конкурентного качества и более широким распространением.

В данной сфере возрастает необходимость в качественных инструментах, позволяющих производителям контента сосредоточиться непосредственно на этапе производства, а не на организации производимого контента в хранилищах и на площадках, предназначенных для потребления контента.

Фотохостинг — это интернет сервис, спроектированный и работающий по методике web 2.0 [1], позволяющий публиковать изображения в интернете с целью хранения и/или показа фотографий другим пользователям сети интернет.

Принципы работы фотохостинга:
\begin{enumerate}
    \item пользователь загружает готовую, или почти готовую фотографию в интернет-сервис;
    \item пользователь производит над фотографией операции по обработке, связанные с пред публикационным состоянием фотографии (такие, как обрезка, наложение фильтров, регулировка контраста);
    \item пользователь добавляет описание к фотографии при необходимости;
    \item пользователь вручную добавляет теги к фотографии, соответствующие классовой принадлежности фотографии;
    \item пользователь публикует фотографию в рамках фотохостинга, доступную к просмотру для всех пользователей сети интернет, а также к оцениванию и комментированию со стороны зарегистрированных и авторизованных пользователей фотохостинга;
    \item пользователь оценивает и комментирует фотографии других пользователей, загруженные на фотохостинг.
\end{enumerate}
    
Главная задача программного обеспечения, так или иначе связанного с творчеством — уменьшить затраты пользователя на рутинные технические вещи, автоматизировать их, оставить только по-настоящему творческие задачи человеку.
Ещё одно применение изощренных технологий машинного обучения, которые ненавязчиво дают софту конкурентные преимущества, а пользователи даже не задумываются о спрятанном искусственном интеллекте.
Поставлена задача разработать приложение, которое станет удобным инструментом для людей, связанных с производством фотоконтента в сети интернет. Приложение поможет пользователю организовать удобное хранение фотографий в автоматическом режиме, помечая фотографии тегами, исходя из содержимого фотографий, позволяя в дальнейшем осуществлять поиск фотографий по необходимым критериям, настраивать их приватность и публиковать в сети интернет при необходимости. Для выполнения задачи классификации фотографий принято решение об использовании нейронных сетей.

\clearpage