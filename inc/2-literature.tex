\section{Проектирование}
\subsection{Требования к функциональным характеристикам}
Программа должна обеспечивать возможность выполнения перечисленных ниже функций:
\begin{itemize}
  \item Загрузка и хранение фотографий с возможностью просмотра и дальнейшего скачивания;
  \item Разграничение прав доступа, регистрация пользователей с подтверждением прохождения регистрации посредством email или смс, авторизация пользователей;
  \item Организация хранилища фотографий пользователей в автоматическом или полуавтоматическом режиме с возможностью осуществления дальнейшего поиска по необходимым критериям;
  \item Возможность публикации фотографий в сети интернет;
  \item Возможность комментирования и оценивания фотографий, загруженных в сервис;
  \item Индексирование загружаемых фотографий по цветам для осуществления возможности дальнейшего поиска фотографий по цветам;
  \item Классификация фотографий с использованием нейронных сетей и индексирование результатов с целью возможности дальнейшего поиска фотографий по данному критерию;
\end{itemize}
\subsection{Общее описание}
\subsection{Варианты использования}
Перечень вариантов использования системы приведена в таблице \ref{use-case-table}.
Диаграмма вариантов использования на рисунке \ref{use-case} показывает варианты использования системы и связанные с ними действующие лица.

\begin{table}[H]
  \caption{Варианты использования системы хранения фотографий}\label{use-case-table}
  \begin{tabular}{|c|c|}
  \hline Основное действующее лицо & Вариант использования \\
  \hline  Посетитель & 1. Пройти регистрацию \\
  \hline   & 2. Просмотреть ленту популярных фотографий \\
  \hline   & 3. Просмотреть фотографии пользователя \\
  \hline   & 4. Просмотреть информацию о фотографии \\
  \hline   & 5. Осуществить поиск фотографии по необходимым критериям \\
  \hline   & 6. Авторизоваться \\
  \hline  Пользователь & 1. Привязать популярные соц сети \\
  \hline   & 2. Создать пост в соц сетях \\
  \hline   & 3. Настроить приватность фотографии \\
  \hline   & 4. Настроить защиту от копирования \\
  \hline   & 5. Оценить фотографию \\
  \hline   & 6. Прокомментировать фотографию \\
  \hline   & 7. Просмотреть статистику \\
  \hline   & 8. Сохранить в избранное \\
  \hline   & 9. Подписаться на публикации других пользователей \\
  \hline   & 10. Опубликовать фотографию \\
  \hline
  \end{tabular}
\end{table}

\addimg{use-case}{0.35}{Диаграмма вариантов использования программного решения для хранения фотографий}{use-case}
\subsection{Структура базы данных}
\subsection{Диаграмма классов}
\clearpage