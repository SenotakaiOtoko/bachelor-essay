\section{Проектирование}

\subsection{Функциональные требования}

Программа должна обеспечивать возможность выполнения перечисленных ниже функций:
\begin{itemize}
  \item загрузка и хранение фотографий с возможностью просмотра и дальнейшего скачивания;
  \item разграничение прав доступа, регистрация пользователей с подтверждением прохождения регистрации посредством email или смс, авторизация пользователей;
  \item организация хранилища фотографий пользователей в автоматическом или полуавтоматическом режиме с возможностью осуществления дальнейшего поиска по необходимым критериям;
  \item возможность публикации фотографий в сети интернет;
  \item возможность комментирования и оценивания фотографий, загруженных в сервис;
  \item индексирование загружаемых фотографий по цветам для осуществления возможности дальнейшего поиска фотографий по цветам;
  \item классификация фотографий с использованием нейронных сетей и индексирование результатов с целью возможности дальнейшего поиска фотографий по данному критерию.
\end{itemize}

\subsection{Классы и характеристики пользователей}

Перечень классов пользователей системы и их краткая характеристика представлены в таблице \ref{user-classes-table}.
\begin{table}[H]
  \caption{Классы и характеристики пользователей системы хранения фотографий}\label{user-classes-table}
  \begin{tabular}{|p{4cm}|p{12cm}|}
  \hline Класс пользователей & Описание \\ 
  \hline Посетитель & Посетитель – человек, просматривающий фотографии, загруженные в сервис и доступные для всех, но не являющийся зарегистрированным пользователем \\ 
  \hline Пользователь & Человек, прошедший регистрацию и подтвердивший себя посредством СМС или email. Отличается от посетителя большим набором возможностей, таким как публикация фотографий \\ 
  \hline
  \end{tabular}
\end{table}

\subsection{Варианты использования}
Перечень вариантов использования системы приведена в таблице \ref{use-case-table}.
Диаграмма вариантов использования на рисунке \ref{use-case} показывает варианты использования системы и связанные с ними действующие лица.

\begin{table}[H]
  \caption{Варианты использования системы хранения фотографий}\label{use-case-table}
  \begin{tabular}{|p{6cm}|p{10cm}|}
  \hline Основное действующее лицо & Вариант использования \\
  \hline \multirow{6}{*}{Посетитель} & 1. Пройти регистрацию \\
  \cline{2-2} & 2. Просмотреть ленту популярных фотографий \\
  \cline{2-2} & 3. Просмотреть фотографии пользователя \\
  \cline{2-2} & 4. Просмотреть информацию о фотографии \\
  \cline{2-2} & 5. Осуществить поиск фотографии по необходимым критериям \\
  \cline{2-2} & 6. Авторизоваться \\
  \hline \multirow{10}{*}{Пользователь} & 1. Привязать популярные соц сети \\
  \cline{2-2} & 2. Создать пост в соц сетях \\
  \cline{2-2} & 3. Настроить приватность фотографии \\
  \cline{2-2} & 4. Настроить защиту от копирования \\
  \cline{2-2} & 5. Оценить фотографию \\
  \cline{2-2} & 6. Прокомментировать фотографию \\
  \cline{2-2} & 7. Просмотреть статистику \\
  \cline{2-2} & 8. Сохранить в избранное \\
  \cline{2-2} & 9. Подписаться на публикации других пользователей \\
  \cline{2-2} & 10. Опубликовать фотографию \\
  \hline
  \end{tabular}
\end{table}

\begin{table}[H]
  \caption{Вариант использования - 1 – Пройти регистрацию}\label{use-case-1-table}
  \begin{tabular}{|p{6cm}|p{10cm}|}
  \hline № варианта использования: & Вариант использования - 1 \\
  \hline Название варианта использования: & Пройти регистрацию \\
  \hline Действующие лица: & Посетитель \\
  \hline Описание: & Посетитель заполняет форму со своими данными для последующей авторизации и подтверждает регистрацию \\
  \hline Предварительные условия: & Нет \\
  \hline Выходные условия: & Нет \\
  \hline \multirow{7}{*}{Нормальное направление:} & 1.0  Пройти регистрацию \\
  \cline{2-2} & 1. Пользователь заполняет форму с данными, основными являются логин, пароль и email или номер телефона \\
  \cline{2-2} & 2. Нажимает кнопку регистрации на форме \\
  \cline{2-2} & 3. Система создает в БД запись о пользователе и генерирует токен для подтверждения учетной записи. Токен отправляется посредством email или смс \\
  \cline{2-2} & 4. Пользователь получает токен для подтверждения прохождения регистрации \\
  \cline{2-2} & 5. Подтверждает регистрацию на сайте \\
  \cline{2-2} & 6. Система отмечает, что пользователь успешно зарегистрирован и авторизовывает его. Посетитель становится пользователем. \\
  \hline Альтернативные направления: &  \\
  %\hline \multirow{4}{*}{Исключения:} & 1.0.И.1 Пользователь ввел пароль, не удовлетворяющий условиям безопасности \\
  %\cline{2-2} & 1. Сообщение об ошибке, возврат к пункту 1. \\
  %\cline{2-2} & 1.2.И.2 Пользователь ввел неверный токен для подтверждения регистрации \\
  %\cline{2-2} & 1. Сообщение об ошибке, генерация нового токена и повторная отправка \\
  %\hline Включает: &  \\
  %\hline Приоритет: & Высокий \\
  %\hline Особые требования: &  \\
  \hline
  \end{tabular}
\end{table}

\begin{table}[H]
  \caption{Вариант использования - 2 – Просмотреть ленту популярных фотографий}\label{use-case-2-table}
  \begin{tabular}{|p{6cm}|p{10cm}|}
  \hline № варианта использования: & Вариант использования - 2 \\
  \hline Название варианта использования: & Просмотреть ленту популярных фотографий \\
  \hline Действующие лица: & Посетитель \\
  \hline Описание: & Посетитель открывает веб страницу, на которой содержатся все популярные фотографии за определенный промежуток времени \\
  \hline Предварительные условия: & В сервисе загружены фотографии, доступ к которым разрешен всем \\
  \hline Выходные условия: & Нет \\
  \hline \multirow{4}{*}{Нормальное направление:} & 2.0 Просмотреть ленту популярных фотографий \\
  \cline{2-2} & 1. Посетитель открывает страницу с популярными фотографиями \\
  \cline{2-2} & 2. Сервис формирует список популярных фотографий, доступных всем, за определенный промежуток времени и отправляет посетителю \\
  \cline{2-2} & 3. Посетитель путем скроллинга веб страницы осуществляет просмотр популярных фотографий \\
  \hline Альтернативные направления: &  \\
  \hline \multirow{2}{*}{Исключения:} & 2.0.И.1 Отсутствуют фотографии, доступные всем \\
  \cline{2-2} & 1. Сообщение об ошибке на странице просмотра фотографий \\
  \hline Включает: &  \\
  \hline Приоритет: & Низкий \\
  \hline Особые требования: & \\
  \hline
  \end{tabular}
\end{table}

\begin{table}[H]
  \caption{Вариант использования - 3 – Просмотреть ленту популярных фотографий}\label{use-case-3-table}
  \begin{tabular}{|p{6cm}|p{10cm}|}
  \hline № варианта использования: & Вариант использования - 3 \\
  \hline Название варианта использования: & Просмотреть фотографии пользователя \\
  \hline Действующие лица: & Посетитель \\
  \hline Описание: & Посетитель просматривает фотографии пользователя, доступные всем \\
  \hline Предварительные условия: & Пользователь, фотографии которого пытается просмотреть посетитель, существует и загрузил хотя бы одну фотографию, доступную для просмотра всем \\
  \hline Выходные условия: & Нет \\
  \hline \multirow{4}{*}{Нормальное направление:} & 3.0 Просмотреть фотографии пользователя \\
  \cline{2-2} & 1. Посетитель открывает страницу в профиле пользователя с фотографиями \\
  \cline{2-2} & 2. Сервис формирует список фотографий пользователя, доступный всем \\
  \cline{2-2} & 3. Посетитель путем скроллинга веб страницы осуществляет просмотр фотографий \\
  \hline Альтернативные направления: &  \\
  %\hline \multirow{4}{*}{Исключения:} & 3.0.И.1 Пользователь, фотографии которого пытается просмотреть посетитель, не существует \\
  %\cline{2-2} & 1. Сообщение об ошибке, система перенаправляет пользователя к главному окну \\
  %\cline{2-2} & 3.0.И.2 Пользователь, фотографии которого пытается просмотреть посетитель, не сделал ни одной фотографии доступной всем \\
  %\cline{2-2} & 1. Сообщение об ошибке на странице просмотра фотографий \\
  %\hline Включает: &  \\
  %\hline Приоритет: & Низкий \\
  %\hline Особые требования: &  \\
  \hline 
  \end{tabular}
\end{table}

\begin{table}[H]
  \caption{Вариант использования - 4 - Просмотреть информацию о фотографии}\label{use-case-4-table}
  \begin{tabular}{|p{6cm}|p{10cm}|}
  \hline № варианта использования: & Вариант использования - 4 \\
  \hline Название варианта использования: & Просмотреть информацию о фотографии \\
  \hline Действующие лица: & Посетитель \\
  \hline Описание: & Посетитель просматривает информацию о фотографии, такую как метаданные фотографии, теги, количество пользователей, которым понравилась фотография и т.п.  \\
  \hline Предварительные условия: & Фотография, информацию о которой пытается просмотреть посетитель, существует и доступна для просмотра всем \\
  \hline Выходные условия: & Нет \\
  \hline \multirow{4}{*}{Нормальное направление:} & 4.0 Просмотреть информацию о фотографии \\
  \cline{2-2} & 1. Посетитель открывает страницу с фотографией \\
  \cline{2-2} & 2. Система находит фотографию и всю информацию о ней \\
  \cline{2-2} & 3. Посетитель просматривает доступную информацию о фотографии \\
  \hline Альтернативные направления: &  \\
  %\hline \multirow{4}{*}{Исключения:} & 4.0.И.1 Отсутствует фотография, информацию о которой. Пытается просмотреть посетитель, или фотография недоступна для всех \\
  %\cline{2-2} & 1. Сообщение об ошибке, система перенаправляет пользователя к главному окну \\
  %\cline{2-2} & 4.0.И.2 Не найдена какая-то информация о фотографии \\
  %\cline{2-2} & 1. Сообщение об ошибке на странице просмотра информации о фотографии \\
  %\hline Включает: &  \\
  %\hline Приоритет: & Высокий \\
  %\hline Особые требования: & \\
  \hline
  \end{tabular}
\end{table}

\begin{table}[H]
  \caption{Вариант использования - 5 – Осуществить поиск фотографии по необходимым критериям}\label{use-case-5-table}
  \begin{tabular}{|p{6cm}|p{10cm}|}
  \hline № варианта использования: & Вариант использования - 5 \\
  \hline Название варианта использования: & Осуществить поиск фотографии по необходимым критериям \\
  \hline Действующие лица: & Посетитель \\
  \hline Описание: & Посетитель осуществляет поиск фотографий с необходимыми ему критериями, такими, как цвет на фотографии, содержимое фотографии и т.д. \\
  \hline Предварительные условия: & Хотя бы одна фотография загружена в сервис и проиндексирована для поиска \\
  \hline Выходные условия: & Нет \\
  \hline \multirow{5}{*}{Нормальное направление:} & 5.0 Осуществить поиск фотографии по необходимым критериям \\
  \cline{2-2} & 1. Посетитель открывает страницу поиска фотографий \\
  \cline{2-2} & 2. Вводит необходимые ему критерии \\
  \cline{2-2} & 3. Система формирует список фотографий \\
  \cline{2-2} & 4. Посетитель путем скроллинга веб страницы просматривает список найденных фотографий \\
  \hline Альтернативные направления: &  \\
  \hline \multirow{2}{*}{Исключения:} & 5.0.И.1 Фотографии с заданными критериями поиска не найдены \\
  \cline{2-2} & 1. Сообщение об ошибке на странице поиска фотографий \\
  \hline Включает: &  \\
  \hline Приоритет: & Высокий \\
  \hline Особые требования: &  \\
  \hline 
  \end{tabular}
\end{table}

\begin{table}[H]
  \caption{Вариант использования - 6 – Авторизоваться}\label{use-case-6-table}
  \begin{tabular}{|p{6cm}|p{10cm}|}
  \hline № варианта использования: & Вариант использования - 6 \\
  \hline Название варианта использования: & Авторизоваться \\
  \hline Действующие лица: & Посетитель \\
  \hline Описание: & Посетитель вводит логин и пароль от принадлежащей ему учетной записи пользователя и авторизовывается \\
  \hline Предварительные условия: & Пользователь, данные которого вводятся, зарегистрирован в системе \\
  \hline Выходные условия: & Нет \\
  \hline \multirow{5}{*}{Нормальное направление:} & 6.0 Авторизоваться \\
  \cline{2-2} & 1. Посетитель открывает страницу авторизации \\
  \cline{2-2} & 2. Вводит логин и пароль \\
  \cline{2-2} & 3. При включенной у учетной записи пользователя двухфакторинговой авторизации вводит дополнительные данные для входа \\
  \cline{2-2} & 4. Система. Авторизует пользователя и перенаправляет на главную страницу \\
  \hline Альтернативные направления: &  \\
  \hline Исключения: &  \\
  \hline Включает: &  \\
  \hline Приоритет: & Высокий \\
  \hline Особые требования: &  \\
  \hline
  \end{tabular}
\end{table}

\begin{table}[H]
  \caption{Вариант использования - 7 - Привязать популярные соц сети}\label{use-case-7-table}
  \begin{tabular}{|p{6cm}|p{10cm}|}
  \hline № варианта использования: & Вариант использования - 7 \\
  \hline Название варианта использования: & Привязать популярные соц сети \\
  \hline Действующие лица: & Пользователь \\
  \hline Описание: & Пользователь привязывает аккаунт соц сети с целью дальнейшего создания постов и публикации фотографий в соц сети \\
  \hline Предварительные условия: & Пользователь авторизован \\
  \hline Выходные условия: & Нет \\
  \hline \multirow{7}{*}{Нормальное направление:} & 7.0 Привязать популярные соц сети \\
  \cline{2-2} & 1. Пользователь открывает страницу своего профиля \\
  \cline{2-2} & 2. Выбирает необходимую ему социальную сеть из списка предложенных \\
  \cline{2-2} & 3. Система перенаправляет пользователя на страницу социальной сети для авторизации  \\
  \cline{2-2} & 4. Пользователь авторизуется в соц сети \\
  \cline{2-2} & 5. Соц сесть перенаправляет пользователя обратно в систему \\
  \cline{2-2} & 6. Система привязывает переданный соц сетью токен к пользователю \\
  \hline Альтернативные направления: &  \\
  \hline Исключения: &  \\
  \hline Включает: &  \\
  \hline Приоритет: & Низкий \\
  \hline Особые требования: &  \\
  \hline 
  \end{tabular}
\end{table}

\begin{table}[H]
  \caption{Вариант использования - 8 – Создать пост в соц сетях}\label{use-case-8-table}
  \begin{tabular}{|p{6cm}|p{10cm}|}
  \hline № варианта использования: & Вариант использования - 8 \\
  \hline Название варианта использования: & Создать пост в соц сетях \\
  \hline Действующие лица: & Пользователь \\
  \hline Описание: & Пользователь при создани поста отмечает о необходимости его публикации в социальных сетях. Пост автоматически публикуется во всех отмеченных соц сетях \\
  \hline Предварительные условия: & Пользователь авторизован и хотя бы один аккаунт социальной сети привязан к аккаунту пользователя \\
  \hline Выходные условия: & Нет \\
  \hline \multirow{4}{*}{Нормальное направление:} & 8.0 Создать пост в соц сетях \\
  \cline{2-2} & 1. Пользователь переходит на страницу создания поста \\
  \cline{2-2} & 2. Формирует пост для дальнейшей публикации \\
  \cline{2-2} & 3. Публикует внутри сервиса и при публикации отмечает о необходимости публикации в аккаунте социальной сети \\
  \hline Альтернативные направления: &  \\
  \hline Исключения: &  \\
  \hline Включает: &  \\
  \hline Приоритет: & Низкий \\
  \hline Особые требования: &  \\
  \hline 
  \end{tabular}
\end{table}

\begin{table}[H]
  \caption{Вариант использования - 9 – Настроить приватность фотографии}\label{use-case-9-table}
  \begin{tabular}{|p{6cm}|p{10cm}|}
  \hline № варианта использования: & Вариант использования - 9 \\
  \hline Название варианта использования: & Настроить приватность фотографии \\
  \hline Действующие лица: & Пользователь \\
  \hline Описание: & Пользователь настраивает доступ к фотографии определенному кругу лиц \\
  \hline Предварительные условия: & Пользователь авторизован и загрузил фотографию, доступ к которой хочет настроить \\
  \hline Выходные условия: & Нет \\
  \hline \multirow{4}{*}{Нормальное направление:} & 9.0 Настроить приватность фотографии \\
  \cline{2-2} & 1. Пользователь открывает страницу настроек фотографии \\
  \cline{2-2} & 2. Переходит к форме настройки листов доступа \\
  \cline{2-2} & 3. Выбирает тип доступа и настраивает листы доступа \\
  \hline Альтернативные направления: &  \\
  \hline Исключения: &  \\
  \hline Включает: &  \\
  \hline Приоритет: & Низкий \\
  \hline Особые требования: &  \\
  \hline 
  \end{tabular}
\end{table}

\begin{table}[H]
  \caption{Вариант использования - 10 – Оценить фотографию}\label{use-case-10-table}
  \begin{tabular}{|p{6cm}|p{10cm}|}
  \hline № варианта использования: & Вариант использования - 10 \\
  \hline Название варианта использования: & Оценить фотографию \\
  \hline Действующие лица: & Пользователь \\
  \hline Описание: & Пользователь оценивает понравившуюся ему фотографию \\
  \hline Предварительные условия: & Пользователь авторизован и имеет доступ к фотографии, которую хочет оценить \\
  \hline Выходные условия: & Нет \\
  \hline \multirow{3}{*}{Нормальное направление:} & 10.0 Оценить фотографию \\
  \cline{2-2} & 1. Пользователь открывает страницу фотографии \\
  \cline{2-2} & 2. Нажимает кнопку оценки фотографии \\
  \hline Альтернативные направления: &  \\
  \hline Исключения: &  \\
  \hline Включает: &  \\
  \hline Приоритет: & Низкий \\
  \hline Особые требования: &  \\
  \hline 
  \end{tabular}
\end{table}

\begin{table}[H]
  \caption{Вариант использования - 11 – Прокомментировать фотографию}\label{use-case-11-table}
  \begin{tabular}{|p{6cm}|p{10cm}|}
  \hline № варианта использования: & Вариант использования - 11 \\
  \hline Название варианта использования: & Прокомментировать фотографию \\
  \hline Действующие лица: & Пользователь \\
  \hline Описание: & Пользователь оставляет комментарий к фотографии или в ответ на другой комментарий \\
  \hline Предварительные условия: & Пользователь авторизован и имеет доступ к фотографии, к которой хочет оставить комментарий \\
  \hline Выходные условия: & Нет \\
  \hline \multirow{3}{*}{Нормальное направление:} & 11.0 Прокомментировать фотографию \\
  \cline{2-2} & 1. Пользователь открывает страницу фотографии \\
  \cline{2-2} & 2. Вводит комментарий в форму ввода под фотографией или необходимым комментарием и нажимает кнопку «отправить» \\
  \hline Альтернативные направления: &  \\
  \hline Исключения: &  \\
  \hline Включает: &  \\
  \hline Приоритет: & Низкий \\
  \hline Особые требования: &  \\
  \hline 
  \end{tabular}
\end{table}

\begin{table}[H]
  \caption{Вариант использования - 12 – Просмотреть статистику}\label{use-case-12-table}
  \begin{tabular}{|p{6cm}|p{10cm}|}
  \hline № варианта использования: & Вариант использования - 12 \\
  \hline Название варианта использования: & Просмотреть статистику \\
  \hline Действующие лица: & Пользователь \\
  \hline Описание: & Пользователь просматривает глобальную статистику фотографий по сервису или статистику по одной из своих фотографий \\
  \hline Предварительные условия: & Пользователь авторизован \\
  \hline Выходные условия: & Нет \\
  \hline \multirow{3}{*}{Нормальное направление:} & 12.0 Просмотреть статистику \\
  \cline{2-2} & 1. Пользователь открывает страницу с глобальной статистикой \\
  \cline{2-2} & 2. Просматривает глобальную статистику фотографий по сервиису \\
  \hline Альтернативные направления: &  \\
  \hline Исключения: &  \\
  \hline Включает: &  \\
  \hline Приоритет: & Низкий \\
  \hline Особые требования: &  \\
  \hline 
  \end{tabular}
\end{table}

\begin{table}[H]
  \caption{Вариант использования - 13 – Сохранить в избранное}\label{use-case-13-table}
  \begin{tabular}{|p{6cm}|p{10cm}|}
  \hline № варианта использования: & Вариант использования - 13 \\
  \hline Название варианта использования: & Сохранить в избранное \\
  \hline Действующие лица: & Пользователь \\
  \hline Описание: & Пользователь сохраняет понравившуюся фотографию в избранное внутри сервиса \\
  \hline Предварительные условия: & Пользователь авторизован и имеет доступ к фотографии, которую хочет сохранить в избранное \\
  \hline Выходные условия: & Нет \\
  \hline \multirow{2}{*}{Нормальное направление:} & 13.0 Сохранить в избранное \\
  \cline{2-2} & 1. Пользователь нажимает на кнопку «сохранить в избранное» рядом с понравившейся фотографией \\
  \hline Альтернативные направления: &  \\
  \hline Исключения: &  \\
  \hline Включает: &  \\
  \hline Приоритет: & Низкий \\
  \hline Особые требования: &  \\
  \hline 
  \end{tabular}
\end{table}

\begin{table}[H]
  \caption{Вариант использования - 14 – Подписаться на публикации других пользователей}\label{use-case-14-table}
  \begin{tabular}{|p{6cm}|p{10cm}|}
  \hline № варианта использования: & Вариант использования - 14 \\
  \hline Название варианта использования: & Подписаться на публикации других пользователей \\
  \hline Действующие лица: & Пользователь \\
  \hline Описание: & Пользователь добавляет в свою персональную ленту интересных публикаций и фотографий все фотографии и посты другого пользователя \\
  \hline Предварительные условия: & Пользователь авторизован \\
  \hline Выходные условия: & Нет \\
  \hline \multirow{3}{*}{Нормальное направление:} & 14.0 Подписаться на публикации других пользователей \\
  \cline{2-2} & 1. Пользователь переходит на страницу пользователя, на публикации которого он хочет подписаться \\
  \cline{2-2} & 2. Нажимает кнопку «подписаться» \\
  \hline Альтернативные направления: &  \\
  \hline Исключения: &  \\
  \hline Включает: &  \\
  \hline Приоритет: & Низкий \\
  \hline Особые требования: &  \\
  \hline 
  \end{tabular}
\end{table}

\begin{table}[H]
  \caption{Вариант использования - 15 – Опубликовать фотографию}\label{use-case-15-table}
  \begin{tabular}{|p{6cm}|p{10cm}|}
  \hline № варианта использования: & Вариант использования - 15 \\
  \hline Название варианта использования: & Опубликовать фотографию \\
  \hline Действующие лица: & Пользователь \\
  \hline Описание: & Пользователь загружает фотографию в сервис \\
  \hline Предварительные условия: & Пользователь авторизован \\
  \hline Выходные условия: & Нет \\
  \hline \multirow{4}{*}{Нормальное направление:} & 15.0 Опубликовать фотографию \\
  \cline{2-2} & 1. Пользователь переходит на страницу загрузки фотографии \\
  \cline{2-2} & 2. Добавляет новую фотографию \\
  \cline{2-2} & 3. Настраивает доступ к фотографии и выставляет теги к фотографии из списка предложенных и/или самостоятельно \\
  \hline Альтернативные направления: &  \\
  \hline Исключения: &  \\
  \hline Включает: &  \\
  \hline Приоритет: & Низкий \\
  \hline Особые требования: & \\
  \hline
  \end{tabular}
\end{table}

\begin{landscape}
    \addimg{use-case}{1.0}{Диаграмма вариантов использования программного решения для хранения фотографий}{use-case}
\end{landscape}

\subsection{Структура базы данных}
На рисунке \ref{database-structure-1} представлена модель базы данных программного решения.
\addimghere{database-structure-1}{0.8}{Модель базы данных программного решения для хранения фотографий}{database-structure-1}

\subsection{Диаграмма классов}
На рисунках \ref{class-diagram-1} и \ref{class-diagram-2} представлены диаграммы классов программного решения.
\addimghere{class-diagram-1}{1}{Диаграмма классов сущностей и репозиториев программного решения для хранения фотографий}{class-diagram-1}
\addimg{class-diagram-2}{1}{Диаграмма классов контроллеров программного решения для хранения фотографий}{class-diagram-2}
\begin{landscape}
    \subsection{Диаграмма активностей}
    На рисунке \ref{activity-registration} представлена диаграмма активностей процесса регистрации пользователя.
    \addimghere{activity-registration}{0.7}{Диаграмма активностей процесса регистрации пользователя}{activity-registration}
\end{landscape}

\subsection{Описание классов}
На рисунках \ref{crc-table-1}-\ref{crc-table-last} представлены CRC карточки сервиса с элементами социальной сети для публикации фотографий

\newcommand{\bdot}{\(\bullet\hspace{0.5em}\)}

\begin{table}[H]
\caption{CRC ����窠 ����� ImageProcessServiceApplication}\label{crc-table-1}
\begin{tabular}{|p{8cm} p{8cm}|} 
\hline class &  \\
\multicolumn{2}{|c|}{ImageProcessServiceApplication} \\ \hline
\end{tabular}
\begin{tabular}{|p{8cm}|p{8cm}|} 
\multirow{2}{=}{ main ����� } 
& \bdot ColorProperties \\
& \bdot FileStorageProperties \\
\hline 
\end{tabular}
\end{table}

\begin{table}[H]
\caption{CRC ����窠 ����� OnRegistrationCompleteEvent}\label{crc-table-2}
\begin{tabular}{|p{8cm} p{8cm}|} 
\hline class & ApplicationEvent \\
\multicolumn{2}{|c|}{OnRegistrationCompleteEvent} \\ \hline
\end{tabular}
\begin{tabular}{|p{8cm}|p{8cm}|} 
  ����⨥ - �����襭�� ॣ����樨  & \bdot FileStorageProperties \\
\hline 
\end{tabular}
\end{table}

\begin{table}[H]
\caption{CRC ����窠 ����� RegistrationListener}\label{crc-table-3}
\begin{tabular}{|p{8cm} p{8cm}|} 
\hline class & ApplicationListener \\
\multicolumn{2}{|c|}{RegistrationListener} \\ \hline
\end{tabular}
\begin{tabular}{|p{8cm}|p{8cm}|} 
\multirow{3}{=}{ ����⥫� ᮡ�⨩ ॣ����樨 } 
& \bdot UserAccount \\
& \bdot OnRegistrationCompleteEvent \\
& \bdot IUserService \\
\hline 
\end{tabular}
\end{table}

\begin{table}[H]
\caption{CRC ����窠 ����� ColorUtils}\label{crc-table-4}
\begin{tabular}{|p{8cm} p{8cm}|} 
\hline class &  \\
\multicolumn{2}{|c|}{ColorUtils} \\ \hline
\end{tabular}
\begin{tabular}{|p{8cm}|p{8cm}|} 
  �⨫��� �� ࠡ�� � 梥⠬�  & \\
\hline 
\end{tabular}
\end{table}

\begin{table}[H]
\caption{CRC ����窠 ����� }\label{crc-table-5}
\begin{tabular}{|p{8cm} p{8cm}|} 
\hline enum &  \\
\multicolumn{2}{|c|}{} \\ \hline
\end{tabular}
\begin{tabular}{|p{8cm}|p{8cm}|} 
  �஢����� ��⥭�䨪�樨  & \\
\hline 
\end{tabular}
\end{table}

\begin{table}[H]
\caption{CRC ����窠 ����� RandomString}\label{crc-table-6}
\begin{tabular}{|p{8cm} p{8cm}|} 
\hline class &  \\
\multicolumn{2}{|c|}{RandomString} \\ \hline
\end{tabular}
\begin{tabular}{|p{8cm}|p{8cm}|} 
  �⨫�� ��� �����樨 ��砩��� ��ப �������� �����  & \\
\hline 
\end{tabular}
\end{table}

\begin{table}[H]
\caption{CRC ����窠 ����� FileStorageProperties}\label{crc-table-7}
\begin{tabular}{|p{8cm} p{8cm}|} 
\hline class &  \\
\multicolumn{2}{|c|}{FileStorageProperties} \\ \hline
\end{tabular}
\begin{tabular}{|p{8cm}|p{8cm}|} 
  ����ன�� 䠩������ �࠭���� �ࢨ�  & \\
\hline 
\end{tabular}
\end{table}

\begin{table}[H]
\caption{CRC ����窠 ����� ColorProperties}\label{crc-table-8}
\begin{tabular}{|p{8cm} p{8cm}|} 
\hline class &  \\
\multicolumn{2}{|c|}{ColorProperties} \\ \hline
\end{tabular}
\begin{tabular}{|p{8cm}|p{8cm}|} 
  ����ன�� �ࢨ� �� 梥⮢�� �����ਧ�樨  & \\
\hline 
\end{tabular}
\end{table}

\begin{table}[H]
\caption{CRC ����窠 ����� SessionListenerWithMetrics}\label{crc-table-9}
\begin{tabular}{|p{8cm} p{8cm}|} 
\hline class & HttpSessionListener \\
\multicolumn{2}{|c|}{SessionListenerWithMetrics} \\ \hline
\end{tabular}
\begin{tabular}{|p{8cm}|p{8cm}|} 
  ����⥫� ��ᨩ, �⢥��騩 �� ���ਪ�  & \bdot IUserService \\
\hline 
\end{tabular}
\end{table}

\begin{table}[H]
\caption{CRC ����窠 ����� PasswordDto}\label{crc-table-10}
\begin{tabular}{|p{8cm} p{8cm}|} 
\hline class &  \\
\multicolumn{2}{|c|}{PasswordDto} \\ \hline
\end{tabular}
\begin{tabular}{|p{8cm}|p{8cm}|} 
  DTO - ���� + ���� ��஫�  & \bdot IUserService \\
\hline 
\end{tabular}
\end{table}

\begin{table}[H]
\caption{CRC ����窠 ����� PixelPoint}\label{crc-table-11}
\begin{tabular}{|p{8cm} p{8cm}|} 
\hline class & Clusterable \\
\multicolumn{2}{|c|}{PixelPoint} \\ \hline
\end{tabular}
\begin{tabular}{|p{8cm}|p{8cm}|} 
  ����� ������  & \\
\hline 
\end{tabular}
\end{table}

\begin{table}[H]
\caption{CRC ����窠 ����� TagDto}\label{crc-table-12}
\begin{tabular}{|p{8cm} p{8cm}|} 
\hline class &  \\
\multicolumn{2}{|c|}{TagDto} \\ \hline
\end{tabular}
\begin{tabular}{|p{8cm}|p{8cm}|} 
  DTO - ⥣  & \\
\hline 
\end{tabular}
\end{table}

\begin{table}[H]
\caption{CRC ����窠 ����� ColorTreemapNode}\label{crc-table-13}
\begin{tabular}{|p{8cm} p{8cm}|} 
\hline class &  \\
\multicolumn{2}{|c|}{ColorTreemapNode} \\ \hline
\end{tabular}
\begin{tabular}{|p{8cm}|p{8cm}|} 
  DTO - 㧥� Treemap  & \\
\hline 
\end{tabular}
\end{table}

\begin{table}[H]
\caption{CRC ����窠 ����� UserDto}\label{crc-table-14}
\begin{tabular}{|p{8cm} p{8cm}|} 
\hline class &  \\
\multicolumn{2}{|c|}{UserDto} \\ \hline
\end{tabular}
\begin{tabular}{|p{8cm}|p{8cm}|} 
\multirow{3}{=}{ DTO - ���짮��⥫� } 
& \bdot PasswordMatches \\
& \bdot ValidEmail \\
& \bdot ValidPassword \\
\hline 
\end{tabular}
\end{table}

\begin{table}[H]
\caption{CRC ����窠 ����� ColorCluster}\label{crc-table-15}
\begin{tabular}{|p{8cm} p{8cm}|} 
\hline class &  \\
\multicolumn{2}{|c|}{ColorCluster} \\ \hline
\end{tabular}
\begin{tabular}{|p{8cm}|p{8cm}|} 
  DTO - 梥⮢�� ������  & \\
\hline 
\end{tabular}
\end{table}

\begin{table}[H]
\caption{CRC ����窠 ����� GenericResponse}\label{crc-table-16}
\begin{tabular}{|p{8cm} p{8cm}|} 
\hline class &  \\
\multicolumn{2}{|c|}{GenericResponse} \\ \hline
\end{tabular}
\begin{tabular}{|p{8cm}|p{8cm}|} 
  ��騩 �⢥�  & \\
\hline 
\end{tabular}
\end{table}

\begin{table}[H]
\caption{CRC ����窠 ����� PhotoController}\label{crc-table-17}
\begin{tabular}{|p{8cm} p{8cm}|} 
\hline class &  \\
\multicolumn{2}{|c|}{PhotoController} \\ \hline
\end{tabular}
\begin{tabular}{|p{8cm}|p{8cm}|} 
  ����஫��� ����㯠 � �⮣���  & \\
\hline 
\end{tabular}
\end{table}

\begin{table}[H]
\caption{CRC ����窠 ����� ColorsController}\label{crc-table-18}
\begin{tabular}{|p{8cm} p{8cm}|} 
\hline class &  \\
\multicolumn{2}{|c|}{ColorsController} \\ \hline
\end{tabular}
\begin{tabular}{|p{8cm}|p{8cm}|} 
\multirow{4}{=}{ ����஫��� 梥⮢ } 
& \bdot PhotoGroupColorArea \\
& \bdot ColorTreemapNodeMapper \\
& \bdot ColorsService \\
& \bdot ColorTreemapNode \\
\hline 
\end{tabular}
\end{table}

\begin{table}[H]
\caption{CRC ����窠 ����� AuthController}\label{crc-table-19}
\begin{tabular}{|p{8cm} p{8cm}|} 
\hline class &  \\
\multicolumn{2}{|c|}{AuthController} \\ \hline
\end{tabular}
\begin{tabular}{|p{8cm}|p{8cm}|} 
  ����஫���  ��⥭�䨪�樨  & \bdot ColorTreemapNode \\
\hline 
\end{tabular}
\end{table}

\begin{table}[H]
\caption{CRC ����窠 ����� RegistrationController}\label{crc-table-20}
\begin{tabular}{|p{8cm} p{8cm}|} 
\hline class &  \\
\multicolumn{2}{|c|}{RegistrationController} \\ \hline
\end{tabular}
\begin{tabular}{|p{8cm}|p{8cm}|} 
\multirow{12}{=}{ ����஫��� ॣ����樨 } 
& \bdot UserAccount \\
& \bdot VerificationToken \\
& \bdot OnRegistrationCompleteEvent \\
& \bdot IUserService \\
& \bdot ISecurityUserService \\
& \bdot PasswordDto \\
& \bdot UserDto \\
& \bdot InvalidOldPasswordException \\
& \bdot GenericResponse \\
& \bdot UserService \\
& \bdot PasswordResetToken \\
& \bdot Privilege \\
\hline 
\end{tabular}
\end{table}

\begin{table}[H]
\caption{CRC ����窠 ����� TagsController}\label{crc-table-21}
\begin{tabular}{|p{8cm} p{8cm}|} 
\hline class &  \\
\multicolumn{2}{|c|}{TagsController} \\ \hline
\end{tabular}
\begin{tabular}{|p{8cm}|p{8cm}|} 
\multirow{5}{=}{ ����஫��� ����㯠 � ⥣�� } 
& \bdot ColorTreemapNodeMapper \\
& \bdot PhotoGroupColorArea \\
& \bdot TagsService \\
& \bdot ColorTreemapNode \\
& \bdot TagDto \\
\hline 
\end{tabular}
\end{table}

\begin{table}[H]
\caption{CRC ����窠 ����� FileController}\label{crc-table-22}
\begin{tabular}{|p{8cm} p{8cm}|} 
\hline class &  \\
\multicolumn{2}{|c|}{FileController} \\ \hline
\end{tabular}
\begin{tabular}{|p{8cm}|p{8cm}|} 
\multirow{3}{=}{ ����஫��� ����㯠 � 䠩��� } 
& \bdot UserDetailService \\
& \bdot GenericResponse \\
& \bdot FileStorageService \\
\hline 
\end{tabular}
\end{table}

\begin{table}[H]
\caption{CRC ����窠 ����� UserController}\label{crc-table-23}
\begin{tabular}{|p{8cm} p{8cm}|} 
\hline class &  \\
\multicolumn{2}{|c|}{UserController} \\ \hline
\end{tabular}
\begin{tabular}{|p{8cm}|p{8cm}|} 
  ����஫��� ���짮��⥫��  & \bdot FileStorageService \\
\hline 
\end{tabular}
\end{table}

\begin{table}[H]
\caption{CRC ����窠 ����� ReCaptchaUnavailableException}\label{crc-table-24}
\begin{tabular}{|p{8cm} p{8cm}|} 
\hline class & RuntimeException \\
\multicolumn{2}{|c|}{ReCaptchaUnavailableException} \\ \hline
\end{tabular}
\begin{tabular}{|p{8cm}|p{8cm}|} 
  ����� ������㯭�  & \\
\hline 
\end{tabular}
\end{table}

\begin{table}[H]
\caption{CRC ����窠 ����� UserAlreadyExistException}\label{crc-table-25}
\begin{tabular}{|p{8cm} p{8cm}|} 
\hline class & RuntimeException \\
\multicolumn{2}{|c|}{UserAlreadyExistException} \\ \hline
\end{tabular}
\begin{tabular}{|p{8cm}|p{8cm}|} 
  ���짮��⥫� 㦥 �������  & \\
\hline 
\end{tabular}
\end{table}

\begin{table}[H]
\caption{CRC ����窠 ����� UserNotFoundException}\label{crc-table-26}
\begin{tabular}{|p{8cm} p{8cm}|} 
\hline class & RuntimeException \\
\multicolumn{2}{|c|}{UserNotFoundException} \\ \hline
\end{tabular}
\begin{tabular}{|p{8cm}|p{8cm}|} 
  ���짮��⥫� �� ������  & \\
\hline 
\end{tabular}
\end{table}

\begin{table}[H]
\caption{CRC ����窠 ����� InvalidOldPasswordException}\label{crc-table-27}
\begin{tabular}{|p{8cm} p{8cm}|} 
\hline class & RuntimeException \\
\multicolumn{2}{|c|}{InvalidOldPasswordException} \\ \hline
\end{tabular}
\begin{tabular}{|p{8cm}|p{8cm}|} 
  �᪫�祭�� ��������� �� ����୮ ��������� ��஬ ��஫�  & \\
\hline 
\end{tabular}
\end{table}

\begin{table}[H]
\caption{CRC ����窠 ����� RestResponseEntityExceptionHandler}\label{crc-table-28}
\begin{tabular}{|p{8cm} p{8cm}|} 
\hline class & ResponseEntityExceptionHandler \\
\multicolumn{2}{|c|}{RestResponseEntityExceptionHandler} \\ \hline
\end{tabular}
\begin{tabular}{|p{8cm}|p{8cm}|} 
\multirow{6}{=}{ REST ��ࠡ��稪 �⢥⮢ } 
& \bdot GenericResponse \\
& \bdot InvalidOldPasswordException \\
& \bdot ReCaptchaInvalidException \\
& \bdot UserNotFoundException \\
& \bdot UserAlreadyExistException \\
& \bdot ReCaptchaUnavailableException \\
\hline 
\end{tabular}
\end{table}

\begin{table}[H]
\caption{CRC ����窠 ����� ReCaptchaInvalidException}\label{crc-table-29}
\begin{tabular}{|p{8cm} p{8cm}|} 
\hline class & RuntimeException \\
\multicolumn{2}{|c|}{ReCaptchaInvalidException} \\ \hline
\end{tabular}
\begin{tabular}{|p{8cm}|p{8cm}|} 
  �᪫�祭�� ��������� �� ����୮ ��������� �����  & \\
\hline 
\end{tabular}
\end{table}

\begin{table}[H]
\caption{CRC ����窠 ����� ColorTreemapNodeMapper}\label{crc-table-30}
\begin{tabular}{|p{8cm} p{8cm}|} 
\hline class &  \\
\multicolumn{2}{|c|}{ColorTreemapNodeMapper} \\ \hline
\end{tabular}
\begin{tabular}{|p{8cm}|p{8cm}|} 
\multirow{2}{=}{  ������ ��魮�� �� ⨯� PhotoGroupColorArea � 㧫� ColorTreemap } 
& \bdot PhotoGroupColorArea \\
& \bdot ColorTreemapNode \\
\hline 
\end{tabular}
\end{table}

\begin{table}[H]
\caption{CRC ����窠 ����� TagRepository}\label{crc-table-31}
\begin{tabular}{|p{8cm} p{8cm}|} 
\hline interface & JpaRepository \\
\multicolumn{2}{|c|}{TagRepository} \\ \hline
\end{tabular}
\begin{tabular}{|p{8cm}|p{8cm}|} 
  �������਩ � ⥣���  & \bdot ColorTreemapNode \\
\hline 
\end{tabular}
\end{table}

\begin{table}[H]
\caption{CRC ����窠 ����� PhotoGroupRepository}\label{crc-table-32}
\begin{tabular}{|p{8cm} p{8cm}|} 
\hline interface & JpaRepository \\
\multicolumn{2}{|c|}{PhotoGroupRepository} \\ \hline
\end{tabular}
\begin{tabular}{|p{8cm}|p{8cm}|} 
  �������਩ � ��㯯��� �⮣�䨩  & \bdot ColorTreemapNode \\
\hline 
\end{tabular}
\end{table}

\begin{table}[H]
\caption{CRC ����窠 ����� PhotoGroupColorAreaRepository}\label{crc-table-33}
\begin{tabular}{|p{8cm} p{8cm}|} 
\hline interface & JpaRepository \\
\multicolumn{2}{|c|}{PhotoGroupColorAreaRepository} \\ \hline
\end{tabular}
\begin{tabular}{|p{8cm}|p{8cm}|} 
\multirow{2}{=}{ �������਩ � 梥⮢묨 �����ࠬ� } 
& \bdot PhotoGroup \\
& \bdot PhotoGroupColorArea \\
\hline 
\end{tabular}
\end{table}

\begin{table}[H]
\caption{CRC ����窠 ����� PhotoGroupMetadataRepository}\label{crc-table-34}
\begin{tabular}{|p{8cm} p{8cm}|} 
\hline interface & JpaRepository \\
\multicolumn{2}{|c|}{PhotoGroupMetadataRepository} \\ \hline
\end{tabular}
\begin{tabular}{|p{8cm}|p{8cm}|} 
  �������਩ � ��⠤���묨 �⮣�䨩  & \bdot PhotoGroupColorArea \\
\hline 
\end{tabular}
\end{table}

\begin{table}[H]
\caption{CRC ����窠 ����� PhotoGroupTagRepository}\label{crc-table-35}
\begin{tabular}{|p{8cm} p{8cm}|} 
\hline interface & JpaRepository \\
\multicolumn{2}{|c|}{PhotoGroupTagRepository} \\ \hline
\end{tabular}
\begin{tabular}{|p{8cm}|p{8cm}|} 
\multirow{2}{=}{ �������਩ � ⥣��� ��㯯� �⮣�䨩 } 
& \bdot PhotoGroup \\
& \bdot PhotoGroupTag \\
\hline 
\end{tabular}
\end{table}

\begin{table}[H]
\caption{CRC ����窠 ����� WorldNetClassRepository}\label{crc-table-36}
\begin{tabular}{|p{8cm} p{8cm}|} 
\hline interface & JpaRepository \\
\multicolumn{2}{|c|}{WorldNetClassRepository} \\ \hline
\end{tabular}
\begin{tabular}{|p{8cm}|p{8cm}|} 
  �������਩ � ����ᠬ� WorldNet  & \bdot PhotoGroupTag \\
\hline 
\end{tabular}
\end{table}

\begin{table}[H]
\caption{CRC ����窠 ����� PhotoSingleRepository}\label{crc-table-37}
\begin{tabular}{|p{8cm} p{8cm}|} 
\hline interface & JpaRepository \\
\multicolumn{2}{|c|}{PhotoSingleRepository} \\ \hline
\end{tabular}
\begin{tabular}{|p{8cm}|p{8cm}|} 
\multirow{3}{=}{  �������਩ � �⮣��ﬨ } 
& \bdot PhotoGroup \\
& \bdot PhotoSingle \\
& \bdot PhotoType \\
\hline 
\end{tabular}
\end{table}

\begin{table}[H]
\caption{CRC ����窠 ����� RoleRepository}\label{crc-table-38}
\begin{tabular}{|p{8cm} p{8cm}|} 
\hline interface & JpaRepository \\
\multicolumn{2}{|c|}{RoleRepository} \\ \hline
\end{tabular}
\begin{tabular}{|p{8cm}|p{8cm}|} 
  �������਩ � ஫ﬨ ���짮��⥫��  & \bdot PhotoType \\
\hline 
\end{tabular}
\end{table}

\begin{table}[H]
\caption{CRC ����窠 ����� UserRepository}\label{crc-table-39}
\begin{tabular}{|p{8cm} p{8cm}|} 
\hline interface & JpaRepository \\
\multicolumn{2}{|c|}{UserRepository} \\ \hline
\end{tabular}
\begin{tabular}{|p{8cm}|p{8cm}|} 
  �������਩ � ���짮��⥫ﬨ  & \bdot PhotoType \\
\hline 
\end{tabular}
\end{table}

\begin{table}[H]
\caption{CRC ����窠 ����� PasswordResetTokenRepository}\label{crc-table-40}
\begin{tabular}{|p{8cm} p{8cm}|} 
\hline interface & JpaRepository \\
\multicolumn{2}{|c|}{PasswordResetTokenRepository} \\ \hline
\end{tabular}
\begin{tabular}{|p{8cm}|p{8cm}|} 
\multirow{2}{=}{ �������਩ � ⮪����� ��� ����⠭������� ��஫�� } 
& \bdot PasswordResetToken \\
& \bdot UserAccount \\
\hline 
\end{tabular}
\end{table}

\begin{table}[H]
\caption{CRC ����窠 ����� VerificationTokenRepository}\label{crc-table-41}
\begin{tabular}{|p{8cm} p{8cm}|} 
\hline interface & JpaRepository \\
\multicolumn{2}{|c|}{VerificationTokenRepository} \\ \hline
\end{tabular}
\begin{tabular}{|p{8cm}|p{8cm}|} 
\multirow{2}{=}{ �������਩ � ⮪����� ��� ���䨪�樨 } 
& \bdot UserAccount \\
& \bdot VerificationToken \\
\hline 
\end{tabular}
\end{table}

\begin{table}[H]
\caption{CRC ����窠 ����� PrivilegeRepository}\label{crc-table-42}
\begin{tabular}{|p{8cm} p{8cm}|} 
\hline interface & JpaRepository \\
\multicolumn{2}{|c|}{PrivilegeRepository} \\ \hline
\end{tabular}
\begin{tabular}{|p{8cm}|p{8cm}|} 
  �������਩ � �ࠢ��� ���짮��⥫��  & \bdot VerificationToken \\
\hline 
\end{tabular}
\end{table}

\begin{table}[H]
\caption{CRC ����窠 ����� Post}\label{crc-table-43}
\begin{tabular}{|p{8cm} p{8cm}|} 
\hline class &  \\
\multicolumn{2}{|c|}{Post} \\ \hline
\end{tabular}
\begin{tabular}{|p{8cm}|p{8cm}|} 
\multirow{2}{=}{ ���� } 
& \bdot UserAccount \\
& \bdot PrivacyType \\
\hline 
\end{tabular}
\end{table}

\begin{table}[H]
\caption{CRC ����窠 ����� PhotoType}\label{crc-table-44}
\begin{tabular}{|p{8cm} p{8cm}|} 
\hline enum &  \\
\multicolumn{2}{|c|}{PhotoType} \\ \hline
\end{tabular}
\begin{tabular}{|p{8cm}|p{8cm}|} 
  ��� �⮮��䨨  & \\
\hline 
\end{tabular}
\end{table}

\begin{table}[H]
\caption{CRC ����窠 ����� PasswordResetToken}\label{crc-table-45}
\begin{tabular}{|p{8cm} p{8cm}|} 
\hline class &  \\
\multicolumn{2}{|c|}{PasswordResetToken} \\ \hline
\end{tabular}
\begin{tabular}{|p{8cm}|p{8cm}|} 
  ����� ��� ��� ��஫�  & \bdot PrivacyType \\
\hline 
\end{tabular}
\end{table}

\begin{table}[H]
\caption{CRC ����窠 ����� PhotoSingle}\label{crc-table-46}
\begin{tabular}{|p{8cm} p{8cm}|} 
\hline class &  \\
\multicolumn{2}{|c|}{PhotoSingle} \\ \hline
\end{tabular}
\begin{tabular}{|p{8cm}|p{8cm}|} 
\multirow{2}{=}{ ��⮣��� } 
& \bdot PhotoGroup \\
& \bdot PhotoType \\
\hline 
\end{tabular}
\end{table}

\begin{table}[H]
\caption{CRC ����窠 ����� PhotoGroup}\label{crc-table-47}
\begin{tabular}{|p{8cm} p{8cm}|} 
\hline class &  \\
\multicolumn{2}{|c|}{PhotoGroup} \\ \hline
\end{tabular}
\begin{tabular}{|p{8cm}|p{8cm}|} 
\multirow{2}{=}{ ��㯯� �⮣�䨩 } 
& \bdot UserAccount \\
& \bdot PrivacyType \\
\hline 
\end{tabular}
\end{table}

\begin{table}[H]
\caption{CRC ����窠 ����� PhotoGroupMetadata}\label{crc-table-48}
\begin{tabular}{|p{8cm} p{8cm}|} 
\hline class &  \\
\multicolumn{2}{|c|}{PhotoGroupMetadata} \\ \hline
\end{tabular}
\begin{tabular}{|p{8cm}|p{8cm}|} 
\multirow{2}{=}{ ��⠤���� � �⮣�䨨 } 
& \bdot PhotoGroup \\
& \bdot Camera \\
\hline 
\end{tabular}
\end{table}

\begin{table}[H]
\caption{CRC ����窠 ����� PrivacyType}\label{crc-table-49}
\begin{tabular}{|p{8cm} p{8cm}|} 
\hline enum &  \\
\multicolumn{2}{|c|}{PrivacyType} \\ \hline
\end{tabular}
\begin{tabular}{|p{8cm}|p{8cm}|} 
  ��� access ����  & \\
\hline 
\end{tabular}
\end{table}

\begin{table}[H]
\caption{CRC ����窠 ����� Tag}\label{crc-table-50}
\begin{tabular}{|p{8cm} p{8cm}|} 
\hline class &  \\
\multicolumn{2}{|c|}{Tag} \\ \hline
\end{tabular}
\begin{tabular}{|p{8cm}|p{8cm}|} 
  ���  & \bdot Camera \\
\hline 
\end{tabular}
\end{table}

\begin{table}[H]
\caption{CRC ����窠 ����� Privilege}\label{crc-table-51}
\begin{tabular}{|p{8cm} p{8cm}|} 
\hline class &  \\
\multicolumn{2}{|c|}{Privilege} \\ \hline
\end{tabular}
\begin{tabular}{|p{8cm}|p{8cm}|} 
  �ࠢ� ���짮��⥫�  & \bdot Camera \\
\hline 
\end{tabular}
\end{table}

\begin{table}[H]
\caption{CRC ����窠 ����� PostUserLike}\label{crc-table-52}
\begin{tabular}{|p{8cm} p{8cm}|} 
\hline class &  \\
\multicolumn{2}{|c|}{PostUserLike} \\ \hline
\end{tabular}
\begin{tabular}{|p{8cm}|p{8cm}|} 
\multirow{2}{=}{ ���� ����� } 
& \bdot Post \\
& \bdot UserAccount \\
\hline 
\end{tabular}
\end{table}

\begin{table}[H]
\caption{CRC ����窠 ����� PhotoGroupColorArea}\label{crc-table-53}
\begin{tabular}{|p{8cm} p{8cm}|} 
\hline class &  \\
\multicolumn{2}{|c|}{PhotoGroupColorArea} \\ \hline
\end{tabular}
\begin{tabular}{|p{8cm}|p{8cm}|} 
  ���⮢�� ������  & \bdot UserAccount \\
\hline 
\end{tabular}
\end{table}

\begin{table}[H]
\caption{CRC ����窠 ����� PhotoGroupUserAccess}\label{crc-table-54}
\begin{tabular}{|p{8cm} p{8cm}|} 
\hline class &  \\
\multicolumn{2}{|c|}{PhotoGroupUserAccess} \\ \hline
\end{tabular}
\begin{tabular}{|p{8cm}|p{8cm}|} 
\multirow{2}{=}{ ������ � access ���� } 
& \bdot PhotoGroup \\
& \bdot UserAccount \\
\hline 
\end{tabular}
\end{table}

\begin{table}[H]
\caption{CRC ����窠 ����� PhotoGroupPost}\label{crc-table-55}
\begin{tabular}{|p{8cm} p{8cm}|} 
\hline class &  \\
\multicolumn{2}{|c|}{PhotoGroupPost} \\ \hline
\end{tabular}
\begin{tabular}{|p{8cm}|p{8cm}|} 
\multirow{2}{=}{ ���� ��㯯� �⮣�䨩-���� } 
& \bdot PhotoGroup \\
& \bdot Post \\
\hline 
\end{tabular}
\end{table}

\begin{table}[H]
\caption{CRC ����窠 ����� PhotoGroupTag}\label{crc-table-56}
\begin{tabular}{|p{8cm} p{8cm}|} 
\hline class &  \\
\multicolumn{2}{|c|}{PhotoGroupTag} \\ \hline
\end{tabular}
\begin{tabular}{|p{8cm}|p{8cm}|} 
\multirow{2}{=}{ ��� �⮣�䨨 } 
& \bdot PhotoGroup \\
& \bdot Tag \\
\hline 
\end{tabular}
\end{table}

\begin{table}[H]
\caption{CRC ����窠 ����� PostTag}\label{crc-table-57}
\begin{tabular}{|p{8cm} p{8cm}|} 
\hline class &  \\
\multicolumn{2}{|c|}{PostTag} \\ \hline
\end{tabular}
\begin{tabular}{|p{8cm}|p{8cm}|} 
\multirow{2}{=}{ ��� ���� } 
& \bdot Post \\
& \bdot Tag \\
\hline 
\end{tabular}
\end{table}

\begin{table}[H]
\caption{CRC ����窠 ����� VerificationToken}\label{crc-table-58}
\begin{tabular}{|p{8cm} p{8cm}|} 
\hline class &  \\
\multicolumn{2}{|c|}{VerificationToken} \\ \hline
\end{tabular}
\begin{tabular}{|p{8cm}|p{8cm}|} 
  ����� ��� ���䨪�樨  & \bdot Tag \\
\hline 
\end{tabular}
\end{table}

\begin{table}[H]
\caption{CRC ����窠 ����� WorldNetClass}\label{crc-table-59}
\begin{tabular}{|p{8cm} p{8cm}|} 
\hline class &  \\
\multicolumn{2}{|c|}{WorldNetClass} \\ \hline
\end{tabular}
\begin{tabular}{|p{8cm}|p{8cm}|} 
  ����� WorldNet  & \\
\hline 
\end{tabular}
\end{table}

\begin{table}[H]
\caption{CRC ����窠 ����� Comment}\label{crc-table-60}
\begin{tabular}{|p{8cm} p{8cm}|} 
\hline class &  \\
\multicolumn{2}{|c|}{Comment} \\ \hline
\end{tabular}
\begin{tabular}{|p{8cm}|p{8cm}|} 
\multirow{2}{=}{ �������਩ ���짮��⥫� } 
& \bdot UserAccount \\
& \bdot PhotoGroup \\
\hline 
\end{tabular}
\end{table}

\begin{table}[H]
\caption{CRC ����窠 ����� UserAuthCookie}\label{crc-table-61}
\begin{tabular}{|p{8cm} p{8cm}|} 
\hline class &  \\
\multicolumn{2}{|c|}{UserAuthCookie} \\ \hline
\end{tabular}
\begin{tabular}{|p{8cm}|p{8cm}|} 
  �㪨 ���짮��⥫�  & \bdot PhotoGroup \\
\hline 
\end{tabular}
\end{table}

\begin{table}[H]
\caption{CRC ����窠 ����� Role}\label{crc-table-62}
\begin{tabular}{|p{8cm} p{8cm}|} 
\hline class &  \\
\multicolumn{2}{|c|}{Role} \\ \hline
\end{tabular}
\begin{tabular}{|p{8cm}|p{8cm}|} 
\multirow{2}{=}{ ���� ���짮��⥫� } 
& \bdot UserAccount \\
& \bdot Privilege \\
\hline 
\end{tabular}
\end{table}

\begin{table}[H]
\caption{CRC ����窠 ����� Camera}\label{crc-table-63}
\begin{tabular}{|p{8cm} p{8cm}|} 
\hline class &  \\
\multicolumn{2}{|c|}{Camera} \\ \hline
\end{tabular}
\begin{tabular}{|p{8cm}|p{8cm}|} 
  ��⮪����  & \\
\hline 
\end{tabular}
\end{table}

\begin{table}[H]
\caption{CRC ����窠 ����� PhotoGroupUserLike}\label{crc-table-64}
\begin{tabular}{|p{8cm} p{8cm}|} 
\hline class &  \\
\multicolumn{2}{|c|}{PhotoGroupUserLike} \\ \hline
\end{tabular}
\begin{tabular}{|p{8cm}|p{8cm}|} 
\multirow{2}{=}{ ���� �⮣�䨨 } 
& \bdot PhotoGroup \\
& \bdot UserAccount \\
\hline 
\end{tabular}
\end{table}

\begin{table}[H]
\caption{CRC ����窠 ����� UserAccount}\label{crc-table-65}
\begin{tabular}{|p{8cm} p{8cm}|} 
\hline class &  \\
\multicolumn{2}{|c|}{UserAccount} \\ \hline
\end{tabular}
\begin{tabular}{|p{8cm}|p{8cm}|} 
  ������ ���짮��⥫�  & \bdot UserAccount \\
\hline 
\end{tabular}
\end{table}

\begin{table}[H]
\caption{CRC ����窠 ����� IUserService}\label{crc-table-66}
\begin{tabular}{|p{8cm} p{8cm}|} 
\hline interface &  \\
\multicolumn{2}{|c|}{IUserService} \\ \hline
\end{tabular}
\begin{tabular}{|p{8cm}|p{8cm}|} 
\multirow{5}{=}{ ����䥩� ���짮��⥫�᪮�� �ࢨ� } 
& \bdot PasswordResetToken \\
& \bdot UserAccount \\
& \bdot VerificationToken \\
& \bdot UserDto \\
& \bdot UserAlreadyExistException \\
\hline 
\end{tabular}
\end{table}

\begin{table}[H]
\caption{CRC ����窠 ����� ImageColorClusteriserService}\label{crc-table-67}
\begin{tabular}{|p{8cm} p{8cm}|} 
\hline class &  \\
\multicolumn{2}{|c|}{ImageColorClusteriserService} \\ \hline
\end{tabular}
\begin{tabular}{|p{8cm}|p{8cm}|} 
\multirow{2}{=}{ ��ࢨ� �����ਧ�樨 } 
& \bdot ColorCluster \\
& \bdot PixelPoint \\
\hline 
\end{tabular}
\end{table}

\begin{table}[H]
\caption{CRC ����窠 ����� FileStorageService}\label{crc-table-68}
\begin{tabular}{|p{8cm} p{8cm}|} 
\hline class &  \\
\multicolumn{2}{|c|}{FileStorageService} \\ \hline
\end{tabular}
\begin{tabular}{|p{8cm}|p{8cm}|} 
\multirow{10}{=}{ ������� �ࢨ� } 
& \bdot FileStorageException \\
& \bdot MyFileNotFoundException \\
& \bdot PhotoGroup \\
& \bdot PhotoSingle \\
& \bdot PhotoType \\
& \bdot UserAccount \\
& \bdot FileStorageProperties \\
& \bdot PhotoGroupRepository \\
& \bdot PhotoSingleRepository \\
& \bdot RandomString \\
\hline 
\end{tabular}
\end{table}

\begin{table}[H]
\caption{CRC ����窠 ����� UserService}\label{crc-table-69}
\begin{tabular}{|p{8cm} p{8cm}|} 
\hline class & IUserService \\
\multicolumn{2}{|c|}{UserService} \\ \hline
\end{tabular}
\begin{tabular}{|p{8cm}|p{8cm}|} 
\multirow{9}{=}{ ��ࢨ� ���짮��⥫�� } 
& \bdot PasswordResetTokenRepository \\
& \bdot RoleRepository \\
& \bdot UserRepository \\
& \bdot VerificationTokenRepository \\
& \bdot PasswordResetToken \\
& \bdot UserAccount \\
& \bdot VerificationToken \\
& \bdot UserDto \\
& \bdot UserAlreadyExistException \\
\hline 
\end{tabular}
\end{table}

\begin{table}[H]
\caption{CRC ����窠 ����� TagsService}\label{crc-table-70}
\begin{tabular}{|p{8cm} p{8cm}|} 
\hline class &  \\
\multicolumn{2}{|c|}{TagsService} \\ \hline
\end{tabular}
\begin{tabular}{|p{8cm}|p{8cm}|} 
\multirow{7}{=}{ ��ࢨ� ⥣�� } 
& \bdot PhotoGroupRepository \\
& \bdot PhotoGroupTagRepository \\
& \bdot TagRepository \\
& \bdot PhotoGroup \\
& \bdot PhotoGroupTag \\
& \bdot TagDto \\
& \bdot Tag \\
\hline 
\end{tabular}
\end{table}

\begin{table}[H]
\caption{CRC ����窠 ����� ColorsService}\label{crc-table-71}
\begin{tabular}{|p{8cm} p{8cm}|} 
\hline class &  \\
\multicolumn{2}{|c|}{ColorsService} \\ \hline
\end{tabular}
\begin{tabular}{|p{8cm}|p{8cm}|} 
\multirow{4}{=}{ ��ࢨ� ��� ࠡ��� � 梥⠬� � �����ࠬ� } 
& \bdot PhotoGroup \\
& \bdot PhotoGroupColorArea \\
& \bdot PhotoGroupColorAreaRepository \\
& \bdot PhotoGroupRepository \\
\hline 
\end{tabular}
\end{table}

\begin{table}[H]
\caption{CRC ����窠 ����� MetadataService}\label{crc-table-72}
\begin{tabular}{|p{8cm} p{8cm}|} 
\hline class &  \\
\multicolumn{2}{|c|}{MetadataService} \\ \hline
\end{tabular}
\begin{tabular}{|p{8cm}|p{8cm}|} 
\multirow{2}{=}{ ��ࢨ� ��⠤����� } 
& \bdot PhotoGroupMetadata \\
& \bdot PhotoGroupMetadataRepository \\
\hline 
\end{tabular}
\end{table}

\begin{table}[H]
\caption{CRC ����窠 ����� PhotoProcessorService}\label{crc-table-73}
\begin{tabular}{|p{8cm} p{8cm}|} 
\hline class &  \\
\multicolumn{2}{|c|}{PhotoProcessorService} \\ \hline
\end{tabular}
\begin{tabular}{|p{8cm}|p{8cm}|} 
\multirow{20}{=}{ ��ࢨ� ��ࠡ�⪨ �⮣�䨩 (ᮤ�ন� �� ������� �� ��ࠡ�⪥) } 
& \bdot PhotoGroupTagRepository \\
& \bdot TagRepository \\
& \bdot WorldNetClassRepository \\
& \bdot PhotoGroupTag \\
& \bdot Tag \\
& \bdot WorldNetClass \\
& \bdot ImageClassifier \\
& \bdot Label \\
& \bdot PhotoGroupColorAreaRepository \\
& \bdot PhotoGroupMetadataRepository \\
& \bdot PhotoGroupRepository \\
& \bdot PhotoSingleRepository \\
& \bdot PhotoGroup \\
& \bdot PhotoGroupColorArea \\
& \bdot PhotoSingle \\
& \bdot PhotoType \\
& \bdot ColorCluster \\
& \bdot ColorUtils \\
& \bdot FileStorageService \\
& \bdot ImageColorClusteriserService \\
\hline 
\end{tabular}
\end{table}

\begin{table}[H]
\caption{CRC ����窠 ����� MySimpleUrlAuthenticationSuccessHandler}\label{crc-table-74}
\begin{tabular}{|p{8cm} p{8cm}|} 
\hline class & AuthenticationSuccessHandler \\
\multicolumn{2}{|c|}{MySimpleUrlAuthenticationSuccessHandler} \\ \hline
\end{tabular}
\begin{tabular}{|p{8cm}|p{8cm}|} 
  ��ࠡ��稪 ��⥭�䨪�樨  & \\
\hline 
\end{tabular}
\end{table}

\begin{table}[H]
\caption{CRC ����窠 ����� ISecurityUserService}\label{crc-table-75}
\begin{tabular}{|p{8cm} p{8cm}|} 
\hline interface &  \\
\multicolumn{2}{|c|}{ISecurityUserService} \\ \hline
\end{tabular}
\begin{tabular}{|p{8cm}|p{8cm}|} 
  ����䥩� �ࢨ� ������᭮�� ���짮��⥫��  & \bdot ImageColorClusteriserService \\
\hline 
\end{tabular}
\end{table}

\begin{table}[H]
\caption{CRC ����窠 ����� SessionFilter}\label{crc-table-76}
\begin{tabular}{|p{8cm} p{8cm}|} 
\hline class & Filter \\
\multicolumn{2}{|c|}{SessionFilter} \\ \hline
\end{tabular}
\begin{tabular}{|p{8cm}|p{8cm}|} 
  ������ ��ᨩ  & \\
\hline 
\end{tabular}
\end{table}

\begin{table}[H]
\caption{CRC ����窠 ����� UserSecurityService}\label{crc-table-77}
\begin{tabular}{|p{8cm} p{8cm}|} 
\hline class & ISecurityUserService \\
\multicolumn{2}{|c|}{UserSecurityService} \\ \hline
\end{tabular}
\begin{tabular}{|p{8cm}|p{8cm}|} 
\multirow{3}{=}{ ��ࢨ� ������᭮�� ���짮��⥫�� } 
& \bdot PasswordResetTokenRepository \\
& \bdot PasswordResetToken \\
& \bdot UserAccount \\
\hline 
\end{tabular}
\end{table}

\begin{table}[H]
\caption{CRC ����窠 ����� ImageClassifier}\label{crc-table-78}
\begin{tabular}{|p{8cm} p{8cm}|} 
\hline class &  \\
\multicolumn{2}{|c|}{ImageClassifier} \\ \hline
\end{tabular}
\begin{tabular}{|p{8cm}|p{8cm}|} 
\multirow{2}{=}{ �����䨪��� ����ࠦ���� } 
& \bdot Prediction \\
& \bdot Label \\
\hline 
\end{tabular}
\end{table}

\begin{table}[H]
\caption{CRC ����窠 ����� Label}\label{crc-table-79}
\begin{tabular}{|p{8cm} p{8cm}|} 
\hline class & Comparable \\
\multicolumn{2}{|c|}{Label} \\ \hline
\end{tabular}
\begin{tabular}{|p{8cm}|p{8cm}|} 
  POJO, �������騩 ����⭮��� �ਭ��������� � ������ � �����  & \\
\hline 
\end{tabular}
\end{table}

\begin{table}[H]
\caption{CRC ����窠 ����� KerasInceptionV3Net}\label{crc-table-80}
\begin{tabular}{|p{8cm} p{8cm}|} 
\hline class &  \\
\multicolumn{2}{|c|}{KerasInceptionV3Net} \\ \hline
\end{tabular}
\begin{tabular}{|p{8cm}|p{8cm}|} 
  ����� ���஭��� ��  & \bdot Label \\
\hline 
\end{tabular}
\end{table}

\begin{table}[H]
\caption{CRC ����窠 ����� Prediction}\label{crc-table-81}
\begin{tabular}{|p{8cm} p{8cm}|} 
\hline class &  \\
\multicolumn{2}{|c|}{Prediction} \\ \hline
\end{tabular}
\begin{tabular}{|p{8cm}|p{8cm}|} 
  �।��������� ���஭��� ��  & \bdot Label \\
\hline 
\end{tabular}
\end{table}

\begin{table}[H]
\caption{CRC ����窠 ����� UserDetailService}\label{crc-table-82}
\begin{tabular}{|p{8cm} p{8cm}|} 
\hline class &  \\
\multicolumn{2}{|c|}{UserDetailService} \\ \hline
\end{tabular}
\begin{tabular}{|p{8cm}|p{8cm}|} 
\multirow{5}{=}{ �࠭�� ���짮��⥫� ⥪�饩 ��ᨨ } 
& \bdot CurrentUser \\
& \bdot UserRepository \\
& \bdot Privilege \\
& \bdot Role \\
& \bdot UserAccount \\
\hline 
\end{tabular}
\end{table}

\begin{table}[H]
\caption{CRC ����窠 ����� MetricRegistrySingleton}\label{crc-table-83}
\begin{tabular}{|p{8cm} p{8cm}|} 
\hline class &  \\
\multicolumn{2}{|c|}{MetricRegistrySingleton} \\ \hline
\end{tabular}
\begin{tabular}{|p{8cm}|p{8cm}|} 
  �뢮��� ���ଠ�� � ⥪��� ����� � ���  & \\
\hline 
\end{tabular}
\end{table}

\begin{table}[H]
\caption{CRC ����窠 ����� FileStorageException}\label{crc-table-84}
\begin{tabular}{|p{8cm} p{8cm}|} 
\hline class & RuntimeException \\
\multicolumn{2}{|c|}{FileStorageException} \\ \hline
\end{tabular}
\begin{tabular}{|p{8cm}|p{8cm}|} 
  �᪫�祭�� � 䠩����� �࠭����  & \\
\hline 
\end{tabular}
\end{table}

\begin{table}[H]
\caption{CRC ����窠 ����� MyFileNotFoundException}\label{crc-table-85}
\begin{tabular}{|p{8cm} p{8cm}|} 
\hline class & RuntimeException \\
\multicolumn{2}{|c|}{MyFileNotFoundException} \\ \hline
\end{tabular}
\begin{tabular}{|p{8cm}|p{8cm}|} 
   �᪫�祭�� �� ���������饬 䠩��. �����頥� ��� 404  & \\
\hline 
\end{tabular}
\end{table}

\begin{table}[H]
\caption{CRC ����窠 ����� CurrentUser}\label{crc-table-86}
\begin{tabular}{|p{8cm} p{8cm}|} 
\hline class & User \\
\multicolumn{2}{|c|}{CurrentUser} \\ \hline
\end{tabular}
\begin{tabular}{|p{8cm}|p{8cm}|} 
  ����� ���짮��⥫�, �易���� � ���짮��⥫�� �� �� ��� ��ᨨ  & \bdot UserAccount \\
\hline 
\end{tabular}
\end{table}

\begin{table}[H]
\caption{CRC ����窠 ����� UserValidator}\label{crc-table-87}
\begin{tabular}{|p{8cm} p{8cm}|} 
\hline class & Validator \\
\multicolumn{2}{|c|}{UserValidator} \\ \hline
\end{tabular}
\begin{tabular}{|p{8cm}|p{8cm}|} 
  �������� ���짮��⥫� �� ॣ����樨  & \bdot UserAccount \\
\hline 
\end{tabular}
\end{table}

\begin{table}[H]
\caption{CRC ����窠 ����� ValidPassword}\label{crc-table-88}
\begin{tabular}{|p{8cm} p{8cm}|} 
\hline interface &  \\
\multicolumn{2}{|c|}{ValidPassword} \\ \hline
\end{tabular}
\begin{tabular}{|p{8cm}|p{8cm}|} 
  �������� ��஫�  & \bdot UserAccount \\
\hline 
\end{tabular}
\end{table}

\begin{table}[H]
\caption{CRC ����窠 ����� EmailValidator}\label{crc-table-89}
\begin{tabular}{|p{8cm} p{8cm}|} 
\hline class & ConstraintValidator \\
\multicolumn{2}{|c|}{EmailValidator} \\ \hline
\end{tabular}
\begin{tabular}{|p{8cm}|p{8cm}|} 
  �������� email  & \bdot UserAccount \\
\hline 
\end{tabular}
\end{table}

\begin{table}[H]
\caption{CRC ����窠 ����� PasswordMatches}\label{crc-table-90}
\begin{tabular}{|p{8cm} p{8cm}|} 
\hline interface &  \\
\multicolumn{2}{|c|}{PasswordMatches} \\ \hline
\end{tabular}
\begin{tabular}{|p{8cm}|p{8cm}|} 
  ����� ��� ��� ����୮�� ����� ��஫��  & \bdot UserAccount \\
\hline 
\end{tabular}
\end{table}

\begin{table}[H]
\caption{CRC ����窠 ����� ValidEmail}\label{crc-table-91}
\begin{tabular}{|p{8cm} p{8cm}|} 
\hline interface &  \\
\multicolumn{2}{|c|}{ValidEmail} \\ \hline
\end{tabular}
\begin{tabular}{|p{8cm}|p{8cm}|} 
  �������� email  & \bdot UserAccount \\
\hline 
\end{tabular}
\end{table}

\begin{table}[H]
\caption{CRC ����窠 ����� EmailExistsException}\label{crc-table-92}
\begin{tabular}{|p{8cm} p{8cm}|} 
\hline class & Throwable \\
\multicolumn{2}{|c|}{EmailExistsException} \\ \hline
\end{tabular}
\begin{tabular}{|p{8cm}|p{8cm}|} 
  �᪫�祭�� �� 㦥 �������饬 ���짮��⥫� � ����� email  & \\
\hline 
\end{tabular}
\end{table}

\begin{table}[H]
\caption{CRC ����窠 ����� PasswordMatchesValidator}\label{crc-table-93}
\begin{tabular}{|p{8cm} p{8cm}|} 
\hline class & ConstraintValidator \\
\multicolumn{2}{|c|}{PasswordMatchesValidator} \\ \hline
\end{tabular}
\begin{tabular}{|p{8cm}|p{8cm}|} 
\multirow{2}{=}{ �������� ��� ��� ����୮�� ����� �࠮�� } 
& \bdot UserDto \\
& \bdot PasswordMatches \\
\hline 
\end{tabular}
\end{table}

\begin{table}[H]
\caption{CRC ����窠 ����� PasswordConstraintValidator}\label{crc-table-94}
\begin{tabular}{|p{8cm} p{8cm}|} 
\hline class & ConstraintValidator \\
\multicolumn{2}{|c|}{PasswordConstraintValidator} \\ \hline
\end{tabular}
\begin{tabular}{|p{8cm}|p{8cm}|} 
  �������� ��஫��  & \bdot PasswordMatches \\
\hline 
\end{tabular}
\end{table}

\begin{table}[H]
\caption{CRC ����窠 ����� ServiceConfig}\label{crc-table-95}
\begin{tabular}{|p{8cm} p{8cm}|} 
\hline class &  \\
\multicolumn{2}{|c|}{ServiceConfig} \\ \hline
\end{tabular}
\begin{tabular}{|p{8cm}|p{8cm}|} 
  ���䨣��樮��� 䠩� ��� �ࢨᮢ  & \\
\hline 
\end{tabular}
\end{table}

\begin{table}[H]
\caption{CRC ����窠 ����� SecSecurityConfig}\label{crc-table-96}
\begin{tabular}{|p{8cm} p{8cm}|} 
\hline class & WebSecurityConfigurerAdapter \\
\multicolumn{2}{|c|}{SecSecurityConfig} \\ \hline
\end{tabular}
\begin{tabular}{|p{8cm}|p{8cm}|} 
  ���䨣��樮��� 䠩� Spring security  & \bdot PasswordMatches \\
\hline 
\end{tabular}
\end{table}

\begin{table}[H]
\caption{CRC ����窠 ����� MvcConfig}\label{crc-table-97}
\begin{tabular}{|p{8cm} p{8cm}|} 
\hline class & WebMvcConfigurer \\
\multicolumn{2}{|c|}{MvcConfig} \\ \hline
\end{tabular}
\begin{tabular}{|p{8cm}|p{8cm}|} 
\multirow{2}{=}{ ���䨣��樮��� 䠩� MVC spring } 
& \bdot EmailValidator \\
& \bdot PasswordMatchesValidator \\
\hline 
\end{tabular}
\end{table}

\begin{table}[H]
\caption{CRC ����窠 ����� SetupDataLoader}\label{crc-table-last}
\begin{tabular}{|p{8cm} p{8cm}|} 
\hline class & ApplicationListener \\
\multicolumn{2}{|c|}{SetupDataLoader} \\ \hline
\end{tabular}
\begin{tabular}{|p{8cm}|p{8cm}|} 
\multirow{6}{=}{ ���樠������ ������ �� ����᪥ �ࢨ� } 
& \bdot PrivilegeRepository \\
& \bdot RoleRepository \\
& \bdot UserRepository \\
& \bdot Privilege \\
& \bdot Role \\
& \bdot UserAccount \\
\hline 
\end{tabular}
\end{table}


\clearpage