\section{Аналитическая часть}\label{analytics}

\subsection{Сравнительный анализ конкурирующих решений} \label{comparsion}

Большинство фотографов хранят исходники или обработанные фотографии на локальных носителях.
Данный способ хранения имеет ряд существенных недостатков, как например, ненадежность массово используемых решений и трудность организации фотографий стандартными средствами ОС.

На сегодняшний день существует большое количество решений, готовых предоставить услуги по хранению фотографий в сети.
Данная услуга имеет массу плюсов для конечных пользователей, и самый главный  - возможность представить фотографии в виде Web-альбома, созданного с помощью соответствующего решения, а не просто в виде бессистемного набора изображений.
Но подобный вариант автоматически накладывает определенные сложности. 
Это относится к тому случаю, когда пользователь предназначает созданный фотоальбом не только для того, чтобы альбом просматривали, знающие его Web-адрес, но и для расширенного круга посетителей, например для выяснения мнения профессионалов по поводу качества изображений.
Также некоторую трудность представляет организация хранилища фотографий в данных решениях, а именно наполнение Web-альбомов и принятие решения о принадлежности фотографии к альбому. 
Нередко пользователи затрудняются, когда фотографию можно отнести в разные альбомы. Например, фотографию человека крупным планом, держащего щенка на руках, можно отнести как к условному альбому "домашние животные", так и к альбомам "люди", "портреты", "собаки", "щенки".
Каждый день появляются новые сервисы для хранения фотографий, сильно похожие на существующие аналоги.
Сервисы соревнуются в типовых характеристиках, таких, как максимальное разрешение загружаемой фотографии, максимально возможное количество загруженных фотографий, тем самым они не педлагают пользователям нового функционала, связанного с организацией хранилища фотографий.

\subsubsection{Яндекс.Диск}
Яндекс.Диск является одним из самых популярных сервисов среди русскоязычного сообщества фотографов. 
Данный хостинг предоставляет 10 гигабайт бесплатного хранилища при регистрации. 
Причём загружать можно как фотографии в формате jpeg, так и исходники фотографий без обработки непосредственно с фотоаппарата. 
Данный хостинг не позиционируется, как решение, предназначенное только для фотографий. 
На Яндекс.Диск можно загружать файлы любых форматов. 
Однако универсальность всегда накладывает некоторые ограничения. 
В случае с Яндекс.Диском это отсутствие взаимодействия между фотографами в виде популярных или актуальных фотографий, отсутствие автотегирования и практически все те же ограничения, что и при хранении фотографий на электронных носителях, а именно трудности, связанные с организацией и упорядочиванием фотографий в хранилище. 
Стоит отметить, что сервис Яндекс.Диск имеет простой и понятный интерфейс ввиду классической абстракции хранения фотографий в альбомах в виде файловой системы с альбомами в виде каталогов и фотографиями в виде файлов с возможностью просмотра и редактирования фотографий без их непосредственной загрузки на электронные носители. Интерфейс сервиса Яндекс.Диск представлен на рисунке \ref{yandex-disk}

\subsubsection{Apple Фотографии}
Одним из самых удобных на сегодняшний день сервисов для хранения фотографий является Фотографии от Apple.
Самый большой недостаток данного сервиса - отсутствие кросплатформенности. 
Фотографии работают только на операционных системах iOS и macOS, что в свою очередь не даёт возможности воспользоваться данным программным приложением начинающим фотографам с ограниченным бюджетом ввиду дороговизны устройств компании Apple.
Однако это не мешает данному сервису пользоваться популярностью у фотографов, имеющих в наличии технику от компании Apple. 
Приложение имеет возможность синхронизации с другими устройствами и возможность резервного копирования фотографий в облако, однако для доступа к данному функционалу необходимо приобретать пространство в облачном хранилище за отдельную плату.
Решение имеет функционал автотегирования за счёт ресурсов пользовательского устройства и функции распознавания лиц на фотографиях.
Частично имеется функция ручной коррекции результатов индексации фотохранилища, а именно функция ручной коррекции результата распознавания лиц.
Сервис работает только через официальные клиентские приложения.
Веб интерфейса у сервиса нет.
Интерфейс клиентского приложения представлен на рисунке \ref{apple-photos}

\subsubsection{Instagram}
Самой массовой соцсетью, совмещающей в себе часть функций фотохостинга является instagram.
Сервис является популярным во всём мире и насчитывает более 1 миллиарда пользователей.
В данной социальной сети одними из первых были внедрены тэги, связывающие публикации по всему сервису.
Однако сервис не позволяет добавить более 30 тегов к одной публикации. 
Длинна текста публикации ограничена, а поиск можно производить только глобальный в рамках всего сервиса и только по одному тегу.
Сервис не предоставляет инструментов для автотегирования и автоиндексирования фотографий.
В следствие чего поиск опубликованных фотографий по тегам явялется весьма затруднительным, а критерии поиска, которые можно сформировать, весьма ограничены. 
Таким образом сервис является в первую очередь социальной сетью с лентой актуальных публикаций и только потом фотохостингом.
Из недостатков можно также отметить низкое качество загруженных фотогафий после сжатия на серверах instagram ввиду ориентированности сервиса на мобильные устройства, отсутствие какой-либо защиты фотографий от копирования и скудные настройки приватности фотографии.
Конкретную фотографию можно скрыть для всех пользователей, не скрытые данным образом фотографии можно показать только друзьям или всем пользователям социальной сети.
Тем не менее, интерфейс данной социальной сети, представленный на рисунке \ref{instagram} интуитивно понятен и прост в освоении.

\subsubsection{500px}
Сервис 500px преследует цель создания удобного облачного хранилища фотографий, преимущественно для любителей.
Данный сервис имеет удобный алгоритм популярности и актуальности фотографий.
Это может быть востребовано среди фотографов, ищущих критики и оценки своих фотографий со стороны других пользователей сервиса.
На сайте явно прослеживаются коммерческие цели компании.
Они проявляются в сильных ограничениях пользования сервисом у пользователей на бесплатной основе, например возможность загружать не более 7 фотографий в неделю.
Продукт имеет низкий функционал.
Так, например, отсутствует возможность размещения исходников фотографий в каком-либо виде.
Все фотографии, загруженные на данный фотохостинг сжимаются
У сервиса отсутствует фоторедактор загруженных в каком-либо виде.
Присутствует возможность монетизации фотографий, навигации с помощью клавиш, извлечения записанных в фотографию метаданных, однако отсутствует какая-либо настройка приватности фотографий.
Основной интерфейс фотохостинга 500px представлен на рисунке \ref{500px}

\subsubsection{Flickr}
Flickr по концепции очень похож на ранее названный 500px.
Несмотря на это сервис является более ориентированным на пользователей, чем его идейный аналог.
Одной из главных и отличительных особенностей сайта является отсутствие рекламы даже для пользователей, не купивших платную подписку. 
В сервисе на одном экране помещается максимально возможное количество контента среди всех конкурентов.
На сайте есть такие приятные функции, как скачивание фотографии в нескольких размерах, замена оригинала фотографии, встроенный фоторедактор, настройки лицензирования, чтение метаинформации из фотографий, теги и лента актуальных публичных фотографий.
Однако на сайте отсутствует в каком-либо виде автоиндексация фотографий.
Также нельзя загружать исходники фотографий.
Хранилище необходимо организовывать полностью вручную.
У данного фотохостинга прослеживается коммерческая составляющая, которая проявляется также в виде ограничения на количество загружаемых фотографий.
В данном случае количество фиксированное и не должно превышать 1 тысячу фотографий.
Веб интерфейс решения представлен на рисунке \ref{flickr}

\subsubsection{Google Photo}

Одной из функций, улучшающих опыт использования от сервиса была бы возможность в автоматическом, или полуавтоматическом режиме организовывать хранилище фотографий, тем самым снимая данную нагрузку с пользователя.
Это возможно благодаря индексации фотографий в сервисе сразу после их загрузки. 
Фотография классифицируется по нескольким наборам признаков и сервис сохраняет индексированную информацию о принадлежности фотографии к набору классов.
Предполагается возможность формирования альбомов с фотографиями "на лету", после ввода пользователем общих критериев, объединяющих сразу несколько фотографий по проиндексированным при загрузке признакам.

В настоящий момент на рынке не так много решений по хранению фотографий, классифицирующих или категоризирующих изображения.
Подобных решений по хранению фотографий непосредственно в сети еще меньше.
Одними из немногих и самым популярным среди таковых является сервис Google Photo. 
В Google Photo можно бесплатно загружать фотографии, размер которых не превышает 16 МПикс в формате jpeg. 
Сразу после загрузки фотографии индексируются для дальнейшего поиска по ключевым словам.
При этом результаты индексирования остаются полностью скрытыми от пользователя. 
В таком случае, если алгоритм индексации ошибся, пользователь никогда об этом не узнает, или по крайней мере, никак не сможет на это повлиять. 
Например, нейронная сеть может распознать класс там, где его на самом деле нет или не распознать класс там, где он присутствует.
Тогда по определённому поисковому запросу пользователь не увидит нужных ему фотографий, или увидит фотографии, ошибочно отнесённые к поисковому классу.
Интерфейс сервиса Google Photo, содержащий ошибочный результат поиска представлен на рисунке \ref{google-photo}

Сравнение ведущих решений для хранения фотографий представлены в таблице \ref{comp-table}
%21cm*29.7cm
%21-3-1=17cm (-2.5cm)
%29.7-2-2=25.7 ()
%\geometry{left=3cm}
%\geometry{right=1cm}
%\geometry{top=2cm}
%\geometry{bottom=2cm}

\begin{landscape}
\begin{table}[H]
  \caption{Функциональность конкурирующих продуктов}\label{comp-table}
  \smalltable
  \begin{tabular}{|p{5.8cm}|p{2.8cm}|p{2.8cm}|p{2.8cm}|p{2.8cm}|p{2.8cm}|p{2.8cm}|} %14.5cm*
  \hline Сравнение особенностей & Конкурент А фотохостинг & Конкурент В фотохостинг & Конкурент C фотохостинг & Конкурент D фотохостинг & Конкурент E фотохостинг & Конкурент F фотохостинг \\ 
  \hline URL компании & flickr.com & 500px.com & photos.google.com & disk.yandex.ru & apple.com & instagram.com \\ 
  \hline Классификация продукта & Фотохостинг-соцсеть & Фотохостинг-соцсеть & Фотохостинг & Фотохостинг & Локальный фотохостинг & Соцсеть-фотохостинг \\ 
  \hline Варианты клиентов & Android, iOS, web & Android, iOS, web & Android, iOS, web & Android, iOS, web & macOS, iOS & iOS, android \\ 
  \hline Объем бесплатного хранилища & 1000 фотографий & 7 фотографий в неделю & ∞ if size <16Mp & 10 gb & Локальное хранилище & ∞ но низкое качество \\ 
  \hline Загрузка исходников фотографиии & - & - & + & + & + & - \\ 
  \hline Возможность тегирования фотографий & + & + & +/- & - & +/- & + \\ 
  \hline Настройки приватности для фотографии & + & - & + & + & - & +/- \\ 
  \hline Настройки лицензирования & + & + & - & - & - & - \\ 
  \hline Встроенный фоторедактор & + & - & + & + & + & + \\ 
  \hline Комментирование & + & + & - & + & - & + \\ 
  \hline Сохранение фотографий других людей & + & + & - & - & - & + \\ 
  \hline Скачивание оригинала фотографии & + & - & + & + & + & - \\ 
  \hline Скачивание фотографии в нескольких размерах & + & - & - & - & - & - \\ 
  \hline Навигация с помощью клавиш & + & + & + & - & + & - \\ 
  \hline Просмотр ссводной статистики & + & + & - & - & - & + \\ 
  \hline Замена фотографии & + & - & - & - & - & - \\ 
  \hline Авторасстановка тегов & - & - & + & - & - & - \\ 
  \hline Распознавание лиц & - & - & + & - & + & - \\ 
  \hline Распознавание разных людей & - & - & - & - & + & - \\ 
  \hline Модерирование комментариев & + & + & - & - & - & + \\ 
  \hline Автозащита фотографий внутри сервиса от копирования & - & - & - & - & - & - \\ 
  \hline Чтение информации из exif & + & + & - & - & + & - \\ 
  \hline Лента актуальных публичных фотографий & + & + & - & - & - & + \\ 
  \hline Монетизация & - & + & - & - & - & +/- \\ 
  \hline Суммарное количество особенностей на сайте & 15 & 11 & 9 & 6 & 7 & 8 \\ 
  \hline Рейтинг эффективности & ****** & ***** & **** & * & ** & *** \\
  \hline
  \end{tabular}
\end{table}
\end{landscape}

\subsection{Постановка задачи}\label{problem-formulation}
После проведённого анализа существующих программных продуктов было принято решение о разработке собственного интернет-сервиса для хранения фотографий с элементами социальной сети. 
Предполагается, что фотохостинг будет реализовывать функционал по автоматической индексации фотографий при загрузке с возможностью в дальнейшем осуществлять поиск фотографий в хранилище по индексированным данным.

Для достижения поставленной цели необходимо решить следующие задачи:
\begin{enumerate}
    \item изучить предметную область;
    \item проанализировать существующие решения;
    \item спроектировать схему индексации;
    \item спроектировать архитектуру приложения;
    \item выбрать технологии реализации;
    \item обосновать выбранные средства программной реализации;
    \item реализовать прототип программного решения;
    \item протестировать прототип программного решения на конечных пользователях;
    \item доработать прототип до стадии готового продукта;
    \item запустить продукт в эксплуатацию.
\end{enumerate}

Онлайн сервис для публикации фотографий должен обладать следующими характеристиками:
\begin{itemize}
    \item возможность загрузки и хранения фотографий с возможностью просмотра и дальнейшего скачивания;
    \item организация хранилища фотографий пользователей в автоматическом или полуавтоматическом режиме с возможностью осуществления дальнейшего поиска по необходимым критериям;
    \item возможность публикации фотографий в сети интернет;
    \item индексирование загружаемых фотографий по цветам для осуществления возможности дальнейшего поиска фотографий по цветам;
    \item возможность комментирования и оценивания фотографий, загруженных в сервис;
    \item возможность обмена сообщениями с другими пользователями;
    \item автоматическая классификация фотографий по набору тегов;
    \item задание пользовательских тегов;
    \item поиск фотографии по кластерам цветов.
\end{itemize}

Для обеспечения безопасной и непрерывной работы пользователями программное решение должно отвечать следующим требованиям:
\begin{itemize}
    \item парольная аутентификация и разграничение прав доступа пользователей;
    \item регистрация пользователей с подтверждением прохождения регистрации посредством email или смс;
    \item протоколирование действий пользователей.
\end{itemize}

\clearpage