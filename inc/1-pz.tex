\section{Аналитическая часть}\label{analytics}

\subsection{Постановка задачи}\label{problem-formulation}

Конечная цель проектирования --- разработка архитектуры программного решения, предназначенного для хранения и публикации фотографий в сети интернет.

Онлайн сервис для публикации фотографий должен обладать следующими характеристиками:
\begin{itemize}
    \item возможность загрузки и хранения фотографий с возможностью просмотра и дальнейшего скачивания;
    \item организация хранилища фотографий пользователей в автоматическом или полуавтоматическом режиме с возможностью осуществления дальнейшего поиска по необходимым критериям;
    \item возможность публикации фотографий в сети интернет;
    \item индексирование загружаемых фотографий по цветам для осуществления возможности дальнейшего поиска фотографий по цветам;
    \item возможность комментирования и оценивания фотографий, загруженных в сервис;
    \item возможность обмена сообщениями с другими пользователями;
    \item автоматическая классификация фотографий по набору тегов;
    \item задание пользовательских тегов;
    \item поиск фотографии по кластерам цветов.
\end{itemize}

Для обеспечения безопасной и непрерывной работы пользователями программное решение должно отвечать следующим требованиям:
\begin{itemize}
    \item Парольная аутентификация и разграничение прав доступа пользователей;
    \item Регистрация пользователей с подтверждением прохождения регистрации посредством email или смс;
    \item Протоколирование действий пользователей.
\end{itemize}

Для разработки фотохостинга необходима реализация следующих этапов:
\begin{itemize}
    \item изучение и анализ предметной области;
    \item выбор технологий реализации;
    \item проектирование архитектуры приложения;
    \item обоснование выбора средств программной реализации;
    \item реализация продукта минимальной жизнедеятельности;
    \item запуск в эксплуатацию продукта минимальной жизнедеятельности;
    \item тестирование продукта минимальной жизнедеятельности на конечных пользователях;
    \item реализация продукта;
    \item запуск в эксплуатацию продукта.
\end{itemize}

\subsection{Сравнительный анализ конкурирующих решений} \label{comparsion}

На сегодняшний день существует большое количество решений, готовых предоставить услуги по хранению фотографий в сети.
Данная услуга имеет массу плюсов для конечных пользователей, и самый главный  - возможность представить фотографии в виде Web-альбома, созданного с помощью соответствующего решения, а не просто в виде бессистемного набора изображений.
Но подобный вариант автоматически накладывает определенные сложности. 
Это относится к тому случаю, когда пользователь предназначает созданный фотоальбом не только для того, чтобы альбом просматривали, знающие его Web-адрес, но и для расширенного круга посетителей, например для выяснения мнения профессионалов по поводу качества изображений.
В этом случае не избежать стандартной процедуры регистрации и раскрутки сайта, поскольку сделать сайт посещаемым — это отдельная и серьезная работа, которая потребует немало времени и специальных знаний.
Каждый день появляются новые сервисы для хранения фотографий, сильно похожие на существующие аналоги.
Сервисы соревнуются в типовых характеристиках, таких, как максимальное разрешение загружаемой фотографии, максимально возможное количество загруженных фотографий, тем самым они не педлагают пользователям нового функционала, связанного с организацией хранилища фотографий.
Сравнение ведущих решений для хранения фотографий представлены в таблице \ref{comp-table}

\begin{table}[H]
  \caption{Функциональность конкуриррующих продуктов}\label{comp-table}
  \begin{spacing}{1}
  \small
  \begin{tabular}{|p{3cm}|p{2.5cm}|p{2.5cm}|p{2.5cm}|p{2.5cm}|p{2cm}|}
  \hline Сравнение особенностей & Конкурент А фотохостинг & Конкурент В фотохостинг & Конкурент C фотохостинг & Конкурент D фотохостинг & Конкурент E фотохостинг \\ 
  \hline URL компании & flickr.com & 500px.com & photos.google.com & disk.yandex.ru & apple.com \\ 
  \hline Классификация продукта & Фотохостинг-соцсеть & Фотохостинг-соцсеть & Фотохостинг & Фотохостинг & Локальный фотохостинг \\ 
  \hline Варианты клиентов & Android, iOS, web & Android, iOS, web & Android, iOS, web & Android, iOS, web & OSX \\ 
  \hline Объем бесплатного хранилища & 1 tb & 7 фото в неделю & ∞ if size <16Mp & 10 gb & Локальное хранилище \\ 
  \hline Загрузка исходников фотографиии & - & - & + & + & + \\ 
  \hline Теги & + & + & + & - & - \\ 
  \hline Настройки приватности для фотографии & + & - & + & + & - \\ 
  %\hline Настройки лицензирования & + & + & - & - & - \\ 
  %\hline Встроенный фоторедактор & + & - & + & + & + \\ 
  \hline Комментирование & + & + & - & - & - \\ 
  \hline Сохранение фотографий других людей & + & + & - & - & - \\ 
  \hline Скачивание оригинала фотографии & + & - & + & + & + \\ 
  %\hline Скачивание фотографии в нескольких размерах & + & - & - & - & - \\ 
  %\hline Навигация с помощью клавиш & + & + & + & - & + \\ 
  %\hline Просмотр ссводной статистики & + & + & - & - & - \\ 
  %\hline Замена фотографии & + & - & - & - & - \\ 
  \hline Авторасстановка тегов & - & - & + & - & - \\ 
  \hline Распознавание лиц & - & - & + & - & + \\ 
  \hline Распознавание разных людей & - & - & - & - & + \\ 
  %\hline Модерирование комментариев & + & + & - & - & - \\ 
  %\hline Автозащита фотографий внутри сервиса от копирования & - & - & - & - & - \\ 
  \hline Чтение информации из exif & + & + & - & - & + \\ 
  %\hline Лента актуальных публичных фотографий & + & + & - & - & - \\ 
  %\hline Монетизация & - & + & - & - & - \\ 
  \hline Суммарное количество особенностей на сайте & 15 & 11 & 9 & 5 & 7 \\ 
  \hline Рейтинг эффективности & ***** & **** & *** & * & ** \\
  \hline
  \end{tabular}
  \end{spacing}
\end{table}

\subsection{Организация хранилища фотографий}

Большинство фотографов хранят исходники или обработанные фотографии на локальных носителях.
Данный способ хранения имеет ряд существенных недостатков, как например, ненадежность массово используемых решений и трудность организации фотографий стандартными средствами ОС.
А инструменты, предоставляющие хранение фотографий в сети сильно ограничены.
Одной из функций, улучшающих опыт использования от сервиса была бы возможность в автоматическом, или полуавтоматическом режиме организовывать хранилище фотографий, тем самым снимая данную нагрузку с пользователя.
Это возможно благодаря индексации фотографий в сервисе сразу после их загрузки. 
Фотография классифицируется и помечается тегом принадлежности к набору классов.
Также выделяются цветовые кластеры на фотографии для дальнейшей возможности поиска фотографии по набору преобладающих на ней цваетов.
Предполагается возможность формирования альбомов с фотографиями "на лету", после ввода пользователем общих критериев, объединяющих сразу несколько фотографий по проиндексированным при загрузке признакам.


В настоящий момент на рынке не так много решений по хранению фотографий, классифицирующих или категоризирующих изображения.
Подобных решений по хранению фотографий непосредственно в сети еще меньше.
Одними из немногих и самым популярным среди таковых является сервис Google Photo. 
В Google Photo можно бесплатно загружать фотографии, размер которых не превышает 16 МПикс в формате jpeg. 
Сразу после загрузки фотографии индексируются для дальнейшего поиска по ключевым словам.
При этом результаты индексирования остаются полностью скрытыми от пользователя. 
В таком случае, если нейронная сеть ошиблась, пользователь никогда об этом не узнает, или по крайней мере, никак не сможет на это повлиять. 
Например, нейронная сеть может распознать класс там, где его на самом деле нет или не распознать класс там, где он присутствует.

\clearpage