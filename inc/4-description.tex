\section{Описание виртуальной инфраструктуры}

В разделе описания виртуальной инфраструктуры представлена общая схема инфраструктуры, схемы мониторинга, репликации и балансировки DNS, используемые технологии виртуализации, алгоритмы взаимодействия пользователя и компонентов инфраструктуры.

\subsection{Общая схема инфраструктуры}

Общая схема инфраструктуры представлена на рис. \ref{infrast-scheme} и чертеже \href{extra/drafts/SevGU_09.03.01.15.A1.pdf}{СевГУ 09.03.01.15.А1}.
\addimghere{infrast-scheme}{1}{Общая схема инфраструктуры}{infrast-scheme}

В дата-центре 1 располагается основная часть инфраструктуры: сервера shared-хостинга, виртуализации OpenVZ и KVM, сервер резервного копирования и выделенные сервера клиентов.

В дата-центре 2 на двух арендованных виртуальных машинах находится один из подчиненных DNS-серверов, а также сервер мониторинга.

На физических серверах располагаются важные элементы инфраструктуры: главный и один подчиненный DNS-сервера, система биллинга и система управления IP-адресами.

\subsection{Схема мониторинга}

Схема мониторинга инфраструктуры представлена на рис. \ref{nagios-munin-scheme}.
\addimghere{nagios-munin-scheme}{1}{Схема мониторинга инфраструктуры}{nagios-munin-scheme}

В случае возникновения проблемы на одном из серверов, Nagios фиксирует изменения и отправляет текстовые уведомления по SMS и электронной почте дежурному администратору, который следит за системой мониторинга.
Nagios работает по протоколу NRPE (Nagios Remote Plugin Executor) и слушает порт 5666.

Munin имеет возможность визуализировать состояние ресурсов с помощью графиков, наблюдая за графиками можно делать выводы о событиях, происходящих в инфраструктуре.
Munin работает по протоколу MGF (Munin Graphing Framework) на порту 4949.

Помимо физических серверов, мониторятся также виртуальные машины, на которых располагаются сервисы инфраструктуры, а также некоторые виртуальные машины клиентов, по требованию.

\subsection{Схема репликации и балансировки нагрузки DNS-серверов}

С целью обеспечения отказоустойчивости системы была разработана следующая схема репликации и балансировки нагрузки DNS-серверов.
Схема представлена на рис. \ref{dns-scheme}.
\addimghere{dns-scheme}{1}{Схема репликации и балансировки нагрузки DNS-серверов}{dns-scheme}

В каждой зоне есть один главный сервер имен на котором хранится официальная копия данных зоны.
Администратор модифицирует информацию, касающуюся зоны, редактируя файлы главного сервера \cite{unix-handbook}.
Подчиненный (slave) сервер копирует свои данные с главного сервера посредством операции, называемой передачей зоны (zone transfer).
В зоне имеется два подчиненных сервера, один из которых располагается в другом дата-центре.

На главном сервере установлен DNS-сервер PowerDNS.
На подчиненных серверах --- BIND 9.

Балансировка нагрузки осуществляется по алгоритму Round Robin (алгоритм кругового обслуживания).
Алгоритм представляет собой перебор по круговому циклу: первый запрос передается одному серверу, затем следующий запрос передается другому и так до достижения последнего сервера \cite{selectel}.

\subsection{Используемые технологии виртуализации}

На основе обзора литературы, проведенного в пункте \ref{literature}, выбраны и внедрены технологии виртуализации OpenVZ (для организации контейнеров) и KVM (для организации виртуальных машин).

\subsection{Алгоритмы функционирования инфраструктуры} \label{algo}

Взаимодействие пользователя с инфраструктурой происходит по сценариям, представленным в пункте \ref{user-infr}.

В пункте \ref{infr-comp} представлены сценарии взаимодействия компонентов инфраструктуры.

\subsubsection{Схема взаимодействия пользователя с инфраструктурой} \label{user-infr}

Сценарий 1. Заказ облачного сервиса:
\begin{enumerate}
  \item пользователь обращается к инфраструктуре через веб-интерфейс с целью заказа услуги;
  \item биллинг регистрирует нового пользователя;
  \item заключается договор на предоставление услуги с указанием реквизитов пользователя;
  \item подсистема управления услугами производит выделение физического либо виртуального сервера (контейнера, виртуальной машины), либо иной услуги с требуемыми характеристиками;
  \item переход к сценарию 2.
\end{enumerate}

Сценарий 2. Использование сервиса:
\begin{enumerate}
  \item клиент пользуется предоставленным сервисом некоторый период времени, используя веб-интерфейс (VNC, SSH);
  \item биллинг выставляет счет пользователю за оказанные услуги;
  \item пользователь оплачивает сервис;
  \item переход к пункту а, пока не наступит событие, соответствующее сценариям 3, 4, 5, 6 или 7.
\end{enumerate}

Сценарий 3. Изменение условий предоставления сервиса:
\begin{enumerate}
  \item пользователь желает изменить условия предоставления сервиса, либо отказаться от него;
  \item биллинг регистрирует запрос;
  \item подсистема управления услугами производит изменение условий предоставления сервиса или удаляет его;
  \item переход к сценарию 2, в случае продолжения оказания сервиса, иначе к сценарию 5.
\end{enumerate}

Сценарий 4. Окончание срока действия договора:
\begin{enumerate}
  \item у пользователя окончился срок действия договора;
  \item биллинг блокирует услуги;
  \item если пользователь желает продлить срок действия договора, переход к сценарию 6, иначе к сценарию 5.
\end{enumerate}

Сценарий 5. Окончание использования облачной услуги:
\begin{enumerate}
  \item пользователь желает завершить использование сервиса;
  \item биллинг регистрирует отмену использования сервиса;
  \item подсистема управления услугами удаляет пользовательскую услугу (выделенный сервер, контейнер, виртуальная машина) из списка используемых.
\end{enumerate}

Сценарий 6. Продление договора:
\begin{enumerate}
  \item пользователь обращается к инфраструктуре через веб-интерфейс с целью продления срока действия договора;
  \item биллинг регистрирует факт заключения нового договора;
  \item переход к сценарию 2.
\end{enumerate}

Сценарий 7. Возникновение проблем в работе услуги:
\begin{enumerate}
  \item пользователь обращается к биллингу через веб-интерфейс с целью предоставления заявки на устранение проблемы и информации о возникшей проблеме;
  \item биллинг регистрирует запрос и направляет его технической поддержке;
  \item техническая поддержка первого уровня устраняет проблему, если нет, то заявка передается в отдел системного администрирования;
  \item переход к сценарию 2.
\end{enumerate}

Алгоритмы взаимодействия пользователем с инфраструктурой представлены на чертеже \href{extra/drafts/SevGU_09.03.01.15.A2.pdf}{СевГУ 09.03.01.15.А2}.

\subsubsection{Алгоритм взаимодействия компонентов инфраструктуры} \label{infr-comp}

Сценарий 1. Мониторинг:
\begin{enumerate}
  \item Nagios мониторит состояние сервисов;
  \item Nagios обнаружил недоступность или сбой сервиса;
  \item отправка электронного и SMS-сообщения дежурному администратору;
  \item дежурный администратор исправляет проблему;
  \item переход к пункту а.
\end{enumerate}

Сценарий 2. Резервное копирование:
\begin{enumerate}
  \item запуск задания инкрементального резервного копирования;
  \item расчет свободного дискового пространства на удаленном хранилище для инкрементальной или полной копии;
  \item в случае нехватки дискового пространства на удаленном хранилище --- отправка электронного сообщения системному администратору, переход к пункту е;
  \item архивация и сжатие данных;
  \item передача данных по сети;
  \item конец резервного копирования.
\end{enumerate}

Сценарий 3. Распределение IP-адресов:
\begin{enumerate}
  \item добавлен новый физический сервер, контейнер, виртуальная машина или клиент хостинга заказал дополнительный адрес;
  \item проверка свободного IP-адреса;
  \item если свободного адреса нет, то уведомление о необходимости заказа новой подсети IP-адресов у провайдера, переход к пункту е;
  \item привязка нового IP-адреса к серверу;
  \item удаление привязанного адреса из списка свободных;
  \item конец.
\end{enumerate}

Сценарий 4. Добавление нового сервера:
\begin{enumerate}
  \item заказ услуги физического сервера, контейнера или виртуальной машины;
  \item выделение IP-адресов;
  \item подключение к системе мониторинга;
  \item трансфер или создание зоны на главном DNS-сервере;
  \item подключение к системе резервного копирования;
  \item конец.
\end{enumerate}

Сценарий 5. Добавление нового домена:
\begin{enumerate}
  \item трансфер или создание зоны на главном DNS-сервере;
  \item проверка доступности подчиненных DNS-серверов;
  \item репликация зоны на подчиненные сервера;
  \item ожидание обновления зон провайдера;
  \item проверка работоспособности DNS;
  \item конец.
\end{enumerate}

Алгоритм взаимодействия компонентов инфраструктуры представлен на чертеже \href{extra/drafts/SevGU_09.03.01.15.A3.pdf}{СевГУ 09.03.01.15.А3}.

\clearpage