\section{Аппаратное обеспечение}
\subsection{Системные требования}
В состав технических средств по обеспечению работы сервиса для качественного обслуживания одновременно до 500 пользователей должен входить минимум один виртуальный или физический сервер, обладающий следующими характеристиками:

\begin{itemize}
  \item 8 и более виртуальных или физических ядер, работающих с набором команд x86, x86-x64 или arm с частотой не менее 2ГГц;
  \item не менее 16 ГБ оперативной памяти;
  \item не менее 1 ТБ пространства на физических носителях, объединенных в избыточный массив независимых дисков не ниже первого уровня;
  \item выделенная линия с пропускной способностью не менее 1 Гбит/с с доступом в сеть интернет или целевую сеть сервиса;
  \item наличие резервного источника питания;
  \item резервный блок питания;
  \item наличие резервной линии с пропускной способностью не менее 400 Мбит/с с доступом в сеть интернет или целевую сеть сервиса;
  \item возможность замены вышедших из строя дисков без необходимости выключения сервера.
\end{itemize}

\subsection{Требования к программному обеспечению}
Для корректной возможности работы сервиса, на сервере должна быть установлена операционная система, поддерживающая работу виртуальной машины Java версии не ниже 1.8 и системы управления базами данных PostgreSQL. База данных может быть заменена на другую при условии дополнительного конфигурирования сервиса. 

\section{Конфигурация окружения}

\subsection{Установка и конфигурация Java окружения}

Способы установки JRE могут отличаться в зависимости от операционной системы. В операционных системах, работающих на ядре linux или BSD способ установки зависит от пакетного менеджера, установленного в системе по умолчанию.
В операционных системах с пакетным менеджером apt-get терминальная команда, устанавливающая JRE выглядит следующим образом:
\begin{lstlisting}
# sudo apt-get update
# sudo apt-get install default-jre
\end{lstlisting}

В операционных системах с пакетным менеджером pkg (например FreeBSD) необходимо выполнить данную команду:
\begin{lstlisting}
# sudo pkg install openjdk8-jre
\end{lstlisting}
После установки JRE необходимо установить переменные среды с помощью команды:
\begin{lstlisting}
# sudo apt-get install oracle-java8-set-default
\end{lstlisting}
или вручную:
\begin{lstlisting}
# export PATH="$PATH:$JAVA_HOME/bin:$JRE_HOME/bin"

# export JAVA_HOME=/usr/lib/jvm/java-8-openjdk-jre
# export JRE_HOME=/usr/lib/jvm/java-8-openjdk-jre
\end{lstlisting}

В операционных системах семейства windows для установки JRE необходимо скачать и запустить исполняемый файл с одного из сайтов с бесплатной реализацией java.
Например, Amazon Corretto с сайта amazon: https://docs.aws.amazon.com/en_us/corretto/latest/corretto-8-ug/downloads-list.html.
После запуска необходимо следовать инструкциям установщика (рисунок \ref{screenshot-corretto}).
\addimghere{screenshot-corretto}{0.8}{Окно установщика Amazon Corretto}{screenshot-corretto}

\subsection{Установка и конфигурация PostgreSQL}

Последовательность действий, необходимая для установки и конфигурации PostgreSQL в операционных системах семейства linux с пакетным менеджером apt-get следующая:
\begin{lstlisting}
# sudo apt-get update
# sudo apt-get install postgresql postgresql-contrib
\end{lstlisting}

Для установки PostgreSQL на операционные системы с пакетным менеджером pkg:
\begin{lstlisting}
# sudo pkg install postgresql11-server postgresql11-client
# sudo sysrc postgresql_enable=yes
# /usr/local/etc/rc.d/postgresql initdb
# /usr/local/etc/rc.d/postgresql start
\end{lstlisting}

Для того, чтобы разрешить подключение к базе данных с устройства, на котором запущен postgresql, необходимо открыть с правами привилегированного пользователя текстовым редактором файл /etc/postgresql/версия_postgresql/main/postgresql.conf и добавить в конец файла строку строку следующего содеражния:
\begin{lstlisting}
listen_addresses = 'localhost'
\end{lstlisting}
где localhost может быть ip адресом или адресом сети, устройствам которой разрешено подключение к базе данных.

После установки необходимо сконфигурировать учетные записи.
\begin{lstlisting}
# sudo -u postgres psql postgres
ALTER USER postgres with encrypted password '_ваш пароль';
\end{lstlisting}
Для разделенного доступа к базе данных и повышеной безопасности, необходимо создать пользователя с доступом только к необходимым базам данных и таблицам:
\begin{lstlisting}
CREATE DATABASE image_hosting;
CREATE USER hosting_service WITH password '_безопасный пароль';
GRANT ALL ON DATABASE image_hosting TO hosting_service;
\end{lstlisting}

После успешной конфигурации, требуется разрешить подключение данному пользователю с удаленного устройства или с устройства, на котором производится конфигурация в случае, если сам сервис планируется установить на тот же сервер, что и PostgreSQL. Для того, чтобы разрешить подключение к базе данных пользователю, необходимо в конец файла /etc/postgresql/версия_postgresql/main/pg_hba.conf добавить следующую строку:
\begin{lstlisting}
local   all         hosting_service                          md5
\end{lstlisting}
где local - адрес устройства или подсеть, с которой разрешена аутентификация пользователю hosting_service.

После всех манипуляций потребуется обновление конфигурации PostgreSQL:
\begin{lstlisting}
# sudo -u postgres psql postgres
SELECT pg_reload_conf()
\end{lstlisting}

В операционных системах семейства windows необходимо перейти на оффициальный сайт PostgreSQL, скачать установщик, следовать его инструкциям (рисунок \ref{postgresql-screenshot}), после чего запустить PgAdmin и сконфигурировать учетные записи в графическом интерфейсе.
\addimghere{postgresql-screenshot}{0.8}{Окно установщика PostgreSQL}{postgresql-screenshot}

\subsection{Установка и конфигурация программного решения}

После проведения всех операций по установке и настройке окружения, необходимо установить исполняемый jar файл сервиса на сервер.
Оптимальной будет директория /opt/hosting-service/

После чего необходимо добавить исполняемый файл в сервисы менеджера системы и сервисов.
Для менеджера systemctl в директории /lib/systemd/system/ необходимо создать свой файл myservice.service следующего содержания:

\begin{lstlisting}
[Unit]
Description=Tomcat Instance

[Service]
Type=simple
RemainAfterExit=yes
ExecStart=java -jar /opt/hosting-service/service.jar
User=hosting
Group=hosting

[Install]
WantedBy=multi-user.target
\end{lstlisting}

Для того, чтобы сервис запускался автоматически после загрузки сервера необходимо добавить сервис в автозагрузку systemctl:
\begin{lstlisting}
$ sudo systemctl enable myservice
\end{lstlisting}

Убрать сервис из автозагрузки в systemctl:
\begin{lstlisting}
$ sudo systemctl disable myservice
\end{lstlisting}

Узнать стоит ли сервис в автозагрузке:
\begin{lstlisting}
# systemctl is-enabled myservice
\end{lstlisting}

Для запуска/остановки/перезапуска сервиса служат следующие три комманды:
\begin{lstlisting}
# systemctl start myservice
# systemctl stop myservice
# systemctl restart myservice
\end{lstlisting}

Для конфигурации сервиса, в первую очередь необходимо поместить файл application.properties из поставки приложения в директорию с установленным приложением.
После чего сконфигурировать сервис.
Прежде всего требуется указать настройки подключения к базе данных и данные аутентификации почтового ящика для отправки электронных писем с ссылкой подтверждения регистрации. Описание всех настроек приложения приведено в таблице \ref{properties-table}

\begin{table}[H]
  \caption{Api контроллера аутентификации}\label{properties-table}
  \begin{tabular}{|p{9cm}|p{7cm}|}
  \hline Настройка & Описание \\
  \hline spring.servlet.multipart.enabled=true & Передача файлов по частям \\
  \hline spring.servlet.multipart.file-size-threshold=2KB & Размер части  принимаемого файла, после получения которой она будет записываться на диск \\
  \hline spring.servlet.multipart.max-file-size=40MB & Максимальный размер загружаемого файла \\
  \hline spring.jpa.hibernate.ddl-auto=update & Автоматическое приведение схемы базы данных в соответствие модели классов сервиса \\
  \hline spring.jpa.generate-ddl=true & Генерация запросов на созда \\
  \hline spring.jpa.database-platform= org.hibernate.dialect.PostgreSQL94Dialect & Используемый диалект языка SQL для обращения к базе данных \\
  \hline spring.datasource.driverClassName= org.postgresql.Driver & Используемый класс драйвера для обращения к базе данных \\
  \hline spring.datasource.url= jdbc:postgresql://localhost:5432/image_hosting & \multirow{2}{=}{Адрес соединения с базой данных} \\
  \cline{1-1} spring.datasource.jdbcUrl= jdbc:postgresql://localhost:5432/image_hosting & \\
  \hline spring.datasource.username=hosting_service & Имя пользователя базы данных\\
  \hline spring.datasource.password=password & Пароль пользователя \\
  \hline spring.jpa.show-sql=true & Логирование SQL запросов к базе данных \\
  \hline file.image-storage-dir=./storage/ & Корневая директория, в которую будут сохраняться фотографии \\
  \end{tabular}
\end{table}
\begin{table}[H]
  \caption*{Окончание таблицы \ref{properties-table}}
  \begin{tabular}{|p{9cm}|p{7cm}|}
  \hline Настройка & Описание \\
  \hline file.max-number-of-files-in-dir=10000 & Максимальное количество фотографий в поддиректории \\
  \hline file.size-of-path-tree=1 & Глубина вложенности директорий хранилища фотографий \\
  \hline file.dir-name-length=2 & Длинна названия директорий с файлами \\
  \hline color.high-value=256 & Верхняя граница значения цвета по одному каналу \\
  \hline color.low-value=0 & Нижняя граница значения цвета по одному каналу \\
  \hline support.email=support@gmail.com & Электронная почта, с которой отправляется письмо для подтверждения регистрации \\
  \hline spring.mail.host=smtp.gmail.com & Адрес почтового сервера \\
  \hline spring.mail.port=465 & Порт почтового сервера \\
  \hline spring.mail.protocol=smtps & Протокол почтовой службы \\
  \hline spring.mail.username=support@gmail.com & Имя пользователя почтового ящика для аутентификации \\
  \hline spring.mail.password=qwertyPassword123 & Пароль пользователя почтового ящика для аутентификации \\
  \hline spring.mail.properties.mail.transport.protocol=smtps & Протокол почтовой службы \\
  \hline spring.mail.properties.mail.smtps.auth=true & Использование аутентификации \\
  \hline spring.mail.properties.mail.smtps.timeout=8000 & Таймаут аутентификации и отправки писем \\
  \hline logging.level.org.hibernate.SQL=DEBUG & Уровень логирования SQL запросов. Возможные значения: DEBUG, INFO, WARN \\
  \hline 
  \end{tabular}
\end{table}

\subsection{Защита от DDoS-атак}

При обнаружении подозрительного трафика следует убедиться, что это действительно DDoS атака, а также проследить за реакцией сетевого оборудования на нее.
Определить DDoS это или нет можно с помощью утилит netstat и iftop.

Намеренную DDoS-атаку остановить с помощью программных средств практически невозможно, однако некоторые программные решения позволяют отражать небольшие атаки.
Одним из таких программных решений является DDoS-Deflate.
Он анализирует подключения к серверу на основании данных программного решения netstat, входящего в поставку большинства дистрибутивов linux и в случае превышения лимита подключений с одного адреса, запросы с него блокируются на некоторое время.

Для блокировки намеренных атак требуется поддержка аппаратной защиты от атак сетевым оборудованием.
Защита от распределенных DDoS-атак основывается на мно­го­фак­тор­ном анализе трафика, который поступает на каждый защищаемый сервер.
Во время нормальной работы система защиты может самообучаться или настраиваться, а после обнаружения атаки либо автоматически, либо по требованию, активно про­тиво­дей­ству­ет нелегитимному трафику.

В данном случае защитником выступает дата-центр, который имеет в распоряжении оборудование, способное осуществлять фильтрацию трафика и предотвращение атак на отказ.

\subsection{Общие рекомендации по администрированию}

Следует следить за новостными рассылками о критических уязвимостях в операционной системе и используемом программном обеспечении.

Если работа сервера замедлилась, в диагностике проблемы помогут утилиты ps, top, atop, htop.
В случае обнаружения сетевых проблем на помощью придут утилиты ping, traceroute, nmap, mtr, tcpdump, iftop.

Важно следить за состоянием RAID на сервере и в случае сбоя одного из дисков, следует произвести замену диска.

Также необходимо подбирать сложные для перебора пароли, периодически их менять и уведомлять пользователей о смене пароля.
Для генерации паролей подходит утилита pwgen:
\begin{lstlisting}
# pwgen 12 -s1yc
Y;Xhfu;04A{O
\end{lstlisting}

Стоит аккуратно работать на сервере под учетной записью root, не стоит запускать неизвестные скрипты или двоичные файлы.
Необходимо проверять контрольные суммы загруженных пакетов и следить за цифровой подписью.

Не следует просто так перезагружать сервер, рекомендуется это делать только в случаях чрезвычайной необходимости (таких как обновление ядра).

Всегда стоит проверять наличие актуальных резервных копий, а также свободное место на системах хранения данных.
Стоит также настроить ротацию логов, это позволит сэкономить место на диске.

Всегда стоит вести документацию в том или ином виде и логировать свои действия на сервере.
Логи, которые могут помочь в диагностике неисправностей:
\begin{itemize}
  \item /var/log/auth.log;
  \item /var/log/dmesg;
  \item /var/log/kern.log;
  \item /var/log/messages;
  \item /var/log/nginx/error.log;
  \item /var/log/syslog;
  \item /var/log/vzctl.log.
\end{itemize}

В случае обнаружения неполадок с аппаратной частью, стоит незамедлительно обратиться к провайдеру для проверки работоспособности аппаратуры.
В случае недочетов в программном обеспечении стоит обратиться к технической поддержке ПО.

\clearpage