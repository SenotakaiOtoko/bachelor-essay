\section{Руководство администратора}

Руководство администратора содержит информацию о сетевой инфраструктуре, используемых технологиях виртуализации и программном обеспечении.

\subsection{Расположение серверов в сетевой инфраструктуре}
Во всей инфраструктуре на физических серверах первого дата-центра располагаются:
\begin{itemize}
  \item shared-хостинг (10.0.0.100);
  \item виртуализация OpenVZ (10.0.1.100);
  \item виртуализация KVM (10.0.2.100);
  \item сервер резервного копирования (10.0.3.100);
  \item физические сервера клиентов (10.0.4.100 --- 10.0.4.200).
\end{itemize}

В свою очередь, некоторые важные сервисы инфраструктуры располагаются на виртуальных машинах:
\begin{itemize}
  \item сервер биллинга (10.0.1.101), OpenVZ;
  \item сервер управления IP-адресами (10.0.2.101), KVM;
  \item сервер DNS 1 (10.0.2.102), KVM.
\end{itemize}

Сервер мониторинга и сервер DNS 2 располагаются на виртуальных машинах KVM в другом дата-центре.
Сервер shared-хостинга выполняет роль главного (master) DNS-сервера и управляется с помощью PowerDNS.

\subsection{Сервер shared-хостинга}

Сервер shared-хостинга является самым уязвимым местом всей инфраструктуры из-за большого количества и плотности клиентов на сервере.
Распределение ресурсов между пользователями происходит за счет встроенных в ядро Linux механизмов.

Стоит отметить, что после переезда на новый сервер для хранения данных доступен RAID-массив из SSD дисков объемом 1 Тб.

На сервере установлен и настроен веб-сервер в составе следующего ПО:
\begin{itemize}
  \item http-сервер Apache HTTP Server, кэширующий и проксирующий http-сервер nginx и обработчик динамических PHP-запросов php-fpm;
  \item memcached кэширует динамические запросы в оперативной памяти;
  \item MySQL и PostgreSQL выполняют роли серверов баз данных.
\end{itemize}

nginx принимает запросы к серверу с порта 80 и решает каким образом обрабатывать запрос.
Если это запрос на получение статического файла (изображение, js-код, CSS, видео, аудио, архивы), то nginx сам его обрабатывает и в случае необходимости кэширует.
Если же запрашивается динамическое содержимое, то nginx передает запрос на бэкенд, где его, в зависимости от настройки виртуального хоста сайта, обрабатывает либо php-fpm, либо встроенный в Apache модуль mod\_php.
nginx не может кэшировать динамические запросы, поэтому для этого используется memcached, который работает в связке с Apache и позволяет кэшировать динамические запросы.
Для работы Apache с memcached необходима установка модуля PHP php5-memcached:
\begin{lstlisting}
# apt-get install php5-memcached
\end{lstlisting}

Почтовой связкой на сервере shared-хостинга выступает SMTP-сервер Exim Internet Mailer, а запросы по протоколам POP3 и IMAP принимает Dovecot.
В качестве антиспам-связки выступает Postgrey, Spamassasin, ClamAV и OpenDKIM.
Конфигурации данных сервисов являются стандартными.

ProFTPd необходим для работы пользователей по протоколу FTP.
Доступна возможность подключения пользователей по протоколу FTPS (порт 22).

Сервер управления временем NTP обращается к следующим серверам за уточнением времени на сервере:
\begin{lstlisting}
# cat /etc/ntp.conf | grep ^server | awk '{print $2}'
0.debian.pool.ntp.org
1.debian.pool.ntp.org
2.debian.pool.ntp.org
3.debian.pool.ntp.org
\end{lstlisting}
Конфигурация SSH требует отдельного пояснения.
Доступ к SSH по стандартному порту закрыт, это позволяет избавиться от большинства злоумышленников, которые пытаются скомпрометировать пароль доступа к серверу.
Также запрещен вход от пользователя root, для этого создан отдельный пользователь.
Можно ограничить доступ к серверу по SSH, оставив только доверенные адреса или подсети адресов.
\begin{lstlisting}
# useradd sshuser
# passwd sshuser
# cat /etc/ssh/sshd_config
Port 222
PermitRootLogin yes
Match host 10.0.0.111, 10.0.0.222
  PermitRootLogin yes
# service ssh restart
\end{lstlisting}
Подключение к серверу по SSH и получение root-прав:
\begin{lstlisting}
user@10.0.0.111~$ ssh -p 222 sshuser@10.0.0.100
sshuser@10.0.0.100's password:
sshuser@10.0.0.100~$ su -
Password:
root@10.0.0.100~#
\end{lstlisting}
Разрешен вход на сервер по ключам:
\begin{lstlisting}
user@10.0.0.111~$ ssh-keygen
user@10.0.0.111~$ ssh-keygen -p
user@10.0.0.111~$ ssh-copy-id -p 222 root@10.0.0.100
user@10.0.0.111~$ ssh -p 222 root@10.0.0.100
\end{lstlisting}
Резервное копирование данных пользователей осуществляется средствами панели ISPmanager 5 в ночное время, применяется инкрементальный метод резервного копирования.
В инкрементальном методе копирования при первом резервном копировании будет создана полная копия, в которой будут сохранены все файлы.
При последующих будут сохраняться файлы, измененные с момента предыдущей полной или инкрементальной копии.
Для восстановления необходимо несколько копий --- полная и все инкрементальные, сохраненные до выбранной точки восстановления.
Раз в неделю (в воскресенье) запускается очередная полная резервная копия данных.
На сервере функционирует скрипт блокировки адресов, которые слишком часто подбирают пароли доступа к панели администратора наиболее популярных CMS (система управления содержимым), таких как WordPress и Joomla.
Скрипт написан на языке Shell, находится в открытом доступе и имеет простое использование:
\begin{lstlisting}
#!/bin/bash
# Nginx block ip more than 3k connections to WordPress \
and Joomla admin panels
# admin@amet13.name
# v0.1 29.06.2014
# v0.2 30.06.2014
# IP's > 3000 connections to WordPress and Joomla admin panels
command=$(cat /var/log/nginx/access.log | \
grep "administrator/index.php \|wp-login.php " | \
cut -f1 -d " " | \
sort | \
uniq -c | \
sort -n | \
awk '{ if ($1 > 2999) print $2}')

# Add new ip's to database
for word in $command; do
   # Is that ip already in database?
   grep $word /etc/nginx/blockips.conf > /dev/null
   if (( $? ));
   then 
      sed -i "s/allow all;/deny $word;\nallow all;/" \
      /etc/nginx/blockips.conf
   fi
done
exit 0
\end{lstlisting}

Основные конфигурации настроек веб-сервера хранятся в файлах:
\begin{itemize}
  \item /etc/nginx/nginx.conf;
  \item /etc/apache2/apache2.conf;
  \item /etc/php5/\{apache2,cgi,cli,fpm\}/php.ini;
  \item /etc/mysql/my.cnf;
  \item /etc/postgresql/9.1/main/postgresql.conf.
\end{itemize}

Для сканирования уязвимостей на сайтах пользователей используется утилита maldet, которая находит подозрительные файлы в системе и составляет отчет по найденным файлам.

Установка и пример использования maldet:
\begin{lstlisting}
# cd /tmp/
# wget http://www.rfxn.com/downloads/maldetect-current.tar.gz
# tar xfz maldetect-current.tar.gz
# cd maldetect-*
# ./install.sh

# maldet -b --scan-all /var/www/
# maldet --report list
# maldet --report 021715-1414.3266
# maldet -q 021715-1414.3266
\end{lstlisting}

На сервере shared-хостинга на сайтах пользователей находится большое количество уязвимостей, которые часто используются для рассылки спама с сервера.
Для обнаружения большого числа рассылок с сервера был написан плагин для системы мониторинга Nagios, который контролирует число писем в почтовой очереди, а также проверяет наиболее популярные организации (например Spamhaus), которые собирают данные о спам-серверах.

Для просмотра списка очереди используется команда exim -bp или mailq.
Таким образом можно просмотреть количество писем для разных доменов и их количество:
\begin{lstlisting}
# exim -bp | exiqsumm -c -s | head -10
Count  Volume  Oldest  Newest  Domain
-----  ------  ------  ------  ------
12151  1378KB     14h      0m  spamsite.ru > host.ru
   93   264KB     14h      5m  spamsite.ru > gmail.com
   16   154KB     13h     11h  site.ru > kk.com
    8    77KB     13h     11h  site.ru > ww.com
\end{lstlisting}

В данном случае очевидна рассылка писем с сайта spamsite.ru.
Посмотреть идентификатор письма (ID), а также содержимое заголовков и тела письма можно теми же командами, но с другими параметрами:
\begin{lstlisting}
# exiqgrep -b -f 'spamsite.ru' | head -1
1YyRgs-0000zD-Qy From: <fobos@spamsite.ru> To: maria@host.ru
# exim -Mvh 1YyRgs-0000zD-Qy
# exim -Mvb 1YyRgs-0000zD-Qy
\end{lstlisting}

Если сайт действительно заражен, то следует отключить его, оповестить пользователя о закрытии сайта и необходимости избавления от вредоносного кода.
Также необходимо удалить спам-письма из очереди сообщений:
\begin{lstlisting}
# exiqgrep -b -f 'spamsite.ru' | \
awk '{print $1}' | xargs exim -Mrm
\end{lstlisting}
В случае неработоспособности панели ISPmanager 5 возможна ее перезагрузка:
\begin{lstlisting}
# pgrep -l core
4437 core
7129 core
21194 core
# killall core
\end{lstlisting}
\subsection{Сервер виртуализации OpenVZ}
Список всех существующих в системе контейнеров можно узнать командой vzlist -a:
\begin{lstlisting}
# vzlist -a
      CTID      NPROC STATUS    IP_ADDR      HOSTNAME
       145        134 running   10.0.0.245   site1.ru
       146         41 running   10.0.0.246   site2.ru
       205          - stopped   10.0.0.205   site3.ru
\end{lstlisting}
В заголовках показывается показывается ID контейнера, число процессов в контейнере, текущий статус, IP-адрес и имя контейнера.
Список всех доступных шаблонов операционной системы и список шаблонов установленных локально:
\begin{lstlisting}
# vztmpl-dl --list-all
centos-5-x86
centos-5-x86_64
centos-5-x86_64-devel
centos-5-x86-devel
centos-6-x86
centos-6-x86_64
centos-6-x86_64-devel
# vztmpl-dl --list-local
centos-7-x86_64
debian-7.0-x86_64
ubuntu-14.04-x86_64
\end{lstlisting}
Каждый контейнер имеет свой конфигурационный файл, которые хранятся в каталоге /etc/sysconfig/vz-scripts/, именуются эти файлы по CTID контейнера.
Например, для контейнера с CTID=101, файл будет называться 101.conf.
При создании контейнера можно использовать типовую конфигурацию для VPS.
Типовые файлы конфигураций находятся в том же каталоге:
\begin{lstlisting}
# ls -1 /etc/sysconfig/vz-scripts/ | grep sample
ve-basic.conf-sample
ve-ispbasic.conf-sample
ve-light.conf-sample
ve-vswap-1024m.conf-sample
ve-vswap-1g.conf-sample
ve-vswap-256m.conf-sample
ve-vswap-2g.conf-sample
ve-vswap-4g.conf-sample
ve-vswap-512m.conf-sample
\end{lstlisting}
В этих конфигурационных файлах описаны контрольные параметры ресурсов, выделенное дисковое пространство, оперативная память и прочие ресурсы.
На базе уже существующего файла конфигурации создан типовой файл, подходящий для большинства контейнеров именуемый ve-custom.conf-sample.
Таким образом, при использовании этого конфигурационного файла, будет создаваться контейнер, которому будет доступен 1 Гб выделенного дискового пространства, 128 Мб оперативной памяти и 128 Мб SWAP.
В дальнейшем, при создании контейнеров, следует использовать данный конфигурационный файл.
Создание контейнера осуществляется командой:
\begin{lstlisting}
# vzctl create 101 --ostemplate debian-7.0-x86_64 --config custom
\end{lstlisting}
где 101 --- CTID контейнера, -{}-ostemplate --- шаблон ОС, -{}-config --- используемый шаблон конфигурационного файла.
Перед первым запуском контейнера необходимо установить его IP адрес, hostname, указать DNS сервера и задать пароль суперпользователя.
Для настройки VPS используется команда vzctl set.
Для того, чтобы контейнер запускался при старте хост-компьютера (например после перезагрузки), необходимо использовать команду:
\begin{lstlisting}
# vzctl set 101 --onboot yes --save
\end{lstlisting}
При использовании ключа -{}-save, сохраняются параметры контейнера в соответствующий ему конфигурационный файл.
Аналогично задается hostname и IP-адрес:
\begin{lstlisting}
# vzctl set 101 --hostname site.ru --save
# vzctl set 101 --ipadd 10.0.0.210 --save
\end{lstlisting}
Адреса DNS серверов (в большинстве случаев адрес DNS совпадает с адресом хост-компьютера, поэтому можно вместо адреса указать параметр inherit):
\begin{lstlisting}
# vzctl set 101 --nameserver inherit --save
\end{lstlisting}
Установка пароля суперпользователя:
\begin{lstlisting}
# vzctl set 101 --userpasswd root:p@ssw0rd
\end{lstlisting}
Пароль будет установлен в VPS, в файл /etc/shadow и не будет сохранен в конфигурационный файл контейнера.
Если же пароль будет утерян или забыт, то можно будет просто задать новый.
После настроек нового контейнера, его можно запустить:
\begin{lstlisting}
# vzctl start 101
\end{lstlisting}
Проверка сетевых интерфейсов внутри гостевой ОС:
\begin{lstlisting}
# vzctl exec 101 ifconfig | grep "lo\|venet" -A 1
lo        Link encap:Local Loopback
          inet addr:127.0.0.1  Mask:255.0.0.0
--
venet0    Link encap:UNSPEC  HWaddr 00-00-00-00-00-00
          inet addr:127.0.0.2  P-t-P:127.0.0.2  \
          Bcast:0.0.0.0  Mask:255.255.255.255
--
venet0:0  Link encap:UNSPEC  HWaddr 00-00-00-00-00-00
          inet addr:10.0.0.210  P-t-P:10.0.0.210  \
          Bcast:10.0.0.201  Mask:255.255.255.255
\end{lstlisting}
Должны присутствовать сетевые интерфейсы:
\begin{itemize}
    \item lo (127.0.0.1);
    \item venet0 (127.0.0.2);
    \item venet0:0 (10.0.0.210).
\end{itemize}
Если сеть в порядке, то можно соединиться к контейнеру по SSH с хост-компьютера:
\begin{lstlisting}
# ssh root@10.0.0.210
root@10.0.0.210's password: p@ssw0rd
\end{lstlisting}
Вход в контейнер напрямую с хост-компьютера осуществляется командой vzctl enter:
\begin{lstlisting}
# vzctl enter 101
entered into CT 101
root@site:/#
\end{lstlisting}
Для остановки контейнера используется команда vzctl stop.
Для полной остановки контейнера, системе требуется немного времени.
Иногда нужно выключить VPS как можно быстрее, например, если контейнер был подвержен взлому.
Для того чтобы срочно выключить VPS, используется ключ -{}-fast:
\begin{lstlisting}
# vzctl stop 101 --fast
\end{lstlisting}
Для перезапуска контейнера можно использовать команду vzctl restart.
Для того чтобы удалить контейнер, его нужно сначала остановить:
\begin{lstlisting}
# vzctl stop 101
# vzctl destroy 101
\end{lstlisting}
Команда выполняет удаление частной области сервера и переименовывает файл конфигурации, дописывая к нему .destroyed.
Иногда бывает нужно выполнить команду на нескольких VPS.
Для этого можно использовать команду:
\begin{lstlisting}
# for i in `vzlist -o veid -H`; do \
echo "VPS $i"; vzctl exec $i command; done
\end{lstlisting}
Например можно узнать сколько времени работают все запущенные контейнеры:
\begin{lstlisting}
# for i in `vzlist -o veid -H`; do \
echo "VPS $i"; vzctl exec $i uptime; done
VPS 101
 05:45:01 up 2 min, 0 users, load average: 0.01, 0.02, 0.03
VPS 102
 05:46:01 up 1 min, 0 users, load average: 0.04, 0.05, 0.06
\end{lstlisting}
Если на хост-ноде наблюдается высокое значение Load Average, то в первую очередь следует посмотреть данные значения непосредственно у контейнеров:
\begin{lstlisting}
# vzlist -o veid,laverage
      CTID       LAVERAGE
       145 0.00/0.01/0.05
       146 0.00/0.00/0.00
       148 0.60/0.51/0.34
\end{lstlisting}
В случае обнаружения нагружающего контейнера, стоит подключиться к нему и устранить причину нагрузки.
Непосредственно с хост-ноды можно узнать, какому контейнеру принадлежит процесс:
\begin{lstlisting}
# vzpid 1048108
Pid	    CTID	Name
1048108	145	  php-cgi
\end{lstlisting}
Свободное место в контейнере можно узнать командой df:
\begin{lstlisting}
# df -h
Filesystem            Size  Used Avail Use% Mounted on
/dev/mapper/lv_root    50G  5,6G   42G  12% /
tmpfs                  16G     0   16G   0% /dev/shm
/dev/cciss/c0d0p1     485M  276M  184M  61% /boot
/dev/mapper/lv_vz     489G  403G   62G  87% /vz
# df -i
Filesystem             Inodes IUsed    IFree IUse% Mounted on
/dev/mapper/lv_root   3276800 55229  3221571    2% /
tmpfs                 4101290     1  4101289    1% /dev/shm
/dev/cciss/c0d0p1      128016    96   127920    1% /boot
/dev/mapper/lv_vz    32546816   248 32546568    1% /vz
\end{lstlisting}
Подключение модуля TUN для контейнера (необходимо для работы OpenVPN).
Прежде чем запускать контейнер нужно убедиться, что модуль TUN загружен на хост-ноде:
\begin{lstlisting}
# lsmod | grep tun
\end{lstlisting}
В случае, если модуль не загружен:
\begin{lstlisting}
# modprobe tun
# lsmod | grep tun
tun     1957    0
\end{lstlisting}
Разрешаем использовать устройство TUN контейнеру:
\begin{lstlisting}
# vzctl set 101 --devnodes net/tun:rw --save
# vzctl set 101 --devices c:10:200:rw --save
# vzctl set 101 --capability net_admin:on --save
\end{lstlisting}
Запуск контейнера:
\begin{lstlisting}
# vzctl start 101
\end{lstlisting}
Создание в контейнере собственного устройства TUN:
\begin{lstlisting}
# vzctl exec 101 mkdir -p /dev/net
# vzctl exec 101 mknod /dev/net/tun c 10 200
# vzctl exec 101 chmod 600 /dev/net/tun
\end{lstlisting}
Для пользователей в контейнерах доступен брандмауэр Netfilter, для его работы в файле /etc/vz/vz.conf должна присутствовать строка:
\begin{lstlisting}
IPTABLES="ipt_owner ipt_REDIRECT ipt_recent ip_tables \
iptable_filter iptable_mangle ipt_limit ipt_multiport \
ipt_tos ipt_TOS ipt_REJECT ipt_TCPMSS ipt_tcpmss \
ipt_ttl ipt_LOG ipt_length ip_conntrack ip_conntrack_ftp \
ipt_state iptable_nat ip_nat_ftp"
\end{lstlisting}
Для того, чтобы для контейнеров был доступен FUSE, его необходимо включить на хост-ноде и проверить, что он успешно включен:
\begin{lstlisting}
# modprobe fuse
# lsmod | grep fuse
fuse  92980  54
\end{lstlisting}
Также необходимо добавить автозагрузку модуля при перезапуске хост-ноды:
\begin{lstlisting}
# vim /etc/rc.local
#!/bin/sh
modprobe fuse
\end{lstlisting}
Проброс FUSE для контейнера 101:
\begin{lstlisting}
# vzctl stop 101
# vzctl set 101 --devnodes fuse:rw --save
# vzctl set 101 --devices c:10:229:rw --save
# vzctl start 101
\end{lstlisting}
В контейнере проверяем, пробросилось ли устройство:
\begin{lstlisting}
[root@site /]# ls /dev/fuse
/dev/fuse
\end{lstlisting}
Резервное копирование контейнеров осуществляется утилитами vzdump и vzrestore:
\begin{lstlisting}
# vzdump --suspend --compress --dumpdir /root/ \
--exclude-path /tmp/ 101
# vzrestore \
/vz/dump/vzdump-openvz-101-2015_01_01-12_12_12.tar 103
\end{lstlisting}
Для работы OpenVZ с файлами используется ploop, так как simfs является устаревшим методом.
Диски с файловыми системами контейнеров хранятся в /vz/private и имеют имя root.hdd.
Смена размера диска для контейнера:
\begin{lstlisting}
# vzctl set 103 --diskspace 2G --save
\end{lstlisting}
\subsection{Сервер виртуализации KVM}
В целом, администрирование KVM-ноды не отличается от администрирования OpenVZ.
Однако в случае OpenVZ возможно получить доступ к пользовательскому контейнеру непосредственно с хост-ноды, в случае с KVM это невозможно, это следует учесть при работе с виртуальной машиной.
Изменение размера диска также является нетривиальной задачей, высок риск краха всей файловой системы без возможности восстановления данных.
Команды, которые чаще всего используются при администрировании виртуальных машин на базе KVM:
\begin{lstlisting}
# virsh list --all
# virsh start lin_templ
# virsh shutdown lin_templ
# virsh destroy lin_templ
# virsh undefine lin_templ
# virsh autostart lin_templ
# virsh autostart --disable lin_templ
# virsh define /etc/libvirt/qemu/lin_templ.xml
# virsh vncdisplay lin_templ
# virsh dumpxml lin_templ | grep vnc
\end{lstlisting}
Для защиты сервера от брутфорса пароля SSH на серверах OpenVZ и KVM используется fail2ban.
Установка (в CentOS 6, для установки fail2ban нужно подключить репозиторий EPEL):
\begin{lstlisting}
# rpm -ivh http://download.fedoraproject.org/pub/epel/6/$(arch)/epel-release-6-8.noarch.rpm
# yum install fail2ban
\end{lstlisting}

fail2ban блокирует с помощью IPTables на некоторое время активные адреса, которые пытаются соединиться по SSH после пяти неудачных попыток, также отправляет администратору письмо с уведомлением о блокировке:
\begin{lstlisting}
# cat /etc/fail2ban/jail.local
[ssh-iptables]
enabled = true
filter = sshd
action = iptables[name=SSH, port=ssh, protocol=tcp]
sendmail-whois[name=SSH, dest=admin@domain.com, \
sender=fail2ban@example.com, sendername="Fail2Ban"]
maxretry = 5
bantime = 86400
\end{lstlisting}

Разблокировать IP адрес из бана:
\begin{lstlisting}
# fail2ban-client set JAIL unbanip IP
\end{lstlisting}

Пример разблокировки:
\begin{lstlisting}
# fail2ban-client set ssh-iptables unbanip 10.0.0.111
\end{lstlisting}

В виртуальных машинах пользователи зачастую склонны забывать пароль от MySQL.
В данном случае существует простое решение проблемы.
Остановка службы MySQL:
\begin{lstlisting}
# service mysql stop
\end{lstlisting}

Запуск службы с опцией -{}-skip-grant-tables:
\begin{lstlisting}
# mysqld_safe --skip-grant-tables &
\end{lstlisting}

Подключение с серверу MySQL при помощи клиента:
\begin{lstlisting}
# mysql -u root
mysql>
\end{lstlisting}

Ввод нового пароля для root:
\begin{lstlisting}
mysql> use mysql;
mysql> update user set password=PASSWORD("NEW-PASS") \
where User='root';
mysql> flush privileges;
mysql> quit
\end{lstlisting}

Остановка сервера MySQL:
\begin{lstlisting}
# service mysql stop
\end{lstlisting}

Запуск MySQL-сервера и вход с новым паролем:
\begin{lstlisting}
# service mysql start
# mysql -u root -p
Enter password: NEW-PASS
\end{lstlisting}

\subsection{Дополнительные виртуальные сервера}

В качестве сервера мониторинга используется Nagios, для построения графиков --- Munin.
Дополнительной настройки этих сервисов не требуется.
Добавление новых хостов происходит вручную, требуется лишь править адреса и сервисы, которые требуется мониторить.

Важно мониторить свободное место на жестком диске сервера резервного копирования и вовремя очищать старые резервные копии.

На DNS серверах используется master-slave репликация, причем один из DNS-серверов, равно как и сервер мониторинга расположены в другом дата-центре для обеспечения отказоустойчивости системы.
Также между подчиненными серверами настроена балансировка нагрузки.

\subsection{Защита от DDoS-атак}

Стоит незамедлительно реагировать на DDoS-атаки.
При обнаружении подозрительного трафика следует убедиться, что это действительно DDoS, а также поинтересоваться у дата-центра реакцию сетевого оборудования на нее.
Определить DDoS это или нет можно с помощью утилит netstat и iftop.

Намеренную DDoS-атаку сложно остановить лишь с помощью программных средств, однако некоторые несложные скрипты позволяют отражать небольшие <<любительские>> атаки на отказ.
Подобный скрипт блокировки DDoS-атак представлен в приложении А.
Скрипт анализирует текущие подключения к серверу на основе утилиты netstat, в случае превышения определенного лимита подключений, адрес попадает в блок IPTables на некоторое время.

Для блокировки намеренных атак требуется аппаратная защита на уровне сетевого оборудования.
Защита от распределенных DDoS-атак основывается на мно­го­фак­тор­ном анализе трафика, который поступает на каждый защищаемый сервер.
Во время нормальной работы система защиты может самообучаться или настраиваться, а после обнаружения атаки либо автоматически, либо по требованию, активно про­тиво­дей­ству­ет нелегитимному трафику.

В данном случае защитником выступает дата-центр, который имеет в распоряжении оборудование, способное осуществлять фильтрацию трафика и предотвращение атак на отказ.

\subsection{Общие рекомендации по администрированию инфраструктуры}

Следует следить за новостными рассылками о критических уязвимостях в операционной системе и используемом программном обеспечении.

Если работа сервера замедлилась, в диагностике проблемы помогут утилиты ps, top, atop, htop.
В случае обнаружения сетевых проблем на помощью придут утилиты ping, traceroute, nmap, mtr, tcpdump, iftop.

Важно следить за состоянием RAID на сервере, в случае сбоя одного из дисков, следует обратиться к поддержке дата-центра с просьбой о замене диска.

Необходимо подбирать сложные для перебора пароли, периодически их менять и уведомлять пользователей о смене пароля.
Для генерации паролей подходит утилита pwgen:
\begin{lstlisting}
# pwgen 12 -s1yc
Y;Xhfu;04A{O
\end{lstlisting}

Стоит аккуратно работать на сервере под учетной записью root, не стоит запускать неизвестные скрипты или двоичные файлы.
Необходимо проверять контрольные суммы загруженных пакетов и следить за цифровой подписью.

Не следует просто так перезагружать сервер, рекомендуется это делать только в случаях чрезвычайной необходимости (например обновление ядра).

Всегда стоит проверять наличие актуальных резервных копий, а также свободное место на системах хранения данных.
Стоит также настроить ротацию логов, это позволит сэкономить место на диске.

На физических нодах стоит устанавливать только самый необходимый, минимальный набор программного обеспечения, для уменьшения вероятности компрометации сервера.

Всегда стоит вести документацию в том или ином виде и логировать свои действия на сервере.
Логи, которые могут помочь в диагностике неисправностей:
\begin{itemize}
  \item /var/log/\{apache2,httpd\}/error.log;
  \item /var/log/auth.log;
  \item /var/log/dmesg;
  \item /var/log/kern.log;
  \item /var/log/libvirt/libvirtd.log;
  \item /var/log/mail.log;
  \item /var/log/messages;
  \item /var/log/mysql/mysql-slow.log;
  \item /var/log/nginx/error.log;
  \item /var/log/syslog;
  \item /var/log/vzctl.log.
\end{itemize}

В случае обнаружения неполадок с аппаратной частью, стоит незамедлительно обратиться к провайдеру для проверки работоспособности аппаратуры.
В случае недочетов в программном обеспечении стоит обратиться к технической поддержке ПО.

\clearpage