\section{Назначение и условия применения программы}
\subsection{Назначение системы}
Проектируемое приложение предназначено для обеспечения надёжного хранения фотографий пользователей на удаленном сервере с возможностью индексирования фотографий и осуществления поиска фотографий по хранилищу.
Приложение позволяет загружать и просматривать фотографии, помечать фотографии тегами в автоматическом или полуавтоматическом режиме, комментировать и оценивать фотографии, производить поиск фотографий по ключевым словам или по преобладающим цветам на фотографии.

\subsection{Классы и характеристики пользователей}

Перечень классов пользователей системы и их краткая характеристика представлены в таблице \ref{user-classes-table}.
\begin{table}[H]
  \caption{\onehalfspacing Классы и характеристики пользователей системы хранения фотографий}\label{user-classes-table}
  \begin{tabular}{|p{4cm}|p{12cm}|}
  \hline Класс пользователей & Описание \\ 
  \hline Посетитель & Посетитель – человек, просматривающий фотографии, загруженные в сервис и доступные для всех, но не являющийся зарегистрированным пользователем \\ 
  \hline Пользователь & Человек, прошедший регистрацию и подтвердивший себя посредством СМС или email. Отличается от посетителя большим набором возможностей, таким как публикация фотографий \\ 
  \hline
  \end{tabular}
\end{table}

\subsection{Варианты использования}
Перечень вариантов использования системы приведена в таблице \ref{use-case-table}.
Диаграмма вариантов использования на рисунке \ref{use-case} показывает варианты использования системы и связанные с ними действующие лица.


\begin{table}[H]
  \caption{\onehalfspacing Варианты использования системы хранения фотографий}\label{use-case-table}
  \begin{tabular}{|p{6cm}|p{10cm}|}
  \hline Основное действующее лицо & Вариант использования \\
  \hline \multirow{6}{*}{Посетитель} & 1. Пройти регистрацию \\
  \cline{2-2} & 2. Просмотреть ленту популярных фотографий \\
  \cline{2-2} & 3. Просмотреть фотографии пользователя \\
  \cline{2-2} & 4. Просмотреть информацию о фотографии \\
  \cline{2-2} & 5. Осуществить поиск фотографии по необходимым критериям \\
  \cline{2-2} & 6. Авторизоваться \\
  \hline \multirow{10}{*}{Пользователь} & 1. Привязать популярные соц сети \\
  \cline{2-2} & 2. Создать пост в соц сетях \\
  \cline{2-2} & 3. Настроить приватность фотографии \\
  \cline{2-2} & 4. Настроить защиту от копирования \\
  \cline{2-2} & 5. Оценить фотографию \\
  \cline{2-2} & 6. Прокомментировать фотографию \\
  \cline{2-2} & 7. Просмотреть статистику \\
  \cline{2-2} & 8. Сохранить в избранное \\
  \cline{2-2} & 9. Подписаться на публикации других пользователей \\
  \cline{2-2} & 10. Опубликовать фотографию \\
  \hline
  \end{tabular}
\end{table}

\begin{landscape}
    \addimg{use-case}{1.0}{Диаграмма вариантов использования программного решения для хранения фотографий}{use-case}
\end{landscape}

\section{Описание структуры программного комплекса}
\subsection{Диаграмма классов}
На рисунке \ref{program-structure} представлена диаграмма классов разбитая по пакетам (утилиты, аутентификация, модель базы данных, сервисы, веб, валидация, свойства, блоки отображений). На рисунках \ref{persistense}-\ref{web} представлены диаграммы классов основных пакетов
\addimghere{program-structure}{0.8}{Диаграмма классов, разбитая по пакетам}{program-structure}

\begin{landscape}
    \addimghere{persistense}{0.75}{Диаграмма классов пакета persistense}{persistense}
\end{landscape}

\addimghere{service}{1}{Диаграмма классов пакета service}{service}
\addimghere{web}{1}{Диаграмма классов пакета web}{web}

\subsection{Описание классов}
На рисунках \ref{crc-table-first}-\ref{crc-table-last} представлены CRC карточки сервиса с элементами социальной сети для публикации фотографий
\newcommand{\bdot}{\(\bullet\hspace{0.5em}\)}

\begin{table}[H]
\caption{CRC ����窠 ����� ImageProcessServiceApplication}\label{crc-table-1}
\begin{tabular}{|p{8cm} p{8cm}|} 
\hline class &  \\
\multicolumn{2}{|c|}{ImageProcessServiceApplication} \\ \hline
\end{tabular}
\begin{tabular}{|p{8cm}|p{8cm}|} 
\multirow{2}{=}{ main ����� } 
& \bdot ColorProperties \\
& \bdot FileStorageProperties \\
\hline 
\end{tabular}
\end{table}

\begin{table}[H]
\caption{CRC ����窠 ����� OnRegistrationCompleteEvent}\label{crc-table-2}
\begin{tabular}{|p{8cm} p{8cm}|} 
\hline class & ApplicationEvent \\
\multicolumn{2}{|c|}{OnRegistrationCompleteEvent} \\ \hline
\end{tabular}
\begin{tabular}{|p{8cm}|p{8cm}|} 
  ����⨥ - �����襭�� ॣ����樨  & \bdot FileStorageProperties \\
\hline 
\end{tabular}
\end{table}

\begin{table}[H]
\caption{CRC ����窠 ����� RegistrationListener}\label{crc-table-3}
\begin{tabular}{|p{8cm} p{8cm}|} 
\hline class & ApplicationListener \\
\multicolumn{2}{|c|}{RegistrationListener} \\ \hline
\end{tabular}
\begin{tabular}{|p{8cm}|p{8cm}|} 
\multirow{3}{=}{ ����⥫� ᮡ�⨩ ॣ����樨 } 
& \bdot UserAccount \\
& \bdot OnRegistrationCompleteEvent \\
& \bdot IUserService \\
\hline 
\end{tabular}
\end{table}

\begin{table}[H]
\caption{CRC ����窠 ����� ColorUtils}\label{crc-table-4}
\begin{tabular}{|p{8cm} p{8cm}|} 
\hline class &  \\
\multicolumn{2}{|c|}{ColorUtils} \\ \hline
\end{tabular}
\begin{tabular}{|p{8cm}|p{8cm}|} 
  �⨫��� �� ࠡ�� � 梥⠬�  & \\
\hline 
\end{tabular}
\end{table}

\begin{table}[H]
\caption{CRC ����窠 ����� }\label{crc-table-5}
\begin{tabular}{|p{8cm} p{8cm}|} 
\hline enum &  \\
\multicolumn{2}{|c|}{} \\ \hline
\end{tabular}
\begin{tabular}{|p{8cm}|p{8cm}|} 
  �஢����� ��⥭�䨪�樨  & \\
\hline 
\end{tabular}
\end{table}

\begin{table}[H]
\caption{CRC ����窠 ����� RandomString}\label{crc-table-6}
\begin{tabular}{|p{8cm} p{8cm}|} 
\hline class &  \\
\multicolumn{2}{|c|}{RandomString} \\ \hline
\end{tabular}
\begin{tabular}{|p{8cm}|p{8cm}|} 
  �⨫�� ��� �����樨 ��砩��� ��ப �������� �����  & \\
\hline 
\end{tabular}
\end{table}

\begin{table}[H]
\caption{CRC ����窠 ����� FileStorageProperties}\label{crc-table-7}
\begin{tabular}{|p{8cm} p{8cm}|} 
\hline class &  \\
\multicolumn{2}{|c|}{FileStorageProperties} \\ \hline
\end{tabular}
\begin{tabular}{|p{8cm}|p{8cm}|} 
  ����ன�� 䠩������ �࠭���� �ࢨ�  & \\
\hline 
\end{tabular}
\end{table}

\begin{table}[H]
\caption{CRC ����窠 ����� ColorProperties}\label{crc-table-8}
\begin{tabular}{|p{8cm} p{8cm}|} 
\hline class &  \\
\multicolumn{2}{|c|}{ColorProperties} \\ \hline
\end{tabular}
\begin{tabular}{|p{8cm}|p{8cm}|} 
  ����ன�� �ࢨ� �� 梥⮢�� �����ਧ�樨  & \\
\hline 
\end{tabular}
\end{table}

\begin{table}[H]
\caption{CRC ����窠 ����� SessionListenerWithMetrics}\label{crc-table-9}
\begin{tabular}{|p{8cm} p{8cm}|} 
\hline class & HttpSessionListener \\
\multicolumn{2}{|c|}{SessionListenerWithMetrics} \\ \hline
\end{tabular}
\begin{tabular}{|p{8cm}|p{8cm}|} 
  ����⥫� ��ᨩ, �⢥��騩 �� ���ਪ�  & \bdot IUserService \\
\hline 
\end{tabular}
\end{table}

\begin{table}[H]
\caption{CRC ����窠 ����� PasswordDto}\label{crc-table-10}
\begin{tabular}{|p{8cm} p{8cm}|} 
\hline class &  \\
\multicolumn{2}{|c|}{PasswordDto} \\ \hline
\end{tabular}
\begin{tabular}{|p{8cm}|p{8cm}|} 
  DTO - ���� + ���� ��஫�  & \bdot IUserService \\
\hline 
\end{tabular}
\end{table}

\begin{table}[H]
\caption{CRC ����窠 ����� PixelPoint}\label{crc-table-11}
\begin{tabular}{|p{8cm} p{8cm}|} 
\hline class & Clusterable \\
\multicolumn{2}{|c|}{PixelPoint} \\ \hline
\end{tabular}
\begin{tabular}{|p{8cm}|p{8cm}|} 
  ����� ������  & \\
\hline 
\end{tabular}
\end{table}

\begin{table}[H]
\caption{CRC ����窠 ����� TagDto}\label{crc-table-12}
\begin{tabular}{|p{8cm} p{8cm}|} 
\hline class &  \\
\multicolumn{2}{|c|}{TagDto} \\ \hline
\end{tabular}
\begin{tabular}{|p{8cm}|p{8cm}|} 
  DTO - ⥣  & \\
\hline 
\end{tabular}
\end{table}

\begin{table}[H]
\caption{CRC ����窠 ����� ColorTreemapNode}\label{crc-table-13}
\begin{tabular}{|p{8cm} p{8cm}|} 
\hline class &  \\
\multicolumn{2}{|c|}{ColorTreemapNode} \\ \hline
\end{tabular}
\begin{tabular}{|p{8cm}|p{8cm}|} 
  DTO - 㧥� Treemap  & \\
\hline 
\end{tabular}
\end{table}

\begin{table}[H]
\caption{CRC ����窠 ����� UserDto}\label{crc-table-14}
\begin{tabular}{|p{8cm} p{8cm}|} 
\hline class &  \\
\multicolumn{2}{|c|}{UserDto} \\ \hline
\end{tabular}
\begin{tabular}{|p{8cm}|p{8cm}|} 
\multirow{3}{=}{ DTO - ���짮��⥫� } 
& \bdot PasswordMatches \\
& \bdot ValidEmail \\
& \bdot ValidPassword \\
\hline 
\end{tabular}
\end{table}

\begin{table}[H]
\caption{CRC ����窠 ����� ColorCluster}\label{crc-table-15}
\begin{tabular}{|p{8cm} p{8cm}|} 
\hline class &  \\
\multicolumn{2}{|c|}{ColorCluster} \\ \hline
\end{tabular}
\begin{tabular}{|p{8cm}|p{8cm}|} 
  DTO - 梥⮢�� ������  & \\
\hline 
\end{tabular}
\end{table}

\begin{table}[H]
\caption{CRC ����窠 ����� GenericResponse}\label{crc-table-16}
\begin{tabular}{|p{8cm} p{8cm}|} 
\hline class &  \\
\multicolumn{2}{|c|}{GenericResponse} \\ \hline
\end{tabular}
\begin{tabular}{|p{8cm}|p{8cm}|} 
  ��騩 �⢥�  & \\
\hline 
\end{tabular}
\end{table}

\begin{table}[H]
\caption{CRC ����窠 ����� PhotoController}\label{crc-table-17}
\begin{tabular}{|p{8cm} p{8cm}|} 
\hline class &  \\
\multicolumn{2}{|c|}{PhotoController} \\ \hline
\end{tabular}
\begin{tabular}{|p{8cm}|p{8cm}|} 
  ����஫��� ����㯠 � �⮣���  & \\
\hline 
\end{tabular}
\end{table}

\begin{table}[H]
\caption{CRC ����窠 ����� ColorsController}\label{crc-table-18}
\begin{tabular}{|p{8cm} p{8cm}|} 
\hline class &  \\
\multicolumn{2}{|c|}{ColorsController} \\ \hline
\end{tabular}
\begin{tabular}{|p{8cm}|p{8cm}|} 
\multirow{4}{=}{ ����஫��� 梥⮢ } 
& \bdot PhotoGroupColorArea \\
& \bdot ColorTreemapNodeMapper \\
& \bdot ColorsService \\
& \bdot ColorTreemapNode \\
\hline 
\end{tabular}
\end{table}

\begin{table}[H]
\caption{CRC ����窠 ����� AuthController}\label{crc-table-19}
\begin{tabular}{|p{8cm} p{8cm}|} 
\hline class &  \\
\multicolumn{2}{|c|}{AuthController} \\ \hline
\end{tabular}
\begin{tabular}{|p{8cm}|p{8cm}|} 
  ����஫���  ��⥭�䨪�樨  & \bdot ColorTreemapNode \\
\hline 
\end{tabular}
\end{table}

\begin{table}[H]
\caption{CRC ����窠 ����� RegistrationController}\label{crc-table-20}
\begin{tabular}{|p{8cm} p{8cm}|} 
\hline class &  \\
\multicolumn{2}{|c|}{RegistrationController} \\ \hline
\end{tabular}
\begin{tabular}{|p{8cm}|p{8cm}|} 
\multirow{12}{=}{ ����஫��� ॣ����樨 } 
& \bdot UserAccount \\
& \bdot VerificationToken \\
& \bdot OnRegistrationCompleteEvent \\
& \bdot IUserService \\
& \bdot ISecurityUserService \\
& \bdot PasswordDto \\
& \bdot UserDto \\
& \bdot InvalidOldPasswordException \\
& \bdot GenericResponse \\
& \bdot UserService \\
& \bdot PasswordResetToken \\
& \bdot Privilege \\
\hline 
\end{tabular}
\end{table}

\begin{table}[H]
\caption{CRC ����窠 ����� TagsController}\label{crc-table-21}
\begin{tabular}{|p{8cm} p{8cm}|} 
\hline class &  \\
\multicolumn{2}{|c|}{TagsController} \\ \hline
\end{tabular}
\begin{tabular}{|p{8cm}|p{8cm}|} 
\multirow{5}{=}{ ����஫��� ����㯠 � ⥣�� } 
& \bdot ColorTreemapNodeMapper \\
& \bdot PhotoGroupColorArea \\
& \bdot TagsService \\
& \bdot ColorTreemapNode \\
& \bdot TagDto \\
\hline 
\end{tabular}
\end{table}

\begin{table}[H]
\caption{CRC ����窠 ����� FileController}\label{crc-table-22}
\begin{tabular}{|p{8cm} p{8cm}|} 
\hline class &  \\
\multicolumn{2}{|c|}{FileController} \\ \hline
\end{tabular}
\begin{tabular}{|p{8cm}|p{8cm}|} 
\multirow{3}{=}{ ����஫��� ����㯠 � 䠩��� } 
& \bdot UserDetailService \\
& \bdot GenericResponse \\
& \bdot FileStorageService \\
\hline 
\end{tabular}
\end{table}

\begin{table}[H]
\caption{CRC ����窠 ����� UserController}\label{crc-table-23}
\begin{tabular}{|p{8cm} p{8cm}|} 
\hline class &  \\
\multicolumn{2}{|c|}{UserController} \\ \hline
\end{tabular}
\begin{tabular}{|p{8cm}|p{8cm}|} 
  ����஫��� ���짮��⥫��  & \bdot FileStorageService \\
\hline 
\end{tabular}
\end{table}

\begin{table}[H]
\caption{CRC ����窠 ����� ReCaptchaUnavailableException}\label{crc-table-24}
\begin{tabular}{|p{8cm} p{8cm}|} 
\hline class & RuntimeException \\
\multicolumn{2}{|c|}{ReCaptchaUnavailableException} \\ \hline
\end{tabular}
\begin{tabular}{|p{8cm}|p{8cm}|} 
  ����� ������㯭�  & \\
\hline 
\end{tabular}
\end{table}

\begin{table}[H]
\caption{CRC ����窠 ����� UserAlreadyExistException}\label{crc-table-25}
\begin{tabular}{|p{8cm} p{8cm}|} 
\hline class & RuntimeException \\
\multicolumn{2}{|c|}{UserAlreadyExistException} \\ \hline
\end{tabular}
\begin{tabular}{|p{8cm}|p{8cm}|} 
  ���짮��⥫� 㦥 �������  & \\
\hline 
\end{tabular}
\end{table}

\begin{table}[H]
\caption{CRC ����窠 ����� UserNotFoundException}\label{crc-table-26}
\begin{tabular}{|p{8cm} p{8cm}|} 
\hline class & RuntimeException \\
\multicolumn{2}{|c|}{UserNotFoundException} \\ \hline
\end{tabular}
\begin{tabular}{|p{8cm}|p{8cm}|} 
  ���짮��⥫� �� ������  & \\
\hline 
\end{tabular}
\end{table}

\begin{table}[H]
\caption{CRC ����窠 ����� InvalidOldPasswordException}\label{crc-table-27}
\begin{tabular}{|p{8cm} p{8cm}|} 
\hline class & RuntimeException \\
\multicolumn{2}{|c|}{InvalidOldPasswordException} \\ \hline
\end{tabular}
\begin{tabular}{|p{8cm}|p{8cm}|} 
  �᪫�祭�� ��������� �� ����୮ ��������� ��஬ ��஫�  & \\
\hline 
\end{tabular}
\end{table}

\begin{table}[H]
\caption{CRC ����窠 ����� RestResponseEntityExceptionHandler}\label{crc-table-28}
\begin{tabular}{|p{8cm} p{8cm}|} 
\hline class & ResponseEntityExceptionHandler \\
\multicolumn{2}{|c|}{RestResponseEntityExceptionHandler} \\ \hline
\end{tabular}
\begin{tabular}{|p{8cm}|p{8cm}|} 
\multirow{6}{=}{ REST ��ࠡ��稪 �⢥⮢ } 
& \bdot GenericResponse \\
& \bdot InvalidOldPasswordException \\
& \bdot ReCaptchaInvalidException \\
& \bdot UserNotFoundException \\
& \bdot UserAlreadyExistException \\
& \bdot ReCaptchaUnavailableException \\
\hline 
\end{tabular}
\end{table}

\begin{table}[H]
\caption{CRC ����窠 ����� ReCaptchaInvalidException}\label{crc-table-29}
\begin{tabular}{|p{8cm} p{8cm}|} 
\hline class & RuntimeException \\
\multicolumn{2}{|c|}{ReCaptchaInvalidException} \\ \hline
\end{tabular}
\begin{tabular}{|p{8cm}|p{8cm}|} 
  �᪫�祭�� ��������� �� ����୮ ��������� �����  & \\
\hline 
\end{tabular}
\end{table}

\begin{table}[H]
\caption{CRC ����窠 ����� ColorTreemapNodeMapper}\label{crc-table-30}
\begin{tabular}{|p{8cm} p{8cm}|} 
\hline class &  \\
\multicolumn{2}{|c|}{ColorTreemapNodeMapper} \\ \hline
\end{tabular}
\begin{tabular}{|p{8cm}|p{8cm}|} 
\multirow{2}{=}{  ������ ��魮�� �� ⨯� PhotoGroupColorArea � 㧫� ColorTreemap } 
& \bdot PhotoGroupColorArea \\
& \bdot ColorTreemapNode \\
\hline 
\end{tabular}
\end{table}

\begin{table}[H]
\caption{CRC ����窠 ����� TagRepository}\label{crc-table-31}
\begin{tabular}{|p{8cm} p{8cm}|} 
\hline interface & JpaRepository \\
\multicolumn{2}{|c|}{TagRepository} \\ \hline
\end{tabular}
\begin{tabular}{|p{8cm}|p{8cm}|} 
  �������਩ � ⥣���  & \bdot ColorTreemapNode \\
\hline 
\end{tabular}
\end{table}

\begin{table}[H]
\caption{CRC ����窠 ����� PhotoGroupRepository}\label{crc-table-32}
\begin{tabular}{|p{8cm} p{8cm}|} 
\hline interface & JpaRepository \\
\multicolumn{2}{|c|}{PhotoGroupRepository} \\ \hline
\end{tabular}
\begin{tabular}{|p{8cm}|p{8cm}|} 
  �������਩ � ��㯯��� �⮣�䨩  & \bdot ColorTreemapNode \\
\hline 
\end{tabular}
\end{table}

\begin{table}[H]
\caption{CRC ����窠 ����� PhotoGroupColorAreaRepository}\label{crc-table-33}
\begin{tabular}{|p{8cm} p{8cm}|} 
\hline interface & JpaRepository \\
\multicolumn{2}{|c|}{PhotoGroupColorAreaRepository} \\ \hline
\end{tabular}
\begin{tabular}{|p{8cm}|p{8cm}|} 
\multirow{2}{=}{ �������਩ � 梥⮢묨 �����ࠬ� } 
& \bdot PhotoGroup \\
& \bdot PhotoGroupColorArea \\
\hline 
\end{tabular}
\end{table}

\begin{table}[H]
\caption{CRC ����窠 ����� PhotoGroupMetadataRepository}\label{crc-table-34}
\begin{tabular}{|p{8cm} p{8cm}|} 
\hline interface & JpaRepository \\
\multicolumn{2}{|c|}{PhotoGroupMetadataRepository} \\ \hline
\end{tabular}
\begin{tabular}{|p{8cm}|p{8cm}|} 
  �������਩ � ��⠤���묨 �⮣�䨩  & \bdot PhotoGroupColorArea \\
\hline 
\end{tabular}
\end{table}

\begin{table}[H]
\caption{CRC ����窠 ����� PhotoGroupTagRepository}\label{crc-table-35}
\begin{tabular}{|p{8cm} p{8cm}|} 
\hline interface & JpaRepository \\
\multicolumn{2}{|c|}{PhotoGroupTagRepository} \\ \hline
\end{tabular}
\begin{tabular}{|p{8cm}|p{8cm}|} 
\multirow{2}{=}{ �������਩ � ⥣��� ��㯯� �⮣�䨩 } 
& \bdot PhotoGroup \\
& \bdot PhotoGroupTag \\
\hline 
\end{tabular}
\end{table}

\begin{table}[H]
\caption{CRC ����窠 ����� WorldNetClassRepository}\label{crc-table-36}
\begin{tabular}{|p{8cm} p{8cm}|} 
\hline interface & JpaRepository \\
\multicolumn{2}{|c|}{WorldNetClassRepository} \\ \hline
\end{tabular}
\begin{tabular}{|p{8cm}|p{8cm}|} 
  �������਩ � ����ᠬ� WorldNet  & \bdot PhotoGroupTag \\
\hline 
\end{tabular}
\end{table}

\begin{table}[H]
\caption{CRC ����窠 ����� PhotoSingleRepository}\label{crc-table-37}
\begin{tabular}{|p{8cm} p{8cm}|} 
\hline interface & JpaRepository \\
\multicolumn{2}{|c|}{PhotoSingleRepository} \\ \hline
\end{tabular}
\begin{tabular}{|p{8cm}|p{8cm}|} 
\multirow{3}{=}{  �������਩ � �⮣��ﬨ } 
& \bdot PhotoGroup \\
& \bdot PhotoSingle \\
& \bdot PhotoType \\
\hline 
\end{tabular}
\end{table}

\begin{table}[H]
\caption{CRC ����窠 ����� RoleRepository}\label{crc-table-38}
\begin{tabular}{|p{8cm} p{8cm}|} 
\hline interface & JpaRepository \\
\multicolumn{2}{|c|}{RoleRepository} \\ \hline
\end{tabular}
\begin{tabular}{|p{8cm}|p{8cm}|} 
  �������਩ � ஫ﬨ ���짮��⥫��  & \bdot PhotoType \\
\hline 
\end{tabular}
\end{table}

\begin{table}[H]
\caption{CRC ����窠 ����� UserRepository}\label{crc-table-39}
\begin{tabular}{|p{8cm} p{8cm}|} 
\hline interface & JpaRepository \\
\multicolumn{2}{|c|}{UserRepository} \\ \hline
\end{tabular}
\begin{tabular}{|p{8cm}|p{8cm}|} 
  �������਩ � ���짮��⥫ﬨ  & \bdot PhotoType \\
\hline 
\end{tabular}
\end{table}

\begin{table}[H]
\caption{CRC ����窠 ����� PasswordResetTokenRepository}\label{crc-table-40}
\begin{tabular}{|p{8cm} p{8cm}|} 
\hline interface & JpaRepository \\
\multicolumn{2}{|c|}{PasswordResetTokenRepository} \\ \hline
\end{tabular}
\begin{tabular}{|p{8cm}|p{8cm}|} 
\multirow{2}{=}{ �������਩ � ⮪����� ��� ����⠭������� ��஫�� } 
& \bdot PasswordResetToken \\
& \bdot UserAccount \\
\hline 
\end{tabular}
\end{table}

\begin{table}[H]
\caption{CRC ����窠 ����� VerificationTokenRepository}\label{crc-table-41}
\begin{tabular}{|p{8cm} p{8cm}|} 
\hline interface & JpaRepository \\
\multicolumn{2}{|c|}{VerificationTokenRepository} \\ \hline
\end{tabular}
\begin{tabular}{|p{8cm}|p{8cm}|} 
\multirow{2}{=}{ �������਩ � ⮪����� ��� ���䨪�樨 } 
& \bdot UserAccount \\
& \bdot VerificationToken \\
\hline 
\end{tabular}
\end{table}

\begin{table}[H]
\caption{CRC ����窠 ����� PrivilegeRepository}\label{crc-table-42}
\begin{tabular}{|p{8cm} p{8cm}|} 
\hline interface & JpaRepository \\
\multicolumn{2}{|c|}{PrivilegeRepository} \\ \hline
\end{tabular}
\begin{tabular}{|p{8cm}|p{8cm}|} 
  �������਩ � �ࠢ��� ���짮��⥫��  & \bdot VerificationToken \\
\hline 
\end{tabular}
\end{table}

\begin{table}[H]
\caption{CRC ����窠 ����� Post}\label{crc-table-43}
\begin{tabular}{|p{8cm} p{8cm}|} 
\hline class &  \\
\multicolumn{2}{|c|}{Post} \\ \hline
\end{tabular}
\begin{tabular}{|p{8cm}|p{8cm}|} 
\multirow{2}{=}{ ���� } 
& \bdot UserAccount \\
& \bdot PrivacyType \\
\hline 
\end{tabular}
\end{table}

\begin{table}[H]
\caption{CRC ����窠 ����� PhotoType}\label{crc-table-44}
\begin{tabular}{|p{8cm} p{8cm}|} 
\hline enum &  \\
\multicolumn{2}{|c|}{PhotoType} \\ \hline
\end{tabular}
\begin{tabular}{|p{8cm}|p{8cm}|} 
  ��� �⮮��䨨  & \\
\hline 
\end{tabular}
\end{table}

\begin{table}[H]
\caption{CRC ����窠 ����� PasswordResetToken}\label{crc-table-45}
\begin{tabular}{|p{8cm} p{8cm}|} 
\hline class &  \\
\multicolumn{2}{|c|}{PasswordResetToken} \\ \hline
\end{tabular}
\begin{tabular}{|p{8cm}|p{8cm}|} 
  ����� ��� ��� ��஫�  & \bdot PrivacyType \\
\hline 
\end{tabular}
\end{table}

\begin{table}[H]
\caption{CRC ����窠 ����� PhotoSingle}\label{crc-table-46}
\begin{tabular}{|p{8cm} p{8cm}|} 
\hline class &  \\
\multicolumn{2}{|c|}{PhotoSingle} \\ \hline
\end{tabular}
\begin{tabular}{|p{8cm}|p{8cm}|} 
\multirow{2}{=}{ ��⮣��� } 
& \bdot PhotoGroup \\
& \bdot PhotoType \\
\hline 
\end{tabular}
\end{table}

\begin{table}[H]
\caption{CRC ����窠 ����� PhotoGroup}\label{crc-table-47}
\begin{tabular}{|p{8cm} p{8cm}|} 
\hline class &  \\
\multicolumn{2}{|c|}{PhotoGroup} \\ \hline
\end{tabular}
\begin{tabular}{|p{8cm}|p{8cm}|} 
\multirow{2}{=}{ ��㯯� �⮣�䨩 } 
& \bdot UserAccount \\
& \bdot PrivacyType \\
\hline 
\end{tabular}
\end{table}

\begin{table}[H]
\caption{CRC ����窠 ����� PhotoGroupMetadata}\label{crc-table-48}
\begin{tabular}{|p{8cm} p{8cm}|} 
\hline class &  \\
\multicolumn{2}{|c|}{PhotoGroupMetadata} \\ \hline
\end{tabular}
\begin{tabular}{|p{8cm}|p{8cm}|} 
\multirow{2}{=}{ ��⠤���� � �⮣�䨨 } 
& \bdot PhotoGroup \\
& \bdot Camera \\
\hline 
\end{tabular}
\end{table}

\begin{table}[H]
\caption{CRC ����窠 ����� PrivacyType}\label{crc-table-49}
\begin{tabular}{|p{8cm} p{8cm}|} 
\hline enum &  \\
\multicolumn{2}{|c|}{PrivacyType} \\ \hline
\end{tabular}
\begin{tabular}{|p{8cm}|p{8cm}|} 
  ��� access ����  & \\
\hline 
\end{tabular}
\end{table}

\begin{table}[H]
\caption{CRC ����窠 ����� Tag}\label{crc-table-50}
\begin{tabular}{|p{8cm} p{8cm}|} 
\hline class &  \\
\multicolumn{2}{|c|}{Tag} \\ \hline
\end{tabular}
\begin{tabular}{|p{8cm}|p{8cm}|} 
  ���  & \bdot Camera \\
\hline 
\end{tabular}
\end{table}

\begin{table}[H]
\caption{CRC ����窠 ����� Privilege}\label{crc-table-51}
\begin{tabular}{|p{8cm} p{8cm}|} 
\hline class &  \\
\multicolumn{2}{|c|}{Privilege} \\ \hline
\end{tabular}
\begin{tabular}{|p{8cm}|p{8cm}|} 
  �ࠢ� ���짮��⥫�  & \bdot Camera \\
\hline 
\end{tabular}
\end{table}

\begin{table}[H]
\caption{CRC ����窠 ����� PostUserLike}\label{crc-table-52}
\begin{tabular}{|p{8cm} p{8cm}|} 
\hline class &  \\
\multicolumn{2}{|c|}{PostUserLike} \\ \hline
\end{tabular}
\begin{tabular}{|p{8cm}|p{8cm}|} 
\multirow{2}{=}{ ���� ����� } 
& \bdot Post \\
& \bdot UserAccount \\
\hline 
\end{tabular}
\end{table}

\begin{table}[H]
\caption{CRC ����窠 ����� PhotoGroupColorArea}\label{crc-table-53}
\begin{tabular}{|p{8cm} p{8cm}|} 
\hline class &  \\
\multicolumn{2}{|c|}{PhotoGroupColorArea} \\ \hline
\end{tabular}
\begin{tabular}{|p{8cm}|p{8cm}|} 
  ���⮢�� ������  & \bdot UserAccount \\
\hline 
\end{tabular}
\end{table}

\begin{table}[H]
\caption{CRC ����窠 ����� PhotoGroupUserAccess}\label{crc-table-54}
\begin{tabular}{|p{8cm} p{8cm}|} 
\hline class &  \\
\multicolumn{2}{|c|}{PhotoGroupUserAccess} \\ \hline
\end{tabular}
\begin{tabular}{|p{8cm}|p{8cm}|} 
\multirow{2}{=}{ ������ � access ���� } 
& \bdot PhotoGroup \\
& \bdot UserAccount \\
\hline 
\end{tabular}
\end{table}

\begin{table}[H]
\caption{CRC ����窠 ����� PhotoGroupPost}\label{crc-table-55}
\begin{tabular}{|p{8cm} p{8cm}|} 
\hline class &  \\
\multicolumn{2}{|c|}{PhotoGroupPost} \\ \hline
\end{tabular}
\begin{tabular}{|p{8cm}|p{8cm}|} 
\multirow{2}{=}{ ���� ��㯯� �⮣�䨩-���� } 
& \bdot PhotoGroup \\
& \bdot Post \\
\hline 
\end{tabular}
\end{table}

\begin{table}[H]
\caption{CRC ����窠 ����� PhotoGroupTag}\label{crc-table-56}
\begin{tabular}{|p{8cm} p{8cm}|} 
\hline class &  \\
\multicolumn{2}{|c|}{PhotoGroupTag} \\ \hline
\end{tabular}
\begin{tabular}{|p{8cm}|p{8cm}|} 
\multirow{2}{=}{ ��� �⮣�䨨 } 
& \bdot PhotoGroup \\
& \bdot Tag \\
\hline 
\end{tabular}
\end{table}

\begin{table}[H]
\caption{CRC ����窠 ����� PostTag}\label{crc-table-57}
\begin{tabular}{|p{8cm} p{8cm}|} 
\hline class &  \\
\multicolumn{2}{|c|}{PostTag} \\ \hline
\end{tabular}
\begin{tabular}{|p{8cm}|p{8cm}|} 
\multirow{2}{=}{ ��� ���� } 
& \bdot Post \\
& \bdot Tag \\
\hline 
\end{tabular}
\end{table}

\begin{table}[H]
\caption{CRC ����窠 ����� VerificationToken}\label{crc-table-58}
\begin{tabular}{|p{8cm} p{8cm}|} 
\hline class &  \\
\multicolumn{2}{|c|}{VerificationToken} \\ \hline
\end{tabular}
\begin{tabular}{|p{8cm}|p{8cm}|} 
  ����� ��� ���䨪�樨  & \bdot Tag \\
\hline 
\end{tabular}
\end{table}

\begin{table}[H]
\caption{CRC ����窠 ����� WorldNetClass}\label{crc-table-59}
\begin{tabular}{|p{8cm} p{8cm}|} 
\hline class &  \\
\multicolumn{2}{|c|}{WorldNetClass} \\ \hline
\end{tabular}
\begin{tabular}{|p{8cm}|p{8cm}|} 
  ����� WorldNet  & \\
\hline 
\end{tabular}
\end{table}

\begin{table}[H]
\caption{CRC ����窠 ����� Comment}\label{crc-table-60}
\begin{tabular}{|p{8cm} p{8cm}|} 
\hline class &  \\
\multicolumn{2}{|c|}{Comment} \\ \hline
\end{tabular}
\begin{tabular}{|p{8cm}|p{8cm}|} 
\multirow{2}{=}{ �������਩ ���짮��⥫� } 
& \bdot UserAccount \\
& \bdot PhotoGroup \\
\hline 
\end{tabular}
\end{table}

\begin{table}[H]
\caption{CRC ����窠 ����� UserAuthCookie}\label{crc-table-61}
\begin{tabular}{|p{8cm} p{8cm}|} 
\hline class &  \\
\multicolumn{2}{|c|}{UserAuthCookie} \\ \hline
\end{tabular}
\begin{tabular}{|p{8cm}|p{8cm}|} 
  �㪨 ���짮��⥫�  & \bdot PhotoGroup \\
\hline 
\end{tabular}
\end{table}

\begin{table}[H]
\caption{CRC ����窠 ����� Role}\label{crc-table-62}
\begin{tabular}{|p{8cm} p{8cm}|} 
\hline class &  \\
\multicolumn{2}{|c|}{Role} \\ \hline
\end{tabular}
\begin{tabular}{|p{8cm}|p{8cm}|} 
\multirow{2}{=}{ ���� ���짮��⥫� } 
& \bdot UserAccount \\
& \bdot Privilege \\
\hline 
\end{tabular}
\end{table}

\begin{table}[H]
\caption{CRC ����窠 ����� Camera}\label{crc-table-63}
\begin{tabular}{|p{8cm} p{8cm}|} 
\hline class &  \\
\multicolumn{2}{|c|}{Camera} \\ \hline
\end{tabular}
\begin{tabular}{|p{8cm}|p{8cm}|} 
  ��⮪����  & \\
\hline 
\end{tabular}
\end{table}

\begin{table}[H]
\caption{CRC ����窠 ����� PhotoGroupUserLike}\label{crc-table-64}
\begin{tabular}{|p{8cm} p{8cm}|} 
\hline class &  \\
\multicolumn{2}{|c|}{PhotoGroupUserLike} \\ \hline
\end{tabular}
\begin{tabular}{|p{8cm}|p{8cm}|} 
\multirow{2}{=}{ ���� �⮣�䨨 } 
& \bdot PhotoGroup \\
& \bdot UserAccount \\
\hline 
\end{tabular}
\end{table}

\begin{table}[H]
\caption{CRC ����窠 ����� UserAccount}\label{crc-table-65}
\begin{tabular}{|p{8cm} p{8cm}|} 
\hline class &  \\
\multicolumn{2}{|c|}{UserAccount} \\ \hline
\end{tabular}
\begin{tabular}{|p{8cm}|p{8cm}|} 
  ������ ���짮��⥫�  & \bdot UserAccount \\
\hline 
\end{tabular}
\end{table}

\begin{table}[H]
\caption{CRC ����窠 ����� IUserService}\label{crc-table-66}
\begin{tabular}{|p{8cm} p{8cm}|} 
\hline interface &  \\
\multicolumn{2}{|c|}{IUserService} \\ \hline
\end{tabular}
\begin{tabular}{|p{8cm}|p{8cm}|} 
\multirow{5}{=}{ ����䥩� ���짮��⥫�᪮�� �ࢨ� } 
& \bdot PasswordResetToken \\
& \bdot UserAccount \\
& \bdot VerificationToken \\
& \bdot UserDto \\
& \bdot UserAlreadyExistException \\
\hline 
\end{tabular}
\end{table}

\begin{table}[H]
\caption{CRC ����窠 ����� ImageColorClusteriserService}\label{crc-table-67}
\begin{tabular}{|p{8cm} p{8cm}|} 
\hline class &  \\
\multicolumn{2}{|c|}{ImageColorClusteriserService} \\ \hline
\end{tabular}
\begin{tabular}{|p{8cm}|p{8cm}|} 
\multirow{2}{=}{ ��ࢨ� �����ਧ�樨 } 
& \bdot ColorCluster \\
& \bdot PixelPoint \\
\hline 
\end{tabular}
\end{table}

\begin{table}[H]
\caption{CRC ����窠 ����� FileStorageService}\label{crc-table-68}
\begin{tabular}{|p{8cm} p{8cm}|} 
\hline class &  \\
\multicolumn{2}{|c|}{FileStorageService} \\ \hline
\end{tabular}
\begin{tabular}{|p{8cm}|p{8cm}|} 
\multirow{10}{=}{ ������� �ࢨ� } 
& \bdot FileStorageException \\
& \bdot MyFileNotFoundException \\
& \bdot PhotoGroup \\
& \bdot PhotoSingle \\
& \bdot PhotoType \\
& \bdot UserAccount \\
& \bdot FileStorageProperties \\
& \bdot PhotoGroupRepository \\
& \bdot PhotoSingleRepository \\
& \bdot RandomString \\
\hline 
\end{tabular}
\end{table}

\begin{table}[H]
\caption{CRC ����窠 ����� UserService}\label{crc-table-69}
\begin{tabular}{|p{8cm} p{8cm}|} 
\hline class & IUserService \\
\multicolumn{2}{|c|}{UserService} \\ \hline
\end{tabular}
\begin{tabular}{|p{8cm}|p{8cm}|} 
\multirow{9}{=}{ ��ࢨ� ���짮��⥫�� } 
& \bdot PasswordResetTokenRepository \\
& \bdot RoleRepository \\
& \bdot UserRepository \\
& \bdot VerificationTokenRepository \\
& \bdot PasswordResetToken \\
& \bdot UserAccount \\
& \bdot VerificationToken \\
& \bdot UserDto \\
& \bdot UserAlreadyExistException \\
\hline 
\end{tabular}
\end{table}

\begin{table}[H]
\caption{CRC ����窠 ����� TagsService}\label{crc-table-70}
\begin{tabular}{|p{8cm} p{8cm}|} 
\hline class &  \\
\multicolumn{2}{|c|}{TagsService} \\ \hline
\end{tabular}
\begin{tabular}{|p{8cm}|p{8cm}|} 
\multirow{7}{=}{ ��ࢨ� ⥣�� } 
& \bdot PhotoGroupRepository \\
& \bdot PhotoGroupTagRepository \\
& \bdot TagRepository \\
& \bdot PhotoGroup \\
& \bdot PhotoGroupTag \\
& \bdot TagDto \\
& \bdot Tag \\
\hline 
\end{tabular}
\end{table}

\begin{table}[H]
\caption{CRC ����窠 ����� ColorsService}\label{crc-table-71}
\begin{tabular}{|p{8cm} p{8cm}|} 
\hline class &  \\
\multicolumn{2}{|c|}{ColorsService} \\ \hline
\end{tabular}
\begin{tabular}{|p{8cm}|p{8cm}|} 
\multirow{4}{=}{ ��ࢨ� ��� ࠡ��� � 梥⠬� � �����ࠬ� } 
& \bdot PhotoGroup \\
& \bdot PhotoGroupColorArea \\
& \bdot PhotoGroupColorAreaRepository \\
& \bdot PhotoGroupRepository \\
\hline 
\end{tabular}
\end{table}

\begin{table}[H]
\caption{CRC ����窠 ����� MetadataService}\label{crc-table-72}
\begin{tabular}{|p{8cm} p{8cm}|} 
\hline class &  \\
\multicolumn{2}{|c|}{MetadataService} \\ \hline
\end{tabular}
\begin{tabular}{|p{8cm}|p{8cm}|} 
\multirow{2}{=}{ ��ࢨ� ��⠤����� } 
& \bdot PhotoGroupMetadata \\
& \bdot PhotoGroupMetadataRepository \\
\hline 
\end{tabular}
\end{table}

\begin{table}[H]
\caption{CRC ����窠 ����� PhotoProcessorService}\label{crc-table-73}
\begin{tabular}{|p{8cm} p{8cm}|} 
\hline class &  \\
\multicolumn{2}{|c|}{PhotoProcessorService} \\ \hline
\end{tabular}
\begin{tabular}{|p{8cm}|p{8cm}|} 
\multirow{20}{=}{ ��ࢨ� ��ࠡ�⪨ �⮣�䨩 (ᮤ�ন� �� ������� �� ��ࠡ�⪥) } 
& \bdot PhotoGroupTagRepository \\
& \bdot TagRepository \\
& \bdot WorldNetClassRepository \\
& \bdot PhotoGroupTag \\
& \bdot Tag \\
& \bdot WorldNetClass \\
& \bdot ImageClassifier \\
& \bdot Label \\
& \bdot PhotoGroupColorAreaRepository \\
& \bdot PhotoGroupMetadataRepository \\
& \bdot PhotoGroupRepository \\
& \bdot PhotoSingleRepository \\
& \bdot PhotoGroup \\
& \bdot PhotoGroupColorArea \\
& \bdot PhotoSingle \\
& \bdot PhotoType \\
& \bdot ColorCluster \\
& \bdot ColorUtils \\
& \bdot FileStorageService \\
& \bdot ImageColorClusteriserService \\
\hline 
\end{tabular}
\end{table}

\begin{table}[H]
\caption{CRC ����窠 ����� MySimpleUrlAuthenticationSuccessHandler}\label{crc-table-74}
\begin{tabular}{|p{8cm} p{8cm}|} 
\hline class & AuthenticationSuccessHandler \\
\multicolumn{2}{|c|}{MySimpleUrlAuthenticationSuccessHandler} \\ \hline
\end{tabular}
\begin{tabular}{|p{8cm}|p{8cm}|} 
  ��ࠡ��稪 ��⥭�䨪�樨  & \\
\hline 
\end{tabular}
\end{table}

\begin{table}[H]
\caption{CRC ����窠 ����� ISecurityUserService}\label{crc-table-75}
\begin{tabular}{|p{8cm} p{8cm}|} 
\hline interface &  \\
\multicolumn{2}{|c|}{ISecurityUserService} \\ \hline
\end{tabular}
\begin{tabular}{|p{8cm}|p{8cm}|} 
  ����䥩� �ࢨ� ������᭮�� ���짮��⥫��  & \bdot ImageColorClusteriserService \\
\hline 
\end{tabular}
\end{table}

\begin{table}[H]
\caption{CRC ����窠 ����� SessionFilter}\label{crc-table-76}
\begin{tabular}{|p{8cm} p{8cm}|} 
\hline class & Filter \\
\multicolumn{2}{|c|}{SessionFilter} \\ \hline
\end{tabular}
\begin{tabular}{|p{8cm}|p{8cm}|} 
  ������ ��ᨩ  & \\
\hline 
\end{tabular}
\end{table}

\begin{table}[H]
\caption{CRC ����窠 ����� UserSecurityService}\label{crc-table-77}
\begin{tabular}{|p{8cm} p{8cm}|} 
\hline class & ISecurityUserService \\
\multicolumn{2}{|c|}{UserSecurityService} \\ \hline
\end{tabular}
\begin{tabular}{|p{8cm}|p{8cm}|} 
\multirow{3}{=}{ ��ࢨ� ������᭮�� ���짮��⥫�� } 
& \bdot PasswordResetTokenRepository \\
& \bdot PasswordResetToken \\
& \bdot UserAccount \\
\hline 
\end{tabular}
\end{table}

\begin{table}[H]
\caption{CRC ����窠 ����� ImageClassifier}\label{crc-table-78}
\begin{tabular}{|p{8cm} p{8cm}|} 
\hline class &  \\
\multicolumn{2}{|c|}{ImageClassifier} \\ \hline
\end{tabular}
\begin{tabular}{|p{8cm}|p{8cm}|} 
\multirow{2}{=}{ �����䨪��� ����ࠦ���� } 
& \bdot Prediction \\
& \bdot Label \\
\hline 
\end{tabular}
\end{table}

\begin{table}[H]
\caption{CRC ����窠 ����� Label}\label{crc-table-79}
\begin{tabular}{|p{8cm} p{8cm}|} 
\hline class & Comparable \\
\multicolumn{2}{|c|}{Label} \\ \hline
\end{tabular}
\begin{tabular}{|p{8cm}|p{8cm}|} 
  POJO, �������騩 ����⭮��� �ਭ��������� � ������ � �����  & \\
\hline 
\end{tabular}
\end{table}

\begin{table}[H]
\caption{CRC ����窠 ����� KerasInceptionV3Net}\label{crc-table-80}
\begin{tabular}{|p{8cm} p{8cm}|} 
\hline class &  \\
\multicolumn{2}{|c|}{KerasInceptionV3Net} \\ \hline
\end{tabular}
\begin{tabular}{|p{8cm}|p{8cm}|} 
  ����� ���஭��� ��  & \bdot Label \\
\hline 
\end{tabular}
\end{table}

\begin{table}[H]
\caption{CRC ����窠 ����� Prediction}\label{crc-table-81}
\begin{tabular}{|p{8cm} p{8cm}|} 
\hline class &  \\
\multicolumn{2}{|c|}{Prediction} \\ \hline
\end{tabular}
\begin{tabular}{|p{8cm}|p{8cm}|} 
  �।��������� ���஭��� ��  & \bdot Label \\
\hline 
\end{tabular}
\end{table}

\begin{table}[H]
\caption{CRC ����窠 ����� UserDetailService}\label{crc-table-82}
\begin{tabular}{|p{8cm} p{8cm}|} 
\hline class &  \\
\multicolumn{2}{|c|}{UserDetailService} \\ \hline
\end{tabular}
\begin{tabular}{|p{8cm}|p{8cm}|} 
\multirow{5}{=}{ �࠭�� ���짮��⥫� ⥪�饩 ��ᨨ } 
& \bdot CurrentUser \\
& \bdot UserRepository \\
& \bdot Privilege \\
& \bdot Role \\
& \bdot UserAccount \\
\hline 
\end{tabular}
\end{table}

\begin{table}[H]
\caption{CRC ����窠 ����� MetricRegistrySingleton}\label{crc-table-83}
\begin{tabular}{|p{8cm} p{8cm}|} 
\hline class &  \\
\multicolumn{2}{|c|}{MetricRegistrySingleton} \\ \hline
\end{tabular}
\begin{tabular}{|p{8cm}|p{8cm}|} 
  �뢮��� ���ଠ�� � ⥪��� ����� � ���  & \\
\hline 
\end{tabular}
\end{table}

\begin{table}[H]
\caption{CRC ����窠 ����� FileStorageException}\label{crc-table-84}
\begin{tabular}{|p{8cm} p{8cm}|} 
\hline class & RuntimeException \\
\multicolumn{2}{|c|}{FileStorageException} \\ \hline
\end{tabular}
\begin{tabular}{|p{8cm}|p{8cm}|} 
  �᪫�祭�� � 䠩����� �࠭����  & \\
\hline 
\end{tabular}
\end{table}

\begin{table}[H]
\caption{CRC ����窠 ����� MyFileNotFoundException}\label{crc-table-85}
\begin{tabular}{|p{8cm} p{8cm}|} 
\hline class & RuntimeException \\
\multicolumn{2}{|c|}{MyFileNotFoundException} \\ \hline
\end{tabular}
\begin{tabular}{|p{8cm}|p{8cm}|} 
   �᪫�祭�� �� ���������饬 䠩��. �����頥� ��� 404  & \\
\hline 
\end{tabular}
\end{table}

\begin{table}[H]
\caption{CRC ����窠 ����� CurrentUser}\label{crc-table-86}
\begin{tabular}{|p{8cm} p{8cm}|} 
\hline class & User \\
\multicolumn{2}{|c|}{CurrentUser} \\ \hline
\end{tabular}
\begin{tabular}{|p{8cm}|p{8cm}|} 
  ����� ���짮��⥫�, �易���� � ���짮��⥫�� �� �� ��� ��ᨨ  & \bdot UserAccount \\
\hline 
\end{tabular}
\end{table}

\begin{table}[H]
\caption{CRC ����窠 ����� UserValidator}\label{crc-table-87}
\begin{tabular}{|p{8cm} p{8cm}|} 
\hline class & Validator \\
\multicolumn{2}{|c|}{UserValidator} \\ \hline
\end{tabular}
\begin{tabular}{|p{8cm}|p{8cm}|} 
  �������� ���짮��⥫� �� ॣ����樨  & \bdot UserAccount \\
\hline 
\end{tabular}
\end{table}

\begin{table}[H]
\caption{CRC ����窠 ����� ValidPassword}\label{crc-table-88}
\begin{tabular}{|p{8cm} p{8cm}|} 
\hline interface &  \\
\multicolumn{2}{|c|}{ValidPassword} \\ \hline
\end{tabular}
\begin{tabular}{|p{8cm}|p{8cm}|} 
  �������� ��஫�  & \bdot UserAccount \\
\hline 
\end{tabular}
\end{table}

\begin{table}[H]
\caption{CRC ����窠 ����� EmailValidator}\label{crc-table-89}
\begin{tabular}{|p{8cm} p{8cm}|} 
\hline class & ConstraintValidator \\
\multicolumn{2}{|c|}{EmailValidator} \\ \hline
\end{tabular}
\begin{tabular}{|p{8cm}|p{8cm}|} 
  �������� email  & \bdot UserAccount \\
\hline 
\end{tabular}
\end{table}

\begin{table}[H]
\caption{CRC ����窠 ����� PasswordMatches}\label{crc-table-90}
\begin{tabular}{|p{8cm} p{8cm}|} 
\hline interface &  \\
\multicolumn{2}{|c|}{PasswordMatches} \\ \hline
\end{tabular}
\begin{tabular}{|p{8cm}|p{8cm}|} 
  ����� ��� ��� ����୮�� ����� ��஫��  & \bdot UserAccount \\
\hline 
\end{tabular}
\end{table}

\begin{table}[H]
\caption{CRC ����窠 ����� ValidEmail}\label{crc-table-91}
\begin{tabular}{|p{8cm} p{8cm}|} 
\hline interface &  \\
\multicolumn{2}{|c|}{ValidEmail} \\ \hline
\end{tabular}
\begin{tabular}{|p{8cm}|p{8cm}|} 
  �������� email  & \bdot UserAccount \\
\hline 
\end{tabular}
\end{table}

\begin{table}[H]
\caption{CRC ����窠 ����� EmailExistsException}\label{crc-table-92}
\begin{tabular}{|p{8cm} p{8cm}|} 
\hline class & Throwable \\
\multicolumn{2}{|c|}{EmailExistsException} \\ \hline
\end{tabular}
\begin{tabular}{|p{8cm}|p{8cm}|} 
  �᪫�祭�� �� 㦥 �������饬 ���짮��⥫� � ����� email  & \\
\hline 
\end{tabular}
\end{table}

\begin{table}[H]
\caption{CRC ����窠 ����� PasswordMatchesValidator}\label{crc-table-93}
\begin{tabular}{|p{8cm} p{8cm}|} 
\hline class & ConstraintValidator \\
\multicolumn{2}{|c|}{PasswordMatchesValidator} \\ \hline
\end{tabular}
\begin{tabular}{|p{8cm}|p{8cm}|} 
\multirow{2}{=}{ �������� ��� ��� ����୮�� ����� �࠮�� } 
& \bdot UserDto \\
& \bdot PasswordMatches \\
\hline 
\end{tabular}
\end{table}

\begin{table}[H]
\caption{CRC ����窠 ����� PasswordConstraintValidator}\label{crc-table-94}
\begin{tabular}{|p{8cm} p{8cm}|} 
\hline class & ConstraintValidator \\
\multicolumn{2}{|c|}{PasswordConstraintValidator} \\ \hline
\end{tabular}
\begin{tabular}{|p{8cm}|p{8cm}|} 
  �������� ��஫��  & \bdot PasswordMatches \\
\hline 
\end{tabular}
\end{table}

\begin{table}[H]
\caption{CRC ����窠 ����� ServiceConfig}\label{crc-table-95}
\begin{tabular}{|p{8cm} p{8cm}|} 
\hline class &  \\
\multicolumn{2}{|c|}{ServiceConfig} \\ \hline
\end{tabular}
\begin{tabular}{|p{8cm}|p{8cm}|} 
  ���䨣��樮��� 䠩� ��� �ࢨᮢ  & \\
\hline 
\end{tabular}
\end{table}

\begin{table}[H]
\caption{CRC ����窠 ����� SecSecurityConfig}\label{crc-table-96}
\begin{tabular}{|p{8cm} p{8cm}|} 
\hline class & WebSecurityConfigurerAdapter \\
\multicolumn{2}{|c|}{SecSecurityConfig} \\ \hline
\end{tabular}
\begin{tabular}{|p{8cm}|p{8cm}|} 
  ���䨣��樮��� 䠩� Spring security  & \bdot PasswordMatches \\
\hline 
\end{tabular}
\end{table}

\begin{table}[H]
\caption{CRC ����窠 ����� MvcConfig}\label{crc-table-97}
\begin{tabular}{|p{8cm} p{8cm}|} 
\hline class & WebMvcConfigurer \\
\multicolumn{2}{|c|}{MvcConfig} \\ \hline
\end{tabular}
\begin{tabular}{|p{8cm}|p{8cm}|} 
\multirow{2}{=}{ ���䨣��樮��� 䠩� MVC spring } 
& \bdot EmailValidator \\
& \bdot PasswordMatchesValidator \\
\hline 
\end{tabular}
\end{table}

\begin{table}[H]
\caption{CRC ����窠 ����� SetupDataLoader}\label{crc-table-last}
\begin{tabular}{|p{8cm} p{8cm}|} 
\hline class & ApplicationListener \\
\multicolumn{2}{|c|}{SetupDataLoader} \\ \hline
\end{tabular}
\begin{tabular}{|p{8cm}|p{8cm}|} 
\multirow{6}{=}{ ���樠������ ������ �� ����᪥ �ࢨ� } 
& \bdot PrivilegeRepository \\
& \bdot RoleRepository \\
& \bdot UserRepository \\
& \bdot Privilege \\
& \bdot Role \\
& \bdot UserAccount \\
\hline 
\end{tabular}
\end{table}


\subsection{Структура базы данных}
База данных проектируемого сервиса организована в системе управления базами данных PostgreSQL. 
В таблице \ref{database-table} приведён перечень таблиц базы данных. 
Спецификация структуры базы данных представлена в таблицах \ref{database-table-camera}-\ref{database-table-photo-group-user-like}
На рисунке \ref{database-structure-1} представлена модель базы данных программного решения.

\begin{table}[H]
  \caption{\onehalfspacing Таблицы базы данных}\label{database-table}
  \begin{tabular}{|p{6cm}|p{6cm}|p{4cm}|}
  \hline Название таблицы & Описание & Тип \\
  \hline camera & Справочник моделей фотокамер & ImageHostingDB \\
  \hline photo_group_metadata & Метаданные фотографии & ImageHostingDB \\
  \hline photo_group_color_area & Индексированные данные о цвете на фотографии & ImageHostingDB \\
  \hline photo_group_post & Связь группы фотографий и поста & ImageHostingDB \\
  \hline photo_single & Фотография конкретного назначения & ImageHostingDB \\
  \hline photo_group_user_like & Бинарная оценка группы фотографий пользователем & ImageHostingDB \\
  \hline world_net_class & Один класс из классификации WordNetId & ImageHostingDB \\
  \hline tag & Ключевое слово, связывающее несколько объектов общими признаками & ImageHostingDB \\
  \hline photo_group_tag & Связь группы фотографий и тегов & ImageHostingDB \\
  \hline photo_group & Совокупность фотографий одного содержания, но разных назначений & ImageHostingDB \\
  \end{tabular}
\end{table}
\begin{table}[H]
  \caption*{\onehalfspacing Окончание таблицы \ref{database-table}}
  \begin{tabular}{|p{6cm}|p{6cm}|p{4cm}|}
  \hline Название таблицы & Описание & Тип \\
  \hline post & Совокупность одной или нескольких фотографий и теста, опубликованных в сервисе  & ImageHostingDB \\
  \hline post_tag & Связь постов и тегов & ImageHostingDB \\
  \hline photo_group_user_access & Лист доступа к группе фотографий & ImageHostingDB \\
  \hline comment & Комментарий пользователя к группе фотографий & ImageHostingDB \\
  \hline post_user_like & Бинарная оценка поста пользователем & ImageHostingDB \\
  \hline user_auth_cookie & Аутентификационные ключи доступа браузеров пользователей & ImageHostingDB \\
  \hline user_account & Аккаунт пользователя & ImageHostingDB \\
  \hline password_reset_token & Токен для сброса пароля & ImageHostingDB \\
  \hline privelege & Справочник привелегий ролей & ImageHostingDB \\
  \hline roles_priveleges & Привелегии ролей & ImageHostingDB \\
  \hline role & Справочник ролей пользователей & ImageHostingDB \\
  \hline user_roles & Роли пользователей & ImageHostingDB \\
  \hline verification_token & Токены подтверждения аккаунтов пользователей & ImageHostingDB \\
  \hline
  \end{tabular}
\end{table}

\addimghere{database-structure-1}{1}{Модель базы данных программного решения для хранения фотографий}{database-structure-1}

\begin{table}[H]
  \caption{\onehalfspacing Описание структуры таблицы camera}\label{database-table-camera}
  \begin{tabular}{|p{6cm}|p{6cm}|p{4cm}|}
  \hline Название атрибута & Описание & Тип \\
  \hline id & Уникальный идентификатор & int8 \\
  \hline name & Наименование камеры & varchar(255) \\
  \hline
  \end{tabular}
\end{table}

\begin{table}[H]
  \caption{\onehalfspacing Описание структуры таблицы photo_group_metadata}\label{database-table-photo-group-metadata}
  \begin{tabular}{|p{6cm}|p{6cm}|p{4cm}|}
  \hline Название атрибута & Описание & Тип \\
  \hline id & Уникальный идентификатор & int8 \\
  \hline camera_id & Идентификатор камеры & int8 \\
  \hline photo_group_id & Идентификатор группы фотографий & int8 \\
  \hline
  \end{tabular}
\end{table}

\begin{table}[H]
  \caption{\onehalfspacing Описание структуры таблицы world_net_class}\label{database-table-world-net-class}
  \begin{tabular}{|p{6cm}|p{6cm}|p{4cm}|}
  \hline Название атрибута & Описание & Тип \\
  \hline id & Уникальный идентификатор & int8 \\
  \hline description & Описание класса & varchar(255) \\
  \hline parent_id & Идентификатор родительского класса & int8 \\
  \hline 
  \end{tabular}
\end{table}

\begin{table}[H]
  \caption{\onehalfspacing Описание структуры таблицы post}\label{database-table-post}
  \begin{tabular}{|p{6cm}|p{6cm}|p{4cm}|}
  \hline Название атрибута & Описание & Тип \\
  \hline id & Уникальный идентификатор & int8 \\
  \hline privacy_type & Приватность поста & int4 \\
  \hline text & Текст поста & varchar(255) \\
  \hline user_id & Уникальный идентификатор пользователя & int8 \\
  \hline 
  \end{tabular}
\end{table}

\begin{table}[H]
  \caption{\onehalfspacing Описание структуры таблицы post_user_like}\label{database-table-post-user-like}
  \begin{tabular}{|p{6cm}|p{6cm}|p{4cm}|}
  \hline Название атрибута & Описание & Тип \\
  \hline id & Уникальный идентификатор & int8 \\
  \hline post_id & Идентификатор поста & int8 \\
  \hline user_id & Идентификатор пользователя\\
  \hline 
  \end{tabular}
\end{table}

\begin{table}[H]
  \caption{\onehalfspacing Описание структуры таблицы privelege}\label{database-table-privelege}
  \begin{tabular}{|p{6cm}|p{6cm}|p{4cm}|}
  \hline Название атрибута & Описание & Тип \\
  \hline id & Уникальный идентификатор & int8 \\
  \hline name & Наименование привелегии & varchar(255) \\
  \hline 
  \end{tabular}
\end{table}

\begin{table}[H]
  \caption{\onehalfspacing Описание структуры таблицы roles_priveleges}\label{database-table-roles-priveleges}
  \begin{tabular}{|p{6cm}|p{6cm}|p{4cm}|}
  \hline Название атрибута & Описание & Тип \\
  \hline id & Уникальный идентификатор & int8 \\
  \hline role_id & Идентификатор роли & int8 \\
  \hline privelege_id Идентификатор привелегии & int8 \\
  \hline 
  \end{tabular}
\end{table}

\begin{table}[H]
  \caption{\onehalfspacing Описание структуры таблицы role}\label{database-table-role}
  \begin{tabular}{|p{6cm}|p{6cm}|p{4cm}|}
  \hline Название атрибута & Описание & Тип \\
  \hline id & Уникальный идентификатор & int8 \\
  \hline name & Наименование роли & varchar(255) \\
  \hline
  \end{tabular}
\end{table}

\begin{table}[H]
  \caption{\onehalfspacing Описание структуры таблицы user_auth_cookie}\label{database-table-user-auth-cookie}
  \begin{tabular}{|p{6cm}|p{6cm}|p{4cm}|}
  \hline Название атрибута & Описание & Тип \\
  \hline id & Уникальный идентификатор & int8 \\
  \hline cookie & Содержимое куки & varchar(255) \\
  \hline expires & Дата и время истечения срока действия & timestamp(6) \\
  \hline user_id & Идентификатор пользователя & int8 \\
  \hline
  \end{tabular}
\end{table}

\begin{table}[H]
  \caption{\onehalfspacing Описание структуры таблицы post_tag}\label{database-table-post-tag}
  \begin{tabular}{|p{6cm}|p{6cm}|p{4cm}|}
  \hline Название атрибута & Описание & Тип \\
  \hline id & Уникальный идентификатор & int8 \\
  \hline post_id & Идентификатор поста & int8 \\
  \hline tag_id & Идентификатор тега & int8 \\
  \hline
  \end{tabular}
\end{table}

\begin{table}[H]
  \caption{\onehalfspacing Описание структуры таблицы tag}\label{database-table-tag}
  \begin{tabular}{|p{6cm}|p{6cm}|p{4cm}|}
  \hline Название атрибута & Описание & Тип \\
  \hline id & Уникальный идентификатор & int8 \\
  \hline name & Наименование тега & varchar(255) \\
  \hline world_net_class_id & Идентификатор класса & int8\\
  \hline 
  \end{tabular}
\end{table}

\begin{table}[H]
  \caption{\onehalfspacing Описание структуры таблицы photo_group_color_area}\label{database-table-photo-group-color-area}
  \begin{tabular}{|p{6cm}|p{6cm}|p{4cm}|}
  \hline Название атрибута & Описание & Тип \\
  \hline id & Уникальный идентификатор & int8 \\
  \hline blue & Значение синего канала цвета & int4 \\
  \hline red & Значение красного канала цвета & int4 \\
  \hline green & Значение зеленого канала цвета & int4 \\
  \hline rgb & Значение трёх каналов цветов & int4 \\
  \hline photo_group_id & Идентификатор группы фотографий & int8 \\
  \hline cluster_size & Размер кластера & int4 \\
  \hline hex_color & hex представление значения цвета & varchar(255) \\
  \hline 
  \end{tabular}
\end{table}

\begin{table}[H]
  \caption{\onehalfspacing Описание структуры таблицы photo_group_post}\label{database-table-photo-group-post}
  \begin{tabular}{|p{6cm}|p{6cm}|p{4cm}|}
  \hline Название атрибута & Описание & Тип \\
  \hline id & Уникальный идентификатор & int8 \\
  \hline photo_group_id & Идентификатор группы фотографий & int8 \\
  \hline post_id & Идентификатор поста & int8 \\
  \hline 
  \end{tabular}
\end{table}

\begin{table}[H]
  \caption{\onehalfspacing Описание структуры таблицы photo_group_tag}\label{database-table-photo-group-tag}
  \begin{tabular}{|p{6cm}|p{6cm}|p{4cm}|}
  \hline Название атрибута & Описание & Тип \\
  \hline id & Уникальный идентификатор & int8 \\
  \hline probability & Вероятность принадлежности к классу & int4 \\
  \hline photo_group_id & Идентификатор группы фотографий & int8 \\
  \hline tag_id & Уникальный идентификатор & int8 \\
  \hline
  \end{tabular}
\end{table}

\begin{table}[H]
  \caption{\onehalfspacing Описание структуры таблицы photo_group_user_access}\label{database-table-photo-group-user-access}
  \begin{tabular}{|p{6cm}|p{6cm}|p{4cm}|}
  \hline Название атрибута & Описание & Тип \\
  \hline id & Уникальный идентификатор & int8 \\
  \hline photo_group_id & Идентификатор группы фотографий & int8 \\
  \hline user_id & Идентификатор пользователя & int8 \\
  \hline 
  \end{tabular}
\end{table}

\begin{table}[H]
  \caption{\onehalfspacing Описание структуры таблицы user_account}\label{database-table-user-account}
  \begin{tabular}{|p{6cm}|p{6cm}|p{4cm}|}
  \hline Название атрибута & Описание & Тип \\
  \hline id & Уникальный идентификатор & int8 \\
  \hline email & Адрес почтового ящика пользователя & varchar(255) \\
  \hline is_enabled & Флаг активности пользователя & bool \\
  \hline is_using2fa & Флаг использования двухфакторинговой авторизации & bool \\
  \hline password & Хэш пароля пользователя & varchar(255) \\
  \hline secret & Ответ на секретный вопрос для восстановления аккаунта & varchar(255) \\
  \hline username & Имя пользователя & varchar(255) \\
  \hline 
  \end{tabular}
\end{table}


\begin{table}[H]
  \caption{\onehalfspacing Описание структуры таблицы user_roles}\label{database-table-user-roles}
  \begin{tabular}{|p{6cm}|p{6cm}|p{4cm}|}
  \hline Название атрибута & Описание & Тип \\
  \hline user_id & Идентификатор пользователя & int8 \\
  \hline role_id & Идентификатор пользователя & int8 \\
  \hline 
  \end{tabular}
\end{table}


\begin{table}[H]
  \caption{\onehalfspacing Описание структуры таблицы verification_token}\label{database-table-verification-token}
  \begin{tabular}{|p{6cm}|p{6cm}|p{4cm}|}
  \hline Название атрибута & Описание & Тип \\
  \hline id & Уникальный идентификатор & int8 \\
  \hline expiry_date & Дата и время истечения срока действия & timestamp(6) \\
  \hline token & Токен & varchar(255) \\
  \hline user_id & Идентификатор пользователя & int8 \\
  \hline
  \end{tabular}
\end{table}

\begin{table}[H]
  \caption{\onehalfspacing Описание структуры таблицы password_reset_token}\label{database-table-password-reset-token}
  \begin{tabular}{|p{6cm}|p{6cm}|p{4cm}|}
  \hline Название атрибута & Описание & Тип \\
  \hline id & Уникальный идентификатор & int8 \\
  \hline expiry_date & Дата и время истечения срока действия & timestamp(6) \\
  \hline token & Токен & varchar(255) \\
  \hline user_id & Идентификатор пользователя & int8 \\
  \hline
  \end{tabular}
\end{table}


\begin{table}[H]
  \caption{\onehalfspacing Описание структуры таблицы comment}\label{database-table-comment}
  \begin{tabular}{|p{6cm}|p{6cm}|p{4cm}|}
  \hline Название атрибута & Описание & Тип \\
  \hline id & Уникальный идентификатор & int8 \\
  \hline text & Текст комментария & varchar(255) \\
  \hline parent_comment_id & Идентификатор родительского комментария & int8 \\
  \hline photo_group_id & Идентификатор комментируемой группы фотографий & int8 \\
  \hline user_id & Идентификатор пользователя, оставившего комментарий & int8 \\
  \hline 
  \end{tabular}
\end{table}


\begin{table}[H]
  \caption{\onehalfspacing Описание структуры таблицы photo_group}\label{database-table-photo-group}
  \begin{tabular}{|p{6cm}|p{6cm}|p{4cm}|}
  \hline Название атрибута & Описание & Тип \\
  \hline id & Уникальный идентификатор & int8 \\
  \hline error_text & Текст ошибки, возникшей в ходе обработки & varchar(255) \\
  \hline has_preview & Уникальный идентификатор & bool \\
  \hline is_clustered & Флаг кластеризованности группы фотографий & bool \\
  \hline is_erroneous & Флаг наличия ошибки в группе фотографий & bool \\
  \end{tabular}
\end{table}

\begin{table}[H]
  \caption*{\onehalfspacing Окончание таблицы \ref{database-table-photo-group}}
  \begin{tabular}{|p{6cm}|p{6cm}|p{4cm}|}
  \hline Название атрибута & Описание & Тип \\
  \hline is_metadata_extracted & Флаг извлеченности метаданных в группе фотографий & bool \\
  \hline is_tagged & Флаг тегированности группы фотографий & bool \\
  \hline privacy_type & Тип приватности группы фотографий & int4 \\
  \hline uuid & Уникальный идентификатор для обращения из веб интерфейса & varchar(255) \\
  \hline user_id & Идентификатор пользователя & int8 \\
  \hline
  \end{tabular}
\end{table}

\begin{table}[H]
  \caption{\onehalfspacing Описание структуры таблицы photo_single}\label{database-table-photo-single}
  \begin{tabular}{|p{6cm}|p{6cm}|p{4cm}|}
  \hline Название атрибута & Описание & Тип \\
  \hline id & Уникальный идентификатор & int8 \\
  \hline is_erroneous & Флаг наличия ошибки в фотографии & bool \\
  \hline path & Путь к фотографии на файловом хранилище & varchar(255) \\
  \hline photo_type & Тип фотографии & int5 \\
  \hline photo_group_id & Идентификатор группы & int8 \\
  \hline
  \end{tabular}
\end{table}

\begin{table}[H]
  \caption{\onehalfspacing Описание структуры таблицы photo_group_user_like}\label{database-table-photo-group-user-like}
  \begin{tabular}{|p{6cm}|p{6cm}|p{4cm}|}
  \hline Название атрибута & Описание & Тип \\
  \hline id & Уникальный идентификатор & int8 \\
  \hline photo_group_id & Идентификатор группы фотографий & int8 \\
  \hline user_id & Идентификатор пользователя\\
  \hline 
  \end{tabular}
\end{table}

\clearpage

\section{Реализация системы}
\subsection{Описание разработанных модулей}
\begin{table}[H]
  \caption{\onehalfspacing Модули онлайн-сервиса с элементами социальной сети для публикации фотографий}\label{user-classes-table}
  \begin{tabular}{|p{4cm}|p{12cm}|}
  \hline Название модуля & Описание \\ 
  \hline authentication & Классы контроллеров аутентификации \\ 
  \hline exception & Классы исключений \\ 
  \hline mapper & Классы блоков отображений \\ 
  \hline monitoring & Классы отслеживания данных приложения \\ 
  \hline persistence & Классы для работы с базой данных \\ 
  \hline property & Классы настроек сервиса \\ 
  \hline registration & Классы контроллеров регистрации \\ 
  \hline service & Классы сервисов \\ 
  \hline Название модуля & Описание \\ 
  \hline spring & Классы настроек spring \\ 
  \hline util & Классы инструментов \\ 
  \hline validation & Классы валидации \\ 
  \hline web & Классы контроллеры сервиса, отвечающие на входящие запросы \\ 
  \hline resources.template & Модуль, содержащий описание графического интерфейса приложения \\ 
  \hline
  \end{tabular}
\end{table}


\subsection{Описание схемы разделения данных приложения.}

Приложение разрабатывалось с использованием схемы разделения данных модель-представление-контроллер.

Модель представляют классы предметной области, являющиеся отображением таблиц базы данных и находящиеся в пакете persistence.
Представление определяют классы, находящиеся в модуле resources.template.
Контроллеры разделены на два слоя: контроллеры запросов, находящиеся в модуле web и сервисы, находящиеся в модуле service.

Контроллеры запросов принимают входящие запросы, преобразуют и формируют запросы к сервисам, которые в свою очередь работают с объектами предметной области и содержат бизнес-логику приложения.
Готовые ответы от сервисов в виде совокупности объектов предметной области преобразуются контроллерами с помощью блоков отображений в корректные ответы для клиентов, посылающих запросы и отправляют им ответы на посланные ими запросы.

Примененная схема разделения данных позволяет отделить бизнес-логику приложения от запросов и визуализации и выделить классы отображающие таблицы базы данных.
Данный вид разделения способствует повторному использованию кода и облегчает процесс сопровождения приложения.

\clearpage