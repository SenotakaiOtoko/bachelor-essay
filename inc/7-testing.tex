\section{Результаты тестирования}

В режиме нормального функционирования мониторинг показывает полную работоспособность всех сервисов системы (рис. \ref{nagios-ok}).
\addimghere{nagios-ok}{1}{Работоспособность сервисов в штатном режиме}{nagios-ok}

В случае появления внештатных ситуаций, система мониторинга фиксирует аномальные явления, такие как недоступность хостов или отдельных сервисов.
Мониторинг способен запоминать код ошибки, поэтому диагностика неисправностей является несложной задачей.
Пример реакции системного мониторинга на недоступность сервера представлен на рис. \ref{nagios-nok}.
\addimghere{nagios-nok}{1}{Недоступность некоторых серверов инфраструктуры}{nagios-nok}

Пример реакции системы мониторинга Nagios на неисправность одного из дисков в RAID-массиве представлен на рис. \ref{nagios-raid-error}.
\addimghere{nagios-raid-error}{1}{Сбой работы диска в RAID-массиве}{nagios-raid-error}

\subsection{Нагрузка сети во время резервного копирования}

Каждую ночь на серверах запускается задача резервного копирования, в это время наблюдается повышенная нагрузка на сервере, связанная со снятием дампов баз данных, архивацией и сжатием архивов для передачи по сети на сервер резервного копирования.

Как правило повышение нагрузки незначительное, что можно наблюдать на рис. \ref{top-ok}.
\addimghere{top-ok}{1}{Состояние системы во время процесса резервного копирования}{top-ok}

На графике Munin можно наблюдать значительное повышение утилизации сетевого канала во время резервного копирования.
График представлен на рис. \ref{munin-backup}.
Значения представленные на графике являются условными, сетевого канала в 10 Мб/с явно недостаточно для такой инфраструктуры.
На деле используется сеть с пропускной способностью 100 Мб/с.
\addimghere{munin-backup}{1}{Состояние системы во время процесса резервного копирования}{munin-backup}

На графике представлено состояние канала сети за последний месяц, пиками на графике являются значения максимальной пропускной способности сети.

\subsection{Состояние сервера в период атаки на отказ}

Можно выделить три основных типа DDoS атак:
\begin{itemize}
  \item переполнение сетевого канала;
  \item большое количество одновременных подключений (syn-flood);
  \item исчерпание процессорных мощностей сервера (HTTP-flood).
\end{itemize}

От последних двух типов атак можно защищаться средствами операционной системы.
В случае переполнения сетевого канала, решением проблемы является расширение сетевого канала.

Состояние нагрузки системы во время DDoS-атаки можно пронаблюдать на рис. \ref{munin-la}.
\addimghere{munin-la}{1}{Значительное повышение нагрузки в период 0:00}{munin-la}

В данный промежуток времени нагрузка составила 40 LA, что превышает среднюю нагрузку на используемом сервере в 20 раз.
При этом, в процессах системы наблюдается явная нагрузка на сервер баз данных (рис. \ref{htop-ddos}).
\addimghere{htop-ddos}{1}{Все процессорные ядра заняты обработкой запросов}{htop-ddos}

На графиках соединений с брандмауэром также наблюдается аномальное явление (рис. \ref{munin-ddos}).
\addimghere{munin-ddos}{1}{Превышено число соединений к брандмауэру в 40 раз}{munin-ddos}

Если взглянуть на вывод команды netstat, то можно заметить наиболее активные адреса:
\begin{lstlisting}
# netstat -ntu | awk '{print $5}' | cut -d: -f1 | sort | \
uniq -c | sort -rn | grep -v 127.0.0.1 | head
  22373 209.72.209.54
  17766 88.81.32.224
  16060 176.209.90.198
  13226 217.209.214.248
    221 62.163.78.143
    221 188.255.255.177
    220 198.27.82.153
    118 216.242.148.101
    113 95.216.88.50
    111 216.73.216.251
\end{lstlisting}
\subsection{Отказ одного из DNS-серверов}
Nagios зафиксировал недоступность одного из DNS-серверов, неспособного осуществлять обработку запросов (рис. \ref{dns-nok}).
\addimghere{dns-nok}{1}{Недоступность DNS-сервера}{dns-nok}
При этом все сайты доступны и работают в штатном режиме, так как отсутствие одного из трех серверов не влияет на работоспособность всей инфраструктуры.
Один из подчиненных серверов все еще доступен и способен обрабатывать пользовательские DNS-запросы.
После возобновления работоспособности сервера, оба подчиненных сервера функционируют в штатном режиме:
\begin{lstlisting}
$ host -t ns site.ru
site.ru name server ns1.site.ru.
site.ru name server ns2.site.ru.
\end{lstlisting}

Таким образом, в ходе проектирования и тестирования инфраструктуры удалось реализовать отказоустойчивую систему, способную оперативно обнаруживать неисправности в инфраструктуре.
Также немаловажным фактором является реакция команды системных администраторов и технической поддержки дата-центра, в условиях критических ситуация важно проявить особую внимательность в решении проблем.

\clearpage