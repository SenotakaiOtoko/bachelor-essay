\anonsection{Заключение}

В ходе выполнения выпускной квалификационной работы бакалавра была разработана виртуальная инфраструктура для реализации облачных услуг.
Данная инфраструктура успешно используется в бизнесе и обладает высоким уровнем отказоустойчивости, надежности и масштабируемости.

Инфраструктура прошла все этапы проектирования и разработки, начиная от закупки серверов и лицензий на программное обеспечение до окончательного тестирования.
Тестирование показало, что при отказе некоторых компонентов инфраструктуры, включая управляющие сервера, система все еще продолжает функционировать.

Продолжительное время находятся в эксплуатации различные системы виртуализации, такие как OpenVZ и KVM.
Сочетание работы этих систем в связке с программным обеспечением, ориентированным на пользователей позволяет организовать продажу услуг в сфере облачных вычислений.

В ходе разработки тарифов для пользователей были подобраны оптимальные параметры контейнеров, виртуальных машин и лимитов хостинга.
Использование свободного программного обеспечения позволило сократить материальные расходы на ПО.

Данная виртуальная инфраструктура может быть расширена за счет использования других систем виртуализаций, таких как Xen, добавления дополнительных обслуживающих серверов, а также улучшение отказоустойчивости за счет применения репликации данных.
Обновление аппаратной части инфраструктуры также увеличит производительность, однако данные действия требуют существенных материальных затрат.

При анализе безопасности и применении мер по ее обеспечении были использованы и дописаны такие скрипты как DDoS Deflate, который является эффективным решением защиты от небольших DDoS-атак.
Скрипт не является законченным продуктом и распространяется под свободной лицензией.

\clearpage