\begingroup 
\renewcommand{\section}[2]{\anonsection{Библиографический список}}
\begin{thebibliography}{00}

\bibitem{oracle-db}
    Каталог программных продуктов семейства Oracle Database
    [Электронный ресурс] //
    Oracle Россия
    URL: http://oracle.ocs.ru/files/catalog\_Oracle\_Database\_12C.pdf
    (дата обращения: 13.04.2015)

\bibitem{dc-tier}
    Tier: уровни надежности ЦОД и что из этого следует
    [Электронный ресурс] //
    AboutDC. Решения для ЦОД
    URL: http://www.aboutdc.ru/page/390.php
    (дата обращения: 14.04.2015)

\bibitem{ibm-virt}
    Джонс Т.М.
    Виртуальный Linux. Обзор методов виртуализации, архитектур и реализаций
    [Электронный ресурс] //
    IBM developerWorks Россия: Ресурс IBM для разработчиков и IT профессионалов
    URL: http://www.ibm.com/developerworks/ru/library/l-linuxvirt/index.html
    (дата обращения: 14.04.2015)

\bibitem{openvz-tutorial}
    Умеров А.Р.
    Руководство по созданию и управлению контейнерами на базе OpenVZ 
    [Электронный ресурс] //
    GitHub
    URL: https://github.com/Amet13/openvz-tutorial/blob/master/main.pdf
    (дата обращения: 15.04.2015)

\bibitem{qemu-ibm}
    Джонс Т.М.
    Эмуляция систем с помощью QEMU
    [Электронный ресурс] //
    IBM developerWorks Россия: Ресурс IBM для разработчиков и IT профессионалов
    URL: http://www.ibm.com/developerworks/ru/library/l-qemu/
    (дата обращения: 16.04.2015)

\bibitem{kvm-ibm}
    Толети Б.П.
    Гипервизоры, виртуализация и облако: Анализ гипервизора KVM
    [Электронный ресурс] //
    IBM developerWorks Россия: Ресурс IBM для разработчиков и IT профессионалов
    Режим доступа: http://www.ibm.com/developerworks/ru/library/cl-hypervisorcompare-kvm/
    (дата обращения: 16.04.2015)

\bibitem{xen-xguru}
    Чубин И.
    Руководство пользователя Xen v3.0
    [Электронный ресурс] //
    Xgu.ru
    URL: http://xgu.ru/xen/manual/
    (дата обращения: 18.04.2015)

\bibitem{lxc-openvz}
    Одинцов П.
    Контейнеризация на Linux в деталях -- LXC и OpenVZ Часть 2
    [Электронный ресурс] //
    Хабрахабр
    URL: http://habrahabr.ru/company/FastVPS/blog/209084/
    (дата обращения: 19.04.2015)

\bibitem{lxc}
    Виртуализация на уровне ОС: теория и практика LXC
    [Электронный ресурс] //
    creativeyp.com
    URL: http://creativeyp.com/568-virtualizaciya-na-urovne-os-teoriya-i-praktika-lxc.html
    (дата обращения: 20.04.2015)

\bibitem{virt-infrast}
    Основы виртуализации. Инфраструктура
    [Электронный ресурс] //
    BW-IT.RU | Информационные технологии
    URL: http://www.bw-it.ru/virtualization\_virtual\_infrastructure.php
    (дата обращения: 20.04.2015)

\bibitem{sys-analyz}
    Методические указания <<Процедура системного анализа при проектировании программных систем>>
    для студентов-дипломников дневной и заочной формы обучения специальности 7.091501 /
    Сост.: Сергеев Г.Г., Скатков А.В., Мащенко Е.Н. -- Севастополь:
    Изд-во СевНТУ, 2005. -- 32с.

\bibitem{unix-handbook}
    Немет Э., Снайдер Г., Хейн Т., Уэйли Б.
    Unix и Linux: руководство системного администратора,
    4-е изд.: Пер. с англ. -- М.: ООО <<И.Д. Вильямс>>, 2012. --
    1312 с.: ил.

\bibitem{selectel}
    Балансировка нагрузки: основные алгоритмы и методы
    [Электронный ресурс] //
    Блог компании Селектел
    URL: http://blog.selectel.ru/balansirovka-nagruzki-osnovnye-algoritmy-i-metody/
    (дата обращения: 06.05.2015)

\bibitem{oot}
    Методические указания для выполнения раздела <<Охрана труда и окружающей среды>>
    в дипломних проектах специальностей
    7.080401 --- <<Информационные и управляющие системы и технологии>>,
    7.092502 --- <<Компьютерно-интегрированные технологические процессы и производства>>,
    7.091401 --- <<Системы управления автоматики>>,
    7.091501 --- <<Компьютерные системы и сети>> / Сост.: Е.И Азаренко. -- Севастополь:
    Изд-во СевНТУ, 2005. -- 10с.

\end{thebibliography}
\endgroup

\clearpage