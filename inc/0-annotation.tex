\annotation{Реферат}

Пояснительная записка содержит \pageref{LastPage} страниц, 119 рисунков, 25 таблиц 13 источников.
Программные компоненты содержат 4794 строки кода 23 таблицы в базе данных, 99 классов.

Цель проекта — обеспечить надёжное хранение фотографий пользователей на удаленном сервере с возможностью индексирования фотографий и осуществления поиска фотографий по хранилищу. 

Изучена предметная область, проанализированы существующие решения, выявлены их недостатки, сформулированы функциональные требования, отличающие программное решение от аналогов с выгодной стороны.

Спроектировано программное средство с учетом поставленных требований.
Разработаны спецификация и архитектура программного средства, в рамках которых были описаны структура базы данных, классы и модули.

Разработан прототип продукта согласно документации. 
Продукт может быть запущен в эксплуатацию при условии дополнительной разработки.
Разработанное решение является удобным инструментом для хранения и организации в автоматическом режиме загруженных фотографий.
Целевой аудиторией являются люди, занимающиеся производством фотоконтента.

Программное решение реализовано с использованием языка программирования Java. Хранение данных осуществляется в базе данных PostgreSQL. Использовались средства разработки: Spring framework, deeplearning4j.

\clearpage